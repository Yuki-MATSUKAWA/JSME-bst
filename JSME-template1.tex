\documentclass[a4paper,fleqn,uplatex]{jsarticle}

%%% テキストエリア指定
\setlength{\hoffset}{-5.4truemm}
\setlength{\voffset}{-10.4truemm}
\setlength{\oddsidemargin}{0truemm}
\setlength{\evensidemargin}{0truemm}
\setlength{\topmargin}{0truemm}
\setlength{\footskip}{10truemm}
\setlength{\textheight}{250truemm}
\setlength{\textwidth}{170truemm}
\setlength{\headsep}{5truept}
\setlength{\headheight}{20truept}
\setlength{\marginparsep}{0truemm}
\setlength{\marginparwidth}{0truemm}

%%% 図の挿入
\usepackage{graphicx}
%%% hyperref
\usepackage[dvipdfmx,bookmarks=true,setpagesize=false,colorlinks,allcolors=blue]{hyperref}
\usepackage{pxjahyper}
%%% ロゴ関連
\usepackage{bxtexlogo}
%%% 数式等
\usepackage{physics}
\usepackage{amsmath,amssymb}
%%% 枠
\usepackage{ascmac}
\usepackage{fancybox}
%%% otf
\usepackage{otf}
%%% フォントを Times に変更
\usepackage{newtxtext,newtxmath}
%%% 色の変更
\usepackage{color}
%%% natbib
\usepackage{natbib}
%%% 参考文献リストの見出しを「参考文献」から「文献」に変更.
\renewcommand{\refname}{文献}

%%% ヘッダー・フッター
\usepackage{fancyhdr}
\usepackage{lastpage}
\fancypagestyle{cover}
{%
	\fancyhf{}
	\cfoot{\thepage/\pageref{LastPage}}
	\renewcommand{\headrulewidth}{0.0pt}
}
\pagestyle{fancy}
	\lhead{【非公式】日本機械学会\BibTeX{}スタイルファイルの使い方}
	\chead{}
	\rhead{東京理科大学 松川裕樹}
	\lfoot{}
	\cfoot{\thepage/\pageref{LastPage}}
	\rfoot{}

\renewcommand{\figurename}{Fig.~}       % 図番号
\renewcommand{\tablename}{Table~}       % 表番号

\title{【非公式】日本機械学会\BibTeX{}スタイルファイル \\ \texttt{JSME-bst} テンプレートその1(標準)}
\author{松川 裕樹\thanks{東京理科大学大学院 創域理工学研究科 機械航空宇宙工学専攻 博士後期課程,Email: \texttt{7523701 _@_ ed.tus.ac.jp}}}

\date{最終更新:\today}

\begin{document}

\maketitle
\thispagestyle{cover}

\tableofcontents

\section{はじめに}
\label{sec:introduction}
\verb|jsme.bst|は日本機械学会\footnote{一般社団法人 日本機械学会(The Japan Society of Mechanical Engineers, JSME),\textless\url{https://www.jsme.or.jp/}\textgreater}の原稿テンプレート\footnote{日本機械学会 原稿テンプレート,\textless\url{https://www.jsme.or.jp/publish/transact/for-authors.html}\textgreater}に基づいた参考文献の出力を実現するために作成した,非公式\BibTeX{}スタイルファイルテンプレートです.
必要なファイル一式はGitHubの\verb|JSME-bst|\footnote{\texttt{JSME-bst}, \textless\url{https://github.com/Yuki-MATSUKAWA/JSME-bst}\textgreater}から入手可能なので,用途に応じて自由に改変してください.
今読んでいるこのファイル(\LaTeX{}ソース:\verb|JSME-template1.tex|,出力結果:\verb|JSME-template1.pdf|)では\BibTeX{}で用意されている全てのエントリの出力結果を表示しています.
参考文献のリストはこの\verb|pdf|の末尾で,出力している文献の\verb|bib|ファイルは英語文献\verb|mybib_en.bib|と日本語文献\verb|mybib_jp.bib|の二つです.
\verb|JSME-bst|の作成者である松川が流体力学,特に乱流遷移の研究をしているため,引用している文献は乱流遷移の周辺のものが多くなっています(全てではありません).
ただ,材料力学など他分野の方でも基本的な使い方は同じです.
また,第~\ref{sec:cite}節でも述べますが,著者数が1名,2名,3名以上のそれぞれで引用時の出力結果が異なります.
そのため\verb|mybib_en.bib|,\verb|mybib_jp.bib|では可能な限り著者数が1名,2名,3名以上の計3パターンを用意しています.
ただ,基本的に実在の文献を集めて載せているので学位論文(\verb|phdthesis|, \verb|masterthesis|)のように原則著者が一人のものや\verb|manual|,\verb|unpublished|など一部のエントリでは全てのパターンを網羅できていない場合もあるのでご了承ください(ただ,参考にするうえで困ることのないくらいにはパターンを網羅しているつもりです).

この文書では\BibTeX{}初心者でも使いやすいよう,\BibTeX{}そのものの使い方や\verb|bib|ファイルの作成方法といった内容も可能な限り説明します.
それでもわからなければさまざまな書籍やwebサイトがあるので参考にしてみてください.
また,その都度説明しますが,日本機械学会の原稿テンプレートに沿ったリストを作成するために\verb|bib|ファイルの作成方法が一部通常と異なる場合があるのでご了承ください.


\section{日本機械学会の原稿執筆要領における注意点}
\label{sec:caution}
この節では日本機械学会の原稿テンプレートに記載されている執筆要領の中でも,特に文献の記載に関する注意点をまとめておきます.
日本機械学会の規定に合わせて論文執筆する際は是非参考にしてください\footnote{\texttt{JSME-bst}自体はもともと,JSMEの規定に沿って卒業論文執筆を行うことが決められている東京理科大学創域理工学部機械航空宇宙工学科の卒研生向けに作成したものです.もちろんGitHub上で公開しているので世界中の誰が使っても構いません.}.

\subsection{使用する書体・フォントについて}
日本機械学会では本文に使用する書体を和文は明朝体,欧文はSerif体と規定されています.
Serif体のフォントとしてはTimes New RomanやCentury,Computer Modernなどが挙げられます.
知らない人が結構多いのですが,\LaTeX{}標準のフォントはComputer Modernであって,Times New Romanではないので注意.
フォントを変える際はプリアンブルで指定します.
この文書の場合はプリアンブルの\verb|\usepackage{newtxtext,newtxmath}|でフォントをTimes New Romanに変更しています.

\subsection{文献の並べ方}
この節では文献リストのフィールド(\BibTeX{}における\verb|author|や\verb|title|)の並べ方について説明します.
\verb|jsme.bst|のユーザーは気にせず使用することができますが,改めて整理しておくので\verb|jsme.bst|を使わないときにも役に立つと思います.
原則,日本機械学会の規定に合わせていますが,原稿テンプレートからは判断が難しい事項は一部独自の解釈を加えているものがあります.

文献を並べる順序の規則は以下の通り.
\begin{itembox}[l]{文献ソート規則}
    \begin{enumerate}
        \item (Family, Givenの順で並べた際の)筆頭著者の氏名のアルファベット順.並べる際,日本人の氏名は漢字と仮名を用いた日本語表記で構わないが順序はアルファベット順とする.
        \item 筆頭著者が同一人物の場合,第二著者以降のアルファベット順で並べる.著者数が異なる場合は著者数が少ない方が先.これを最後の著者まで繰り返す.
        \item 著者が全員一致する文献があった場合は発行が早い順で並べる.
        \item 確認できる範囲で発行年月日が同じだった場合,タイトルのアルファベット順で並べ,西暦の後に小文字でアルファベットを順に振る.
        \begin{quote}
            Matsukawa, Y. and Tsukahara, T., Laminarization in subcritical Taylor--Couette--Poiseuille flow with increasing pressure gradient, Proceedings of Nineteenth International Conference on Flow Dynamics (2022a). \\
            Matsukawa, Y. and Tsukahara, T., Subcritical transition of Taylor--Couette--Poiseuille flow at high radius ratio, \href{https://doi.org/10.1063/5.0096676}{Physics of Fluids}, Vol.~34, No.~7 (2022b), 074109.
        \end{quote}
    \end{enumerate}
\end{itembox}
文献を並べる際の表記に関する注意事項は以下の通り.
\renewcommand\UrlFont{\rmfamily}
\begin{itembox}[l]{文献リスト作成の注意事項(全体)}
    \begin{itemize}
        \item 並び順は原則として \\
            著者(\verb|author|)$\to$タイトル(\verb|title|)$\to$誌名(\verb|journal|)・書名(\verb|booktitle|)$\to$出版社(\verb|publisher|)・大学名(\verb|school|)・機関名(\verb|institution|)$\to$巻(\verb|volume|)$\to$号(\verb|number|)$\to$発行年(\verb|year|)$\to$ページ(\verb|pages|)$\to$論文番号・講演番号(\verb|note|) \\
            とする.
            ただし,存在しないフィールドがある場合は抜かす.
            また,webページ等の引用は極力避けるべきだが,引用する場合には必ず末尾にURLと参照日を明記する(第~\ref{ssec:online}節\verb|@online|を参照).
            \begin{quote}
                Kawamura Laboratory, DNS database of wall turbulence and heat transfer: Text database of Poiseuille flow for $\mathit{Re}_\tau = 64$, available from \textless\url{https://www.rs.tus.ac.jp/~t2lab/db/index.html}\textgreater, (accessed on 10 October, 2023).
            \end{quote}
        \item 文献自体が日本語で書かれている場合は外国人が書いていても日本語文献とする.
        \item 文献自体が英語で書かれている場合は日本人が書いていても英語文献とする.
        \item フィールドとフィールドの間は日本語文献・英語文献問わず半角カンマと半角スペースで繋ぐ.
        \item 他学会の文献テンプレートでは誌名を\textit{Italic}で書いたり巻数を\textbf{Bold}にしたりすることがあるが,日本機械学会では全てローマン体で統一する.
    \end{itemize}
\end{itembox}
\renewcommand\UrlFont{\ttfamily}
\begin{itembox}[l]{文献リスト作成の注意事項(著者名)}
    \begin{itemize}
        \item 著者の氏名は「姓」「名」の順で書き,間にスペース等は入れない(例:松川裕樹).
        \item 英語文献の著者の氏名はFamily nameのみ略記せず,Middle nameやGiven nameはイニシャルで記載する(例:Matsukawa Yuki $\to$ Matsukawa, Y.).
        \item 著者は全員記載する.
        \item 日本語文献の著者数が二人以上の場合は
            \begin{quote}
                松川裕樹, 塚原隆裕        
            \end{quote}
            のように半角カンマと半角スペースを間に入れて繋ぐ.全角カンマではないので注意.
        \item 英語文献の著者数が二人の場合は
            \begin{quote}
                Matsukawa, Y. and Tsukahara, T.        
            \end{quote}
            のようにandで繋ぎ,著者数が三人以上の場合は
            \begin{quote}
                Araki, R., Bos, W. J. T. and Goto, S.            
            \end{quote}
            のように最後だけandで繋ぐ.最後のandの前にカンマは入れない.
    \end{itemize}
\end{itembox}
\begin{itembox}[l]{文献リスト作成の注意事項(その他)}
    \begin{itemize}
        \item 英語文献のタイトルは最初の単語の頭文字のみ大文字(固有名詞等は除く).
        \item 誌名・書名は省略せずに記載する.
        \item 巻・号は日本語・英語文献問わずVol.~xx,No.~xxとする.
        \item 発行年は西暦で表記し,括弧で括る.ただし,発行年の前のみ半角カンマは不要.
        \item ページ数は単ページの場合はp.~xxとし,複数ページに亘る場合はpp.~xx--yyとする.このとき,xxとyyを結ぶ横棒はハイフン(\verb|-|)ではなくenダッシュとする(\verb|--|).ただし,\verb|jsme.bst|を使用すれば\verb|pages|内でハイフンとしていても自動でenダッシュに変換してくれるので安心.
    \end{itemize}
\end{itembox}
以上の内容が満たされていれば他の情報は入れなくてよいと思われる.



\section{引用のコマンドと本文中での表示}
\label{sec:cite}
Table~\ref{tab:English}, \ref{tab:Japanese}は\verb|\citep|や\verb|\citealp|等の,本文中で文献を引用する際のコマンドとその出力結果の一覧を示しています.
これらのコマンドを使用するには\verb|natbib.sty|というスタイルファイルを読み込む必要があります.
このテンプレートではプリアンブルに\verb|\usepackage{natbib}|と書くことで\verb|natbib.sty|を読み込んでいます.
\verb|\cite|と\verb|\citet|は出力結果が同じなのでどちらを使っても構いませんが,他のコマンドは細かい箇所(カンマや括弧の有無)に違いがあるので自分の出力したい内容に合わせて使うようにしてください.

\subsection*{\texttt{\textbackslash{}cite}, \texttt{\textbackslash{}citet}}
\verb|\cite|と\verb|\citet|は文章中で著者名と発行年を \verb|author (year)| の形で引用する際に使用するコマンドです.


\begin{table}[t]
    \centering
    \caption{Commands and output results for English bibliography.}
    \label{tab:English}
    \begin{tabular}{ll}
        コマンド &出力結果 \\
        \multicolumn{2}{c}{------単著------} \\
        \verb|\cite{Reynolds:PhilTransRoySoc1883}|          &\cite{Reynolds:PhilTransRoySoc1883} \\
        \verb|\citet{Reynolds:PhilTransRoySoc1883}|         &\citet{Reynolds:PhilTransRoySoc1883} \\
        \verb|\citep{Reynolds:PhilTransRoySoc1883}|         &\citep{Reynolds:PhilTransRoySoc1883} \\
        \verb|\citealt{Reynolds:PhilTransRoySoc1883}|       &\citealt{Reynolds:PhilTransRoySoc1883} \\
        \verb|\citealp{Reynolds:PhilTransRoySoc1883}|       &\citealp{Reynolds:PhilTransRoySoc1883} \\
        \verb|\citeauthor{Reynolds:PhilTransRoySoc1883}|    &\citeauthor{Reynolds:PhilTransRoySoc1883} \\
        \verb|\citeyear{Reynolds:PhilTransRoySoc1883}|      &\citeyear{Reynolds:PhilTransRoySoc1883} \\
        \verb|\citeyearpar{Reynolds:PhilTransRoySoc1883}|   &\citeyearpar{Reynolds:PhilTransRoySoc1883} \\
        \multicolumn{2}{c}{------著者2名以上------} \\
        \verb|\cite{Matsukawa:PoF2022}|         &\cite{Matsukawa:PoF2022} \\
        \verb|\citet{Matsukawa:PoF2022}|        &\citet{Matsukawa:PoF2022} \\
        \verb|\citep{Matsukawa:PoF2022}|        &\citep{Matsukawa:PoF2022} \\
        \verb|\citealt{Matsukawa:PoF2022}|      &\citealt{Matsukawa:PoF2022} \\
        \verb|\citealp{Matsukawa:PoF2022}|      &\citealp{Matsukawa:PoF2022} \\
        \verb|\citeauthor{Matsukawa:PoF2022}|   &\citeauthor{Matsukawa:PoF2022} \\
        \verb|\citeyear{Matsukawa:PoF2022}|     &\citeyear{Matsukawa:PoF2022} \\
        \verb|\citeyearpar{Matsukawa:PoF2022}|  &\citeyearpar{Matsukawa:PoF2022} \\
        \multicolumn{2}{c}{------著者3名以上------} \\
        \verb|\cite{Berghout:JFM2020}|          &\cite{Berghout:JFM2020} \\
        \verb|\citet{Berghout:JFM2020}|         &\citet{Berghout:JFM2020} \\
        \verb|\citep{Berghout:JFM2020}|         &\citep{Berghout:JFM2020} \\
        \verb|\citealt{Berghout:JFM2020}|       &\citealt{Berghout:JFM2020} \\
        \verb|\citealp{Berghout:JFM2020}|       &\citealp{Berghout:JFM2020} \\
        \verb|\citeauthor{Berghout:JFM2020}|    &\citeauthor{Berghout:JFM2020} \\
        \verb|\citeyear{Berghout:JFM2020}|      &\citeyear{Berghout:JFM2020} \\
        \verb|\citeyearpar{Berghout:JFM2020}|   &\citeyearpar{Berghout:JFM2020}
    \end{tabular}
\end{table}


\begin{table}[t]
    \centering
    \caption{Commands and output results for Japanese bibliography.}
    \label{tab:Japanese}
    \begin{tabular}{ll}
        コマンド &出力結果 \\
        \multicolumn{2}{c}{------単著------} \\
        \verb|\cite{塚原:ながれ2023}|            &\cite{塚原:ながれ2023} \\
        \verb|\citet{塚原:ながれ2023}|           &\citet{塚原:ながれ2023} \\
        \verb|\citep{塚原:ながれ2023}|           &\citep{塚原:ながれ2023} \\
        \verb|\citealt{塚原:ながれ2023}|         &\citealt{塚原:ながれ2023} \\
        \verb|\citealp{塚原:ながれ2023}|         &\citealp{塚原:ながれ2023} \\
        \verb|\citeauthor{塚原:ながれ2023}|      &\citeauthor{塚原:ながれ2023} \\
        \verb|\citeyear{塚原:ながれ2023}|        &\citeyear{塚原:ながれ2023} \\
        \verb|\citeyearpar{塚原:ながれ2023}|     &\citeyearpar{塚原:ながれ2023} \\
        \multicolumn{2}{c}{------著者2名以上------} \\
        \verb|\cite{塚原:ながれ2015}|            &\cite{塚原:ながれ2015} \\
        \verb|\citet{塚原:ながれ2015}|           &\citet{塚原:ながれ2015} \\
        \verb|\citep{塚原:ながれ2015}|           &\citep{塚原:ながれ2015} \\
        \verb|\citealt{塚原:ながれ2015}|         &\citealt{塚原:ながれ2015} \\
        \verb|\citealp{塚原:ながれ2015}|         &\citealp{塚原:ながれ2015} \\
        \verb|\citeauthor{塚原:ながれ2015}|      &\citeauthor{塚原:ながれ2015} \\
        \verb|\citeyear{塚原:ながれ2015}|        &\citeyear{塚原:ながれ2015} \\
        \verb|\citeyearpar{塚原:ながれ2015}|     &\citeyearpar{塚原:ながれ2015} \\
        \multicolumn{2}{c}{------著者3名以上------} \\
        \verb|\cite{塚原:伝熱2007}|             &\cite{塚原:伝熱2007} \\
        \verb|\citet{塚原:伝熱2007}|            &\citet{塚原:伝熱2007} \\
        \verb|\citep{塚原:伝熱2007}|            &\citep{塚原:伝熱2007} \\
        \verb|\citealt{塚原:伝熱2007}|          &\citealt{塚原:伝熱2007} \\
        \verb|\citealp{塚原:伝熱2007}|          &\citealp{塚原:伝熱2007} \\
        \verb|\citeauthor{塚原:伝熱2007}|       &\citeauthor{塚原:伝熱2007} \\
        \verb|\citeyear{塚原:伝熱2007}|         &\citeyear{塚原:伝熱2007} \\
        \verb|\citeyearpar{塚原:伝熱2007}|      &\citeyearpar{塚原:伝熱2007}
    \end{tabular}
\end{table}


\begin{table}[t]
    \centering
    \caption{Commands and output results for English bibliography.}
    \label{tab:English2}
    \begin{tabular}{ll}
        コマンド &出力結果 \\
        \multicolumn{2}{c}{------英語文献と英語文献------} \\
        \verb|\cite{Matsukawa:ICFD2022,Matsukawa:PoF2022}|            &\cite{Matsukawa:ICFD2022,Matsukawa:PoF2022} \\
        \verb|\citet{Matsukawa:ICFD2022,Matsukawa:PoF2022}|           &\citet{Matsukawa:ICFD2022,Matsukawa:PoF2022} \\
        \verb|\citep{Matsukawa:ICFD2022,Matsukawa:PoF2022}|           &\citep{Matsukawa:ICFD2022,Matsukawa:PoF2022} \\
        \verb|\citealt{Matsukawa:ICFD2022,Matsukawa:PoF2022}|         &\citealt{Matsukawa:ICFD2022,Matsukawa:PoF2022} \\
        \verb|\citealp{Matsukawa:ICFD2022,Matsukawa:PoF2022}|         &\citealp{Matsukawa:ICFD2022,Matsukawa:PoF2022} \\
        \verb|\citeauthor{Matsukawa:ICFD2022,Matsukawa:PoF2022}|      &\citeauthor{Matsukawa:ICFD2022,Matsukawa:PoF2022} \\
        \verb|\citeyear{Matsukawa:ICFD2022,Matsukawa:PoF2022}|        &\citeyear{Matsukawa:ICFD2022,Matsukawa:PoF2022} \\
        \verb|\citeyearpar{Matsukawa:ICFD2022,Matsukawa:PoF2022}|     &\citeyearpar{Matsukawa:ICFD2022,Matsukawa:PoF2022} \\
        \multicolumn{2}{c}{------英語文献と日本語文献------} \\
        \verb|\cite{Matsukawa:ICFD2022,松川:流力年会2022}|            &\cite{Matsukawa:ICFD2022,松川:流力年会2022} \\
        \verb|\citet{Matsukawa:ICFD2022,松川:流力年会2022}|           &\citet{Matsukawa:ICFD2022,松川:流力年会2022} \\
        \verb|\citep{Matsukawa:ICFD2022,松川:流力年会2022}|           &\citep{Matsukawa:ICFD2022,松川:流力年会2022} \\
        \verb|\citealt{Matsukawa:ICFD2022,松川:流力年会2022}|         &\citealt{Matsukawa:ICFD2022,松川:流力年会2022} \\
        \verb|\citealp{Matsukawa:ICFD2022,松川:流力年会2022}|         &\citealp{Matsukawa:ICFD2022,松川:流力年会2022} \\
        \verb|\citeauthor{Matsukawa:ICFD2022,松川:流力年会2022}|      &\citeauthor{Matsukawa:ICFD2022,松川:流力年会2022} \\
        \verb|\citeyear{Matsukawa:ICFD2022,松川:流力年会2022}|        &\citeyear{Matsukawa:ICFD2022,松川:流力年会2022} \\
        \verb|\citeyearpar{Matsukawa:ICFD2022,松川:流力年会2022}|     &\citeyearpar{Matsukawa:ICFD2022,松川:流力年会2022}
    \end{tabular}
\end{table}


\clearpage
\section{書誌情報ファイル(\texttt{bib}ファイル)の作り方}
\label{sec:bib}
ここでは書誌情報ファイル(\texttt{bib}ファイル)の作り方,使い方を説明します.
この\verb|pdf|の末尾の参考文献リストは同じディレクトリにある\verb|mybib_en.bib|と\verb|mybib_jp.bib|から作成しているので参考にしてください.
\verb|jsme.bst|でサポートされているエントリは\hyperref[ssec:article]{\texttt{@article}}, \hyperref[ssec:book]{\texttt{@book}}, \hyperref[ssec:booklet]{\texttt{@booklet}}, \hyperref[ssec:conference]{\texttt{@conference}}, \hyperref[ssec:inbook]{\texttt{@inbook}}, \hyperref[ssec:incollection]{\texttt{@incollection}}, \hyperref[ssec:inproceedings]{\texttt{@inproceedings}}, \hyperref[ssec:manual]{\texttt{@manual}}, \hyperref[ssec:mastersthesis]{\texttt{@mastersthesis}}, \hyperref[ssec:misc]{\texttt{@misc}}, \hyperref[ssec:online]{\texttt{@online}}, \hyperref[ssec:phdthesis]{\texttt{@phdthesis}}, \hyperref[ssec:proceedings]{\texttt{@proceedings}}, \hyperref[ssec:techreport]{\texttt{@techreport}}, \hyperref[ssec:unpublished]{\texttt{@unpublished}}の15種類です.
それぞれのエントリで必須となる項目(フィールド)が異なり,文献一覧への出力の方法も異なるので面倒くさがらずに分類しましょう.
ただ,全ての文献を正確に分類することは難しく,判断が人により異なることもあります.
\verb|JSME-bst|における分類の仕方も見る人によっては違和感を覚えるものがあるかもしれません.

固有名詞


\citet{流力年会2023}

\cite{THMT2023}





\subsection{\texttt{@article}}
\label{ssec:article}
\begin{screen}
    \begin{itemize}
        \item 必須項目 \\
        \verb|author|, \verb|title|, \verb|journal|, \verb|year|
        \item オプション項目 \\
        \verb|volume|, \verb|number|, \verb|pages|, \verb|month|, \verb|note|, \verb|key|
        \item 出力理想形 \\
            \colorbox[gray]{0.8}{\texttt{author 1}}, \colorbox[gray]{0.8}{\texttt{author 2}} and \colorbox[gray]{0.8}{\texttt{author 3}}, \colorbox[gray]{0.8}{\texttt{title}}, \colorbox[gray]{0.8}{\texttt{journal}} (\colorbox[gray]{0.8}{\texttt{year}}), Vol.~\colorbox[gray]{0.8}{\texttt{volume}}, No.~\colorbox[gray]{0.8}{\texttt{number}}, pp.~\colorbox[gray]{0.8}{\texttt{pages}}, \colorbox[gray]{0.8}{\texttt{note}}.
        \item \verb|bib|ファイル作成例 \vspace{-3mm}
\begin{verbatim}
@article{Matsukawa:PoF2022,
    author  = {Matsukawa, Yuki and Tsukahara, Takahiro},
    title   = {Subcritical transition of {Taylor--Couette--Poiseuille} flow 
                at high radius ratio},
    journal = {Physics of Fluids},
    volume  = {34},
    number  = {7},
    year    = {2022},
    doi     = {10.1063/5.0096676},
    url     = {https://doi.org/10.1063/5.0096676},
    note    = {074109}
}
\end{verbatim}
    \end{itemize}
\end{screen}

\texttt{@article}は雑誌に掲載された論文です.
流体力学分野では英文雑誌だとJournal of Fluid Mechanics\footnote{Journal of Fluid Mechanics, \textless\url{https://www.cambridge.org/core/journals/journal-of-fluid-mechanics}\textgreater}やPhysics of Fluids\footnote{Physics of Fluids, \textless\url{https://pubs.aip.org/aip/pof}\textgreater}などが挙げられます.
国内雑誌だと日本機械学会誌\footnote{日本機械学会誌, \textless\url{https://www.jsme.or.jp/publication/kaisi/}\textgreater}や日本流体力学会誌『ながれ』\footnote{日本流体力学会誌『ながれ』, \textless\url{https://www.nagare.or.jp/publication/nagare.html}\textgreater}などが該当します.


\subsection{\texttt{@book}}
\label{ssec:book}
\begin{screen}
    \begin{itemize}
        \item 必須項目 \\
        \verb|author| / \verb|editor|, \verb|title|, \verb|publisher|, \verb|year|
        \item オプション項目 \\
        \verb|volume|, \verb|number|, \verb|series|, \verb|address|, \verb|edition|, \verb|month|, \verb|note|, \verb|key|
        \item 出力理想形 \\
            \colorbox[gray]{0.8}{\texttt{author 1}}, \colorbox[gray]{0.8}{\texttt{author 2}} and \colorbox[gray]{0.8}{\texttt{author 3}}, \colorbox[gray]{0.8}{\texttt{title}}, \colorbox[gray]{0.8}{\texttt{publisher}} (\colorbox[gray]{0.8}{\texttt{year}}), \colorbox[gray]{0.8}{\texttt{note}}.
        \item \verb|bib|ファイル作成例 \vspace{-3mm}
\begin{verbatim}
@book{Schmid:Springer2001,
    author      = {Peter J. Schmid and Dan S. Henningson},
    title       = {Stability and Transition in Shear Flows},
    publisher   = {Springer New York},
    year        = {2001},
    doi         = {10.1007/978-1-4613-0185-1}
}
\end{verbatim}
    \end{itemize}
\end{screen}

出版社が刊行した書籍を引用する際は\verb|@book|を使います.
似たエントリとして\hyperref[ssec:inbook]{\texttt{@inbook}}がありますが,特定のページを参照したのではなく,書籍全体を参照した場合は\verb|@book|を使いましょう.

\subsection{\texttt{@booklet}}
\label{ssec:booklet}
\begin{screen}
    \begin{itemize}
        \item 必須項目 \\
        \verb|title|
        \item オプション項目 \\
        \verb|author|, \verb|howpublished|, \verb|address|, \verb|month|, \verb|year|, \verb|note|, \verb|key|
        \item 出力理想形 \\
            \colorbox[gray]{0.8}{\texttt{author 1}}, \colorbox[gray]{0.8}{\texttt{author 2}} and \colorbox[gray]{0.8}{\texttt{author 3}}, \colorbox[gray]{0.8}{\texttt{title}}, \colorbox[gray]{0.8}{\texttt{howpublished}} (\colorbox[gray]{0.8}{\texttt{year}}), \colorbox[gray]{0.8}{\texttt{note}}.
        \item \verb|bib|ファイル作成例 \vspace{-3mm}
\begin{verbatim}
@booklet{Wang:MEnews2014,
    author          = {Wang, Lin},
    title           = {Exchange student from 
                        {Northwestern Polytechnical University (China)}},
    howpublished    = {ME Newsletter, Department of Mechanical Engineering, 
                        Tokyo University of Science},
    year            = {2014},
    url             = {https://www.rs.tus.ac.jp/me/pdf/newsletter/ME_NL_No15.pdf}
}
\end{verbatim}
    \end{itemize}
\end{screen}

\verb|@booklet|は使う機会が少ないため分類が難しいエントリですが,出版社が明記されていないような(薄い)冊子媒体が該当します.
\verb|JSME-bst|では例として東京理科大学理工学部機械工学科(現・創域理工学部機械航空宇宙工学科)が毎年出しているMEニュースレター\footnote{MEニュースレター, \textless\url{https://www.rs.tus.ac.jp/me/newsletter.html}\textgreater}という広報の冊子を引用しました.

\subsection{\texttt{@conference}}
\label{ssec:conference}
\hyperref[ssec:inproceedings]{\texttt{@inproceedings}}と同様なので省略.
Scribeというシステムとの互換性のために残されているらしい(\citealp{奥村:技評2020}).


\subsection{\texttt{@inbook}}
\label{ssec:inbook}
\begin{screen}
    \begin{itemize}
        \item 必須項目 \\
        \verb|author| / \verb|editor|, \verb|title|, \verb|chapter| / \verb|pages|, \verb|publisher|, \verb|year|
        \item オプション項目 \\
        \verb|volume|, \verb|series|, \verb|address|, \verb|edition|, \verb|month|, \verb|note|, \verb|key|
        \item 出力理想形 \\
            \colorbox[gray]{0.8}{\texttt{author 1}}, \colorbox[gray]{0.8}{\texttt{author 2}} and \colorbox[gray]{0.8}{\texttt{author 3}}, \colorbox[gray]{0.8}{\texttt{title}}, \colorbox[gray]{0.8}{\texttt{publisher}} (\colorbox[gray]{0.8}{\texttt{year}}), \colorbox[gray]{0.8}{\texttt{pages}}, \colorbox[gray]{0.8}{\texttt{note}}.
        \item \verb|bib|ファイル作成例 \vspace{-3mm}
\begin{verbatim}
@inbook{Davidson:Oxford2015,
    author      = {Peter A. Davidson},
    title       = {Turbulence: 
                    An Introduction for Scientists and Engineers, Second Edition},
    publisher   = {Oxford University Press},
    year        = {2015},
    pages       = {61--104}
}
\end{verbatim}
    \end{itemize}
\end{screen}


\subsection{\texttt{@incollection}}
\label{ssec:incollection}
\begin{screen}
    \begin{itemize}
        \item 必須項目 \\
        \verb|author|, \verb|title|, \verb|booktitle|, \verb|year|
        \item オプション項目 \\
        \verb|editor|, \verb|pages|, \verb|organization|, \verb|publisher|, \verb|address|, \verb|month|, \verb|note|, \verb|key|
        \item 出力理想形 \\
            \colorbox[gray]{0.8}{\texttt{author 1}}, \colorbox[gray]{0.8}{\texttt{author 2}} and \colorbox[gray]{0.8}{\texttt{author 3}}, \colorbox[gray]{0.8}{\texttt{title}}, \colorbox[gray]{0.8}{\texttt{booktitle}}, \colorbox[gray]{0.8}{\texttt{publisher}} (\colorbox[gray]{0.8}{\texttt{year}}), \colorbox[gray]{0.8}{\texttt{pages}}, \colorbox[gray]{0.8}{\texttt{note}}.
        \item \verb|bib|ファイル作成例 \vspace{-3mm}
\begin{verbatim}
@incollection{Lueptow:Springer2000,
    author      = {Lueptow, Richard M.},
    title       = {Stability and experimental velocity field 
                    in {Taylor--Couette} flow with axial and radial flow},
    booktitle   = {Physics of Rotating Fluids},
    publisher   = {Springer-Verlag Berlin Heidelberg New York},
    pages       = {137--155},
    year        = {2000},
    doi         = {10.1007/3-540-45549-3}
}
\end{verbatim}
    \end{itemize}
\end{screen}


\subsection{\texttt{@inproceedings}}
\label{ssec:inproceedings}

\subsection{\texttt{@manual}}
\label{ssec:manual}

\subsection{\texttt{@mastersthesis}}
\label{ssec:mastersthesis}
\begin{screen}
    \begin{itemize}
        \item 必須項目 \\
        \verb|author|, \verb|title|, \verb|school|, \verb|year|
        \item オプション項目 \\
        \verb|address|, \verb|month|, \verb|note|, \verb|key|
        \item 出力理想形(英語文献) \\
            \colorbox[gray]{0.8}{\texttt{author 1}}, \colorbox[gray]{0.8}{\texttt{author 2}} and \colorbox[gray]{0.8}{\texttt{author 3}}, \colorbox[gray]{0.8}{\texttt{title}}, Master's thesis, \colorbox[gray]{0.8}{\texttt{school}} (\colorbox[gray]{0.8}{\texttt{year}}), \colorbox[gray]{0.8}{\texttt{note}}.
        \item 出力理想形(日本語文献) \\
            \colorbox[gray]{0.8}{\texttt{author 1}}, \colorbox[gray]{0.8}{\texttt{author 2}} and \colorbox[gray]{0.8}{\texttt{author 3}}, \colorbox[gray]{0.8}{\texttt{title}}, \colorbox[gray]{0.8}{\texttt{school}}修士論文 (\colorbox[gray]{0.8}{\texttt{year}}), \colorbox[gray]{0.8}{\texttt{note}}.
        \item \verb|bib|ファイル作成例 \vspace{-3mm}
\begin{verbatim}
@mastersthesis{松川:修論2023,
    author  = {松川, 裕樹},
    yomi    = {Matsukawa, Yuki},
    title   = {直接数値解析を用いた高円筒比
                Taylor--Couette--Poiseuille流の流動状態遷移過程の分類},
    school  = {東京理科大学大学院理工学研究科機械工学専攻},
    year    = {2023}
}
\end{verbatim}
    \end{itemize}
\end{screen}

修士論文は\verb|@mastersthesis|に分類します.
\verb|@masterthesis|ではなく\verb|@mastersthesis|です.
\verb|s|を忘れないでください.
また,\verb|year|は修了「年度」ではなく修了「年」を西暦で書いてください.
例えば,2023年3月に修了した人は2022年度修了生ですが\verb|year = {2023}|です.

\subsection{\texttt{@misc}}
\label{ssec:misc}
\begin{screen}
    \begin{itemize}
        \item 必須項目 \\
        無し
        \item オプション項目 \\
        \verb|author|, \verb|title|, \verb|howpublished|, \verb|archivePrefix|, \verb|eprint|, \verb|month|, \verb|year|, \verb|note|, \verb|key|
        \item 出力理想形(通常) \\
            \colorbox[gray]{0.8}{\texttt{author 1}}, \colorbox[gray]{0.8}{\texttt{author 2}} and \colorbox[gray]{0.8}{\texttt{author 3}}, \colorbox[gray]{0.8}{\texttt{title}}, \colorbox[gray]{0.8}{\texttt{howpublished}} (\colorbox[gray]{0.8}{\texttt{year}}), \colorbox[gray]{0.8}{\texttt{note}}.
        \item 出力理想形(arXivの場合) \\
            \colorbox[gray]{0.8}{\texttt{author 1}}, \colorbox[gray]{0.8}{\texttt{author 2}} and \colorbox[gray]{0.8}{\texttt{author 3}}, \colorbox[gray]{0.8}{\texttt{title}}, arXiv: \colorbox[gray]{0.8}{\texttt{eprint}} (\colorbox[gray]{0.8}{\texttt{year}}), \colorbox[gray]{0.8}{\texttt{note}}.
        \item \verb|bib|ファイル作成例(通常) \vspace{-3mm}
\begin{verbatim}
@misc{湯村:卒論2006,
    author          = {湯村, 翼},
    yomi            = {Yumura, Tsubasa},
    title           = {レイリーテイラー不安定による赤道電離圏プラズマバブルの発生},
    howpublished    = {北海道大学理学部地球科学科卒業論文},
    year            = {2006},
    url             = {https://researchmap.jp/yumu/published_papers/1902404}
}
\end{verbatim}
        \item \verb|bib|ファイル作成例(arXivの場合) \vspace{-3mm}
\begin{verbatim}
@misc{Araki:arXiv2023,
    author          = {Araki, Ryo and Bos, Wouter J. T. and Goto, Susumu},
    title           = {Space-local {Navier--Stokes} turbulence}, 
    year            = {2023},
    eprint          = {2308.07255},
    archivePrefix   = {arXiv},
    primaryClass    = {physics.flu-dyn}
}
\end{verbatim}
    \end{itemize}
\end{screen}

その他該当種別が無いものは\verb|@misc|とします.
学部の卒業論文は\verb|misc|でいいと思います.
ただし,\verb|@mastersthesis|や\verb|@phdthesis|と異なり,\verb|school|のフィールドを使用できないので\verb|howpublished|で代用します.
したがって,\verb|@mastersthesis|や\verb|@phdthesis|では\verb|school|に所属名だけ(例:\verb|school = {東京理科大学大学院理工学研究科機械工学専攻}|)書けばよかったものが\verb|@misc|で卒論を出力する際には\verb|howpublished = {北海道大学理学部地球科学科卒業論文}|のように「\verb|卒業論文|」の文字まで書く必要があります.
また,arXiv\footnote{arXiv, \textless\url{https://arxiv.org/}\textgreater}と呼ばれるプレプリントサーバーから引用した文献は\verb|@misc|に分類します.
arXiv上のExport BibTeX Citationと書いてあるところから文献情報を見ると\verb|@misc|に分類されていることがわかると思います.
この文献情報では上記のように\verb|eprint = {2308.07255}|, \verb|archivePrefix = {arXiv}|などと書かれていることが多いです.
\verb|jsme.bst|では\verb|archivePrefix|フィールドがあるとこの文献がarXiv上の文献だと認識してくれて\verb|eprint|の情報と合わせて
\begin{quote}
    Araki, R., Bos, W. J. T. and Goto, S., Space-local Navier--Stokes turbulence, \href{https://doi.org/10.48550/arXiv.2308.07255}{arXiv: 2308.07255} (2023).
\end{quote}
のように自動で書いてくれます.
\verb|eprint|の情報からURLを自動生成するので,\href{https://doi.org/10.48550/arXiv.2308.07255}{arXiv: 2308.07255}と書かれている(青字の)箇所をクリックしたらarXivの該当ページにジャンプできます.


\subsection{\texttt{@online}}
\label{ssec:online}
\begin{screen}
    \begin{itemize}
        \item 必須項目 \\
        後で確認.
        \item オプション項目 \\
        後で確認.
        \item 出力理想形(英語文献) \\
            \colorbox[gray]{0.8}{\texttt{author 1}}, \colorbox[gray]{0.8}{\texttt{author 2}} and \colorbox[gray]{0.8}{\texttt{author 3}}, \colorbox[gray]{0.8}{\texttt{title}}, available from \textless\colorbox[gray]{0.8}{\texttt{url}}\textgreater, (accessed on \colorbox[gray]{0.8}{access}).
        \item 出力理想形(日本語文献) \\
            \colorbox[gray]{0.8}{\texttt{author 1}}, \colorbox[gray]{0.8}{\texttt{author 2}} and \colorbox[gray]{0.8}{\texttt{author 3}}, \colorbox[gray]{0.8}{\texttt{title}}, available from \textless\colorbox[gray]{0.8}{\texttt{url}}\textgreater, (参照日 \colorbox[gray]{0.8}{access}).
        \item \verb|bib|ファイル作成例 \vspace{-3mm}
\begin{verbatim}
@article{Matsukawa:PoF2022,
    author  = {Matsukawa, Yuki and Tsukahara, Takahiro},
    title   = {Subcritical transition of {Taylor--Couette--Poiseuille} flow 
                at high radius ratio},
    journal = {Physics of Fluids},
    volume  = {34},
    number  = {7},
    year    = {2022},
    doi     = {10.1063/5.0096676},
    url     = {https://doi.org/10.1063/5.0096676},
    note    = {074109}
}
\end{verbatim}
    \end{itemize}
\end{screen}

\verb|@online|は\verb|jsme.bst|独自のエントリなので他の\BibTeX{}スタイルファイルを用いるときには注意してください(\verb|jsme.bst|の元となった\verb|jecon.bst|では使用できます).
本来,webページ等の引用はあまり推奨されるものではありませんが,データベース等を研究室のwebページ等で公開していることがあるので使う機会がゼロとは言えないでしょう.
英語のwebページの場合は
\renewcommand\UrlFont{\rmfamily}
\begin{quote}
    Kawamura Laboratory, DNS database of wall turbulence and heat transfer: Text database of Poiseuille flow for $\mathit{Re}_\tau = 64$, available from \textless\url{https://www.rs.tus.ac.jp/~t2lab/db/index.html}\textgreater, (accessed on 10 October, 2023).
\end{quote}
のように,末尾にwebページのURLと参照日を入れることが日本機械学会の原稿テンプレートに記載されています.
日本語のwebページの場合は
\begin{quote}
    立川裕二, 博士論文執筆の際にお願いしたいこと, \textless\url{https://member.ipmu.jp/yuji.tachikawa/misc/dron.html}\textgreater, (参照日 2023年10月10日).
\end{quote}
\renewcommand\UrlFont{\ttfamily}
のように出力します.

\subsection{\texttt{@phdthesis}}
\label{ssec:phdthesis}
\begin{screen}
    \begin{itemize}
        \item 必須項目 \\
        \verb|author|, \verb|title|, \verb|school|, \verb|year|
        \item オプション項目 \\
        \verb|address|, \verb|month|, \verb|note|, \verb|key|
        \item 出力理想形 \\
            \colorbox[gray]{0.8}{\texttt{author 1}}, \colorbox[gray]{0.8}{\texttt{author 2}} and \colorbox[gray]{0.8}{\texttt{author 3}}, \colorbox[gray]{0.8}{\texttt{title}}, \colorbox[gray]{0.8}{\texttt{booktitle}}, \colorbox[gray]{0.8}{\texttt{publisher}} (\colorbox[gray]{0.8}{\texttt{year}}), \colorbox[gray]{0.8}{\texttt{pages}}, \colorbox[gray]{0.8}{\texttt{note}}.
        \item \verb|bib|ファイル作成例 \vspace{-3mm}
\begin{verbatim}
@mastersthesis{松川:修論2023,
    author  = {松川, 裕樹},
    yomi    = {Matsukawa, Yuki},
    title   = {直接数値解析を用いた高円筒比
                Taylor--Couette--Poiseuille流の流動状態遷移過程の分類},
    school  = {東京理科大学大学院理工学研究科機械工学専攻},
    year    = {2023}
}
\end{verbatim}
    \end{itemize}
\end{screen}


\subsection{\texttt{@proceedings}}
\label{ssec:proceedings}

\subsection{\texttt{@techreport}}
\label{ssec:techreport}

\subsection{\texttt{@unpublished}}
\label{ssec:unpublished}




\section{\texttt{jsme.bst}の使い方}
それでは実際に\verb|jsme.bst|を使ってみましょう.

\section*{謝辞}
\addcontentsline{toc}{section}{謝辞}
\verb|jsme.bst|は武田史郎氏作の経済学用\BibTeX{}スタイルファイル\verb|jecon.bst|\footnote{\texttt{jecon-bst}, \textless\url{https://github.com/ShiroTakeda/jecon-bst}\textgreater}を改変して作成したものです.
\verb|jecon.bst|内に残されていた武田氏の懇切丁寧なコメントおよび説明用の\verb|jecon-example.pdf|は\verb|jsme.bst|を作成するにあたり大変参考になり,これらが無ければ実現しませんでした.
また,\citet{奥村:技評2020}と\citet{吉永:翔泳社2018}は\verb|JSME-bst|を作成するにあたり参考にした,この文書における「本当の」参考文献です.
ここに,深く感謝の意を表します.

\clearpage
%%% 参考文献内の URL 表示をタイプライター調にしない.
\renewcommand\UrlFont{\rmfamily}
%%% \nocite{*}が有効のとき,引用していない文献も含めて全て表示.
\nocite{*}
%%% 使用する bst ファイル
\bibliographystyle{jsme}
%%% 読み込む bib ファイル
\bibliography{
mybib_en.bib,
mybib_jp.bib
}
\addcontentsline{toc}{section}{\refname}




\end{document}