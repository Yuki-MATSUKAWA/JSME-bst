%%%%%%%%%%%%%%%%%%%%%%%%%%%%%%%%%%%%%%%%%%%%%%%%%%%%%%%%%
%%%
%%%      【非公式】日本機械学会 BibTeX スタイルファイル
%%%           JSME-bst テンプレートその1(標準)
%%%                jsme.bst 取扱説明書
%%%
%%%                    v1.0.0 Yuki MATSUKAWA 22 Oct. 2023
%%%                    v2.0.0 Yuki MATSUKAWA 28 Oct. 2023
%%%                    v2.3.1 Yuki MATSUKAWA 08 Dec. 2023
%%%
%%%%%%%%%%%%%%%%%%%%%%%%%%%%%%%%%%%%%%%%%%%%%%%%%%%%%%%%%
\documentclass[a4paper,fleqn,uplatex,dvipdfmx]{jsarticle}

%%% 共通設定の読み込み
\usepackage{JSME-template}
%%% フォントを Times に変更
\usepackage{newtxtext,newtxmath}
%%% フォントを Palatino に変更
% \usepackage{mathpazo}
%%% hyperref
%%% ハイパーリンクに色が付かないようにしたいときは allcolors=blue の箇所を hidelinks にする
\usepackage[bookmarks=true,bookmarksnumbered=true,setpagesize=false,colorlinks,allcolors=blue]{hyperref}
\usepackage{pxjahyper}
%%% natbib
\usepackage{natbib}
%%% 参考文献リストの見出しを「参考文献」から「文献」に変更.
\renewcommand{\refname}{文献}
%%% その他の定義
\newcommand{\jsmefile}{\texttt{jsme.bst}}


\title{【非公式】日本機械学会\BibTeX{}スタイルファイル \\ \JSMErepos テンプレートその1(標準)}
\author{松川 裕樹\thanks{東京理科大学大学院 創域理工学研究科 機械航空宇宙工学専攻 博士後期課程,Email: \texttt{7523701 _@_ ed.tus.ac.jp}}}

\date{最終更新:\today}

\begin{document}

\maketitle
\thispagestyle{cover}

\vspace{-5mm}
\tableofcontents

\section{はじめに}
\label{sec:introduction}
\jsmefile は日本機械学会\footnote{一般社団法人 日本機械学会(The Japan Society of Mechanical Engineers, JSME),\textless\url{https://www.jsme.or.jp/}\textgreater}の原稿テンプレート\footnote{日本機械学会 原稿テンプレート,\textless\url{https://www.jsme.or.jp/publish/transact/for-authors.html}\textgreater}に基づいた参考文献の出力を実現するために作成した,非公式\BibTeX{}スタイルファイルテンプレートです.
必要なファイル一式はGitHubの\JSMErepos\footnote{\JSMErepos, \textless\url{https://github.com/Yuki-MATSUKAWA/JSME-bst}\textgreater}から入手可能なので,用途に応じて自由に改変してください.
今読んでいるこのファイル(\LaTeX{}ソース:\verb|JSME-template1.tex|,出力結果:\verb|JSME-template1.pdf|)では\BibTeX{}で用意されている全てのエントリーの出力結果を表示しています.
参考文献のリストはこの\verb|pdf|の末尾で,出力している文献の\verb|bib|ファイルは英語文献\verb|mybib_en.bib|と日本語文献\verb|mybib_jp.bib|の二つです.
\JSMErepos の作成者である松川が流体力学,特に乱流遷移の研究をしているため,引用している文献は乱流遷移の周辺のものが多くなっています(全てではありません).
ただ,材料力学など他分野の方でも基本的な使い方は同じです.
また,第~\ref{sec:cite}節でも述べますが,著者数が1名,2名,3名以上のそれぞれで引用時の出力結果が異なります.
そのため\verb|mybib_en.bib|,\verb|mybib_jp.bib|では可能な限り著者数が1名,2名,3名以上の計3パターンを用意しています.
ただ,基本的に実在の文献を集めて載せているので学位論文(\hyperref[ssec:phdthesis]{\ttphdthesis}, \hyperref[ssec:mastersthesis]{\ttmastersthesis})のように原則著者が一人のものや\hyperref[ssec:manual]{\ttmanual},\hyperref[ssec:unpublished]{\ttunpublished}など一部のエントリーでは全てのパターンを網羅できていない場合もあるのでご了承ください(ただ,参考にするうえで困ることのないくらいにはパターンを網羅しているつもりです).

この文書では\BibTeX{}初心者でも使いやすいよう,\BibTeX{}そのものの使い方や\verb|bib|ファイルの作成方法といった内容も可能な限り説明します.
それでもわからなければさまざまな書籍やwebサイトがあるので参考にしてみてください.
また,その都度説明しますが,日本機械学会の原稿テンプレートに沿ったリストを作成するために\verb|bib|ファイルの作成方法が一部通常の\BibTeX{}と異なる場合があるのでご了承ください.

\JSMErepos 作成にあたり可能な限り日本機械学会の原稿テンプレートを忠実に再現できるよう努力していますが,まだ十分に再現できていない箇所もあります.
もし何か問題や不明点があればGitHubにコメントしていただくかメールをいただけると幸いです.
可能な限り対応しますが,\JSMErepos を使用したことにより発生した問題に対しては一切の責任を持ちませんのでご了承ください.
また,\JSMErepos リポジトリ内のファイルは全て日本機械学会の書式の実現のために作成した非公式ファイルなので使用方法等を日本機械学会に問い合わせることもご遠慮ください.


\section{日本機械学会の原稿執筆要領における注意点}
\label{sec:caution}
この節では日本機械学会の原稿テンプレートに記載されている執筆要領の中でも,特に文献の記載に関する注意点をまとめておきます.
日本機械学会の規定に合わせて論文執筆する際は是非参考にしてください\footnote{\JSMErepos 自体はもともと,日本機械学会の規定に沿って卒業論文執筆を行うことが決められている東京理科大学創域理工学部機械航空宇宙工学科の卒研生向けに作成したものです.もちろんGitHub上で公開しているので世界中の誰が使っても構いません.}.

\subsection{使用する書体・フォントについて}
日本機械学会では本文に使用する書体を和文は明朝体,欧文はSerif体,サイズは$10\,\mathrm{pt}$と規定されています.
これに倣い,参考文献を記載する際も和文を明朝体,欧文をSerif体とします.
Serif体のフォントとしては{\usefont{T1}{ptm}{m}{n} Times}や{\usefont{T1}{pplx}{m}{n} Palatino},{\usefont{OT1}{cmr}{m}{n} Computer Modern}などが挙げられます.
\LaTeX{}標準のフォントは{\usefont{OT1}{cmr}{m}{n} Computer Modern}であって,{\usefont{T1}{ptm}{m}{n} Times}ではないので注意.
フォントを変える際はプリアンブル(\verb|\begin{document}|以前に書く設定)で指定します.
この文書の場合はプリアンブルの\verb|\usepackage{newtxtext,newtxmath}|でフォントを{\usefont{T1}{ptm}{m}{n} Times}に変更しています.
もし文書全体を{\usefont{OT1}{cmr}{m}{n} Computer Modern}に変更したいときは\verb|\usepackage{newtxtext,newtxmath}|をコメントアウトし,{\usefont{T1}{pplx}{m}{n} Palatino}に変更したいときは\verb|\usepackage{mathpazo}|のコメントアウトを外します.
ここでは参考文献の書き方に焦点を当てて説明するのでその他の箇所の書式は日本機械学会の原稿テンプレートをよく読んで作成してください.

\subsection{文献の並べ方・引用の仕方}
この節では文献リストのフィールド(\BibTeX{}における\ttauthor や\tttitle)の並べ方について説明します.
\jsmefile のユーザーは気にせず使用することができますが,改めて整理しておくので\jsmefile を使わないときにも役に立つと思います.
原則,日本機械学会の規定に合わせていますが,原稿テンプレートからは判断が難しい事項は一部独自の解釈を加えているものがあります.

文献を並べる順序の規則は以下の通りです.
\begin{tcolorbox}[enhanced, title=\textgt{文献ソート規則}, drop fuzzy shadow]
    \begin{enumerate}
        \item (Family, Givenの順で並べた際の)筆頭著者の氏名のアルファベット順.並べる際,日本人の氏名は漢字と仮名を用いた日本語表記で構わないが順序はアルファベット順とする.
        \item 筆頭著者が同一人物の場合,第二著者以降のアルファベット順で並べる.著者数が異なる場合は著者数が少ない方が先.これを最後の著者まで繰り返す.
        \item 著者が全員一致する文献があった場合は発行が早い順で並べる.
        \item 確認できる範囲で発行年月日が同じだった場合,タイトルのアルファベット順で並べ,西暦の後に小文字でアルファベットを順に振る(現在,\jsmefile の設定を調整中).
        \begin{quote}
            Matsukawa, Y. and Tsukahara, T., Laminarization in subcritical Taylor--Couette--Poiseuille flow with increasing pressure gradient, Proceedings of Nineteenth International Conference on Flow Dynamics (2022a). \\
            Matsukawa, Y. and Tsukahara, T., Subcritical transition of Taylor--Couette--Poiseuille flow at high radius ratio, \href{https://doi.org/10.1063/5.0096676}{Physics of Fluids}, Vol.~34, No.~7 (2022b), 074109.
        \end{quote}
    \end{enumerate}
\end{tcolorbox}
\noindent
文献を並べる際の表記に関する注意事項は以下の通りです.
\renewcommand\UrlFont{\rmfamily}
\begin{tcolorbox}[enhanced, title=\textgt{文献リスト作成の注意事項(全体)}, drop fuzzy shadow]
    \begin{itemize}
        \item 並び順は原則として \\
            著者(\ttauthor)$\to$タイトル(\tttitle)$\to$誌名(\ttjournal)・書名(\ttbooktitle)$\to$出版社(\ttpublisher)・大学名(\ttschool)・機関名(\ttinstitution)$\to$巻(\ttvolume)$\to$号(\ttnumber)$\to$発行年(\ttyear)$\to$ページ(\ttpages)$\to$論文番号・講演番号(\ttnote) \\
            とする.
            ただし,存在しないフィールドがある場合は抜かす.
            また,webページ等の引用は極力避けるべきだが,引用する場合には必ず末尾にURLと参照日を明記する(第~\ref{ssec:online}節\ttonline を参照).
            \begin{quote}
                Kawamura Laboratory, DNS database of wall turbulence and heat transfer: Text database of Poiseuille flow for $\mathit{Re}_\tau = 64$, available from \textless\url{https://www.rs.tus.ac.jp/~t2lab/db/index.html}\textgreater, (accessed on 10 October, 2023).
            \end{quote}
        \item 文献自体が日本語で書かれている場合は外国人が書いていても日本語文献とする.
        \item 文献自体が英語で書かれている場合は日本人が書いていても英語文献とする.
        \item フィールドとフィールドの間は日本語文献・英語文献問わず半角カンマと半角スペース(, )で繋ぐ.全角カンマ(,)ではないので注意.
        \item 他学会の文献テンプレートでは誌名を\textit{Italic}体にしたり巻数を\textbf{Boldface}体にしたりすることがあるが,日本機械学会では全て\textrm{Roman}体で統一する(WebページのURLを\LaTeX{}で出力するときに\texttt{Typewriter}体が用いられることが多いが,これも\textrm{Roman}体にする).
        \item 以上の内容が満たされていて,文献にアクセスするうえで十分な情報が書かれていれば月(\ttmonth)や章(\ttchapter)等は不要.
    \end{itemize}
\end{tcolorbox}
\renewcommand\UrlFont{\ttfamily}
\noindent
\ttauthor や\tttitle などのフィールドについての詳細は第~\ref{ssec:field}節を,各エントリー(\ttarticle や\ttbook など)ごとの詳細は第~\ref{ssec:article}節以降を参照してください.
次に,文献リストを作成する際の著者名表記に関する注意事項です.
\begin{tcolorbox}[enhanced, title=\textgt{文献リスト作成の注意事項(著者名)}, drop fuzzy shadow]
    \begin{itemize}
        \item 著者の氏名は「姓」「名」の順で書き,間にスペース等は入れない(例:松川裕樹).
        \item 英語文献の著者の氏名はFamily nameのみ略記せず,Middle nameやGiven nameはイニシャルで記載する(例:Matsukawa Yuki $\to$ Matsukawa, Y.).
        \item 著者は全員記載する.
        \item 日本語文献の著者数が二人以上の場合は
            \begin{quote}
                松川裕樹, 塚原隆裕        
            \end{quote}
            のように半角カンマと半角スペースを間に入れて繋ぐ.
        \item 英語文献の著者数が二人の場合は
            \begin{quote}
                Matsukawa, Y. and Tsukahara, T.        
            \end{quote}
            のようにandで繋ぎ,著者数が三人以上の場合は
            \begin{quote}
                Araki, R., Bos, W. J. T. and Goto, S.            
            \end{quote}
            のように最後だけandで繋ぐ.最後のandの前にカンマは入れない.
    \end{itemize}
\end{tcolorbox}
\noindent
最後に,文献リスト作成時のその他の注意事項を説明します.
\begin{tcolorbox}[enhanced, title=\textgt{文献リスト作成の注意事項(その他)}, drop fuzzy shadow]
    \begin{itemize}
        \item 英語文献のタイトルは最初の単語の頭文字のみ大文字(固有名詞等は除く).
        \item 誌名・書名は省略せずに記載する.
        \item 巻・号は日本語・英語文献問わずVol.~xx,No.~xxとする.
        \item 発行年は西暦で表記し,括弧で括る.ただし,発行年の前のみ半角カンマは不要.
        \item ページ数は単ページの場合はp.~xxとし,複数ページに亘る場合はpp.~xx--yyとする.
        \item 「2014年以降発行の日本機械学会論文集は通しページを廃止したためDOIを記載するように」との注釈が日本機械学会の原稿テンプレートに書かれているが,\jsmefile ではDOIから生成されたURLを誌名(に相当する箇所)にハイパーリンクとして埋め込むように設定している.
    \end{itemize}
\end{tcolorbox}


\clearpage
\section{引用のコマンドと本文中での表示}
\label{sec:cite}
本文中での文献の引用の仕方は以下の通りです.
\begin{tcolorbox}[enhanced, title=\textgt{本文中での引用の仕方}, drop fuzzy shadow]
    \begin{itemize}
        \item 本文中の引用箇所には,著者名と発行年を記載する(Harvard方式).
        \item 英語文献の場合
        \begin{itemize}
            \item 著者1名:\citet{Reynolds:PhilTransRoySoc1883}または\citep{Reynolds:PhilTransRoySoc1883}
            \item 著者2名:\citet{Schmid:Springer2001}または\citep{Schmid:Springer2001}
            \item 著者3名以上:\citet{Berghout:JFM2020}または\citep{Berghout:JFM2020}
        \end{itemize}
        \item 日本語文献の場合
        \begin{itemize}
            \item 著者1名:\citet{塚原:ながれ2023}または\citep{塚原:ながれ2023}
            \item 著者2名:\citet{塚原:ながれ2015}または\citep{塚原:ながれ2015}
            \item 著者3名以上:\citet{塚原:伝熱2007}または\citep{塚原:伝熱2007}
        \end{itemize}
        \item 発行年が同じである同じ著者からの二つ以上の引用を記載する場合には,発行年の後にa, b, c, \ldots を記載する. \\
        例:\citet{Matsukawa:ICFD2022}, \citet{松川:東北大SENAC2022}
\end{itemize}
\end{tcolorbox}

日本機械学会の規定では本文中での文献の引用の際には,「著者名」と「発行年」を記載し,文献一覧は著者名によりソートするHarvard方式\footnote{文献の引用箇所に番号を付け,引用順に文献一覧を並べる方式をVancouver方式と言います.}が採用されています.
そのため,\jsmefile では\verb|\citet|や\verb|\citealp|等のコマンドを使用して引用することをおすすめします.
Table~\ref{tab:English}--\ref{tab:same_authors}は,本文中で文献を引用する際のコマンドとその出力結果の一覧を示しています.
これらのコマンドを使用するには\verb|natbib.sty|というスタイルファイルを読み込む必要があります.
このテンプレートではプリアンブルに\verb|\usepackage{natbib}|と書くことで\verb|natbib.sty|を読み込んでいます.
\verb|\cite|と\verb|\citet|は出力結果が同じなのでどちらを使っても構いませんが,他のコマンドは細かい箇所(カンマや括弧の有無)に違いがあるので自分の出力したい内容に合わせて使うようにしてください.

\begin{itemize}
    \item \verb|\cite|, \verb|\citet| \\
    \verb|\cite|と\verb|\citet|は文章中で著者名と発行年を\citet{Reynolds:PhilTransRoySoc1883}のように \verb|author (year)| の形で引用する際に使用するコマンドです.
\begin{quote}
入力:
\begin{verbatim}
\citet{Reynolds:PhilTransRoySoc1883}の実験では水の入った円管に染料を流すことにより,整った流れ(層流)と乱れた流れ(乱流)の切り替わりを明らかにした.
\end{verbatim}
出力:\\
    \citet{Reynolds:PhilTransRoySoc1883}の実験では水の入った円管に染料を流すことにより,整った流れ(層流)と乱れた流れ(乱流)の切り替わりを明らかにした.
\end{quote}
    \item \verb|\citep| \\
    著者と発行年を\citep{Reynolds:PhilTransRoySoc1883}のように\verb|(author, year)|の形で引用する際に使用するコマンドです.
    括弧が半角括弧なので英文の中でのみ使用してください.
\begin{quote}
入力:
\begin{verbatim}
Fluid flow can be broadly divided into laminar flow and 
turbulent flow \citep{Reynolds:PhilTransRoySoc1883}.
\end{verbatim}
出力:\\
    Fluid flow can be broadly divided into laminar flow and turbulent flow \citep{Reynolds:PhilTransRoySoc1883}.
\end{quote}
    \item \verb|\citealt| \\
    著者と発行年を\citealt{Reynolds:PhilTransRoySoc1883}のように半角スペースのみで分離し,\verb|author year|の形で出力するコマンドです.
    日本機械学会の書き方には合わないので使用しないでください.
    \item \verb|\citealp| \\
    著者と発行年を\citealp{Reynolds:PhilTransRoySoc1883}のように半角カンマと半角スペースで分離し,\verb|author, year|の形で出力するコマンドです.
    和文中の全角括弧内で使用してください.
\begin{quote}
入力:
\begin{verbatim}
流体流れは層流と乱流に大別することができ,流速が遅い場合には整った流れ(層流)となり,速い場合には乱れた流れ(乱流)となる(\citealp{Reynolds:PhilTransRoySoc1883}).
\end{verbatim}
出力:\\
    流体流れは層流と乱流に大別することができ,流速が遅い場合には整った流れ(層流)となり,速い場合には乱れた流れ(乱流)となる(\citealp{Reynolds:PhilTransRoySoc1883}).
\end{quote}
    \item \verb|\citeauthor| \\
    著者名のみ出力するコマンドです.
    \item \verb|\citeyear| \\
    発行年のみ出力するコマンドです.
    \item \verb|\citeyearpar| \\
    発行年のみを\verb|(year)|の形で出力するコマンドです.
\end{itemize}

\begin{table}[t]
    \centering
    \caption{Commands and output results for English references.}
    \label{tab:English}
    \begin{tabular}{ll}
        コマンド &出力結果 \\
        \multicolumn{2}{c}{------単著------} \\
        \verb|\cite{Reynolds:PhilTransRoySoc1883}|          &\cite{Reynolds:PhilTransRoySoc1883} \\
        \verb|\citet{Reynolds:PhilTransRoySoc1883}|         &\citet{Reynolds:PhilTransRoySoc1883} \\
        \verb|\citet[sec.~1]{Reynolds:PhilTransRoySoc1883}|         &\citet[sec.~1]{Reynolds:PhilTransRoySoc1883} \\
        \verb|\citep{Reynolds:PhilTransRoySoc1883}|         &\citep{Reynolds:PhilTransRoySoc1883} \\
        \verb|\citep[sec.~1]{Reynolds:PhilTransRoySoc1883}|         &\citep[sec.~1]{Reynolds:PhilTransRoySoc1883} \\
        \verb|\citep[see][]{Reynolds:PhilTransRoySoc1883}|         &\citep[see][]{Reynolds:PhilTransRoySoc1883} \\
        \verb|\citep[see][sec.~1]{Reynolds:PhilTransRoySoc1883}|         &\citep[see][sec.~1]{Reynolds:PhilTransRoySoc1883} \\
        \verb|\citealt{Reynolds:PhilTransRoySoc1883}|       &\citealt{Reynolds:PhilTransRoySoc1883} \\
        \verb|\citealp{Reynolds:PhilTransRoySoc1883}|       &\citealp{Reynolds:PhilTransRoySoc1883} \\
        \verb|\citeauthor{Reynolds:PhilTransRoySoc1883}|    &\citeauthor{Reynolds:PhilTransRoySoc1883} \\
        \verb|\citeyear{Reynolds:PhilTransRoySoc1883}|      &\citeyear{Reynolds:PhilTransRoySoc1883} \\
        \verb|\citeyearpar{Reynolds:PhilTransRoySoc1883}|   &\citeyearpar{Reynolds:PhilTransRoySoc1883} \\
        \multicolumn{2}{c}{------著者2名以上------} \\
        \verb|\cite{Matsukawa:PoF2022}|         &\cite{Matsukawa:PoF2022} \\
        \verb|\citet{Matsukawa:PoF2022}|        &\citet{Matsukawa:PoF2022} \\
        \verb|\citet[sec.~3]{Matsukawa:PoF2022}|        &\citet[sec.~3]{Matsukawa:PoF2022} \\
        \verb|\citep{Matsukawa:PoF2022}|        &\citep{Matsukawa:PoF2022} \\
        \verb|\citep[sec.~3]{Matsukawa:PoF2022}|        &\citep[sec.~3]{Matsukawa:PoF2022} \\
        \verb|\citep[see][]{Matsukawa:PoF2022}|        &\citep[see][]{Matsukawa:PoF2022} \\
        \verb|\citep[see][sec.~3]{Matsukawa:PoF2022}|        &\citep[see][sec.~3]{Matsukawa:PoF2022} \\
        \verb|\citealt{Matsukawa:PoF2022}|      &\citealt{Matsukawa:PoF2022} \\
        \verb|\citealp{Matsukawa:PoF2022}|      &\citealp{Matsukawa:PoF2022} \\
        \verb|\citeauthor{Matsukawa:PoF2022}|   &\citeauthor{Matsukawa:PoF2022} \\
        \verb|\citeyear{Matsukawa:PoF2022}|     &\citeyear{Matsukawa:PoF2022} \\
        \verb|\citeyearpar{Matsukawa:PoF2022}|  &\citeyearpar{Matsukawa:PoF2022} \\
        \multicolumn{2}{c}{------著者3名以上------} \\
        \verb|\cite{Berghout:JFM2020}|          &\cite{Berghout:JFM2020} \\
        \verb|\citet{Berghout:JFM2020}|         &\citet{Berghout:JFM2020} \\
        \verb|\citet[sec.~3]{Berghout:JFM2020}|         &\citet[sec.~3]{Berghout:JFM2020} \\
        \verb|\citep{Berghout:JFM2020}|         &\citep{Berghout:JFM2020} \\
        \verb|\citep[sec.~3]{Berghout:JFM2020}|         &\citep[sec.~3]{Berghout:JFM2020} \\
        \verb|\citep[see][]{Berghout:JFM2020}|         &\citep[see][]{Berghout:JFM2020} \\
        \verb|\citep[see][sec.~3]{Berghout:JFM2020}|         &\citep[see][sec.~3]{Berghout:JFM2020} \\
        \verb|\citealt{Berghout:JFM2020}|       &\citealt{Berghout:JFM2020} \\
        \verb|\citealp{Berghout:JFM2020}|       &\citealp{Berghout:JFM2020} \\
        \verb|\citeauthor{Berghout:JFM2020}|    &\citeauthor{Berghout:JFM2020} \\
        \verb|\citeyear{Berghout:JFM2020}|      &\citeyear{Berghout:JFM2020} \\
        \verb|\citeyearpar{Berghout:JFM2020}|   &\citeyearpar{Berghout:JFM2020}
    \end{tabular}
\end{table}


\begin{table}[t]
    \centering
    \caption{Commands and output results for Japanese references.}
    \label{tab:Japanese}
    \begin{tabular}{ll}
        コマンド &出力結果 \\
        \multicolumn{2}{c}{------単著------} \\
        \verb|\cite{松川:修論2023}|            &\cite{松川:修論2023} \\
        \verb|\citet{松川:修論2023}|           &\citet{松川:修論2023} \\
        \verb|\citet[第~3章]{松川:修論2023}|           &\citet[第~3章]{松川:修論2023} \\
        \verb|\citep{松川:修論20233}|           &\citep{松川:修論2023} \\
        \verb|\citealt{松川:修論2023}|         &\citealt{松川:修論2023} \\
        \verb|\citealp{松川:修論2023}|         &\citealp{松川:修論2023} \\
        \verb|\citeauthor{松川:修論2023}|      &\citeauthor{松川:修論2023} \\
        \verb|\citeyear{松川:修論2023}|        &\citeyear{松川:修論2023} \\
        \verb|\citeyearpar{松川:修論2023}|     &\citeyearpar{松川:修論2023} \\
        \multicolumn{2}{c}{------著者2名以上------} \\
        \verb|\cite{塚原:ながれ2015}|            &\cite{塚原:ながれ2015} \\
        \verb|\citet{塚原:ながれ2015}|           &\citet{塚原:ながれ2015} \\
        \verb|\citet[第~3章]{塚原:ながれ2015}|           &\citet[第~3章]{塚原:ながれ2015} \\
        \verb|\citep{塚原:ながれ2015}|           &\citep{塚原:ながれ2015} \\
        \verb|\citealt{塚原:ながれ2015}|         &\citealt{塚原:ながれ2015} \\
        \verb|\citealp{塚原:ながれ2015}|         &\citealp{塚原:ながれ2015} \\
        \verb|\citeauthor{塚原:ながれ2015}|      &\citeauthor{塚原:ながれ2015} \\
        \verb|\citeyear{塚原:ながれ2015}|        &\citeyear{塚原:ながれ2015} \\
        \verb|\citeyearpar{塚原:ながれ2015}|     &\citeyearpar{塚原:ながれ2015} \\
        \multicolumn{2}{c}{------著者3名以上------} \\
        \verb|\cite{塚原:伝熱2007}|             &\cite{塚原:伝熱2007} \\
        \verb|\citet{塚原:伝熱2007}|            &\citet{塚原:伝熱2007} \\
        \verb|\citet[第~3章]{塚原:伝熱2007}|            &\citet[第~3章]{塚原:伝熱2007} \\
        \verb|\citep{塚原:伝熱2007}|            &\citep{塚原:伝熱2007} \\
        \verb|\citealt{塚原:伝熱2007}|          &\citealt{塚原:伝熱2007} \\
        \verb|\citealp{塚原:伝熱2007}|          &\citealp{塚原:伝熱2007} \\
        \verb|\citeauthor{塚原:伝熱2007}|       &\citeauthor{塚原:伝熱2007} \\
        \verb|\citeyear{塚原:伝熱2007}|         &\citeyear{塚原:伝熱2007} \\
        \verb|\citeyearpar{塚原:伝熱2007}|      &\citeyearpar{塚原:伝熱2007}
    \end{tabular}
\end{table}

\clearpage
\begin{table}[t]
    \centering
    \caption{Commands and output results for the multiple references from the different authors.}
    \label{tab:different_authors}
    \begin{tabular}{ll}
        コマンド &出力結果 \\
        \verb|\cite{Schmid:Springer2001,Hirsch:AP2013}|            &\cite{Schmid:Springer2001,Hirsch:AP2013} \\
        \verb|\citet{Schmid:Springer2001,Hirsch:AP2013}|           &\citet{Schmid:Springer2001,Hirsch:AP2013} \\
        \verb|\citep{Schmid:Springer2001,Hirsch:AP2013}|           &\citep{Schmid:Springer2001,Hirsch:AP2013} \\
        \verb|\citep[see][]{Schmid:Springer2001,Hirsch:AP2013}|           &\citep[see][]{Schmid:Springer2001,Hirsch:AP2013} \\
        \verb|\citealt{Schmid:Springer2001,Hirsch:AP2013}|         &\citealt{Schmid:Springer2001,Hirsch:AP2013} \\
        \verb|\citealp{Schmid:Springer2001,Hirsch:AP2013}|         &\citealp{Schmid:Springer2001,Hirsch:AP2013} \\
        \verb|\citeauthor{Schmid:Springer2001,Hirsch:AP2013}|      &\citeauthor{Schmid:Springer2001,Hirsch:AP2013} \\
        \verb|\citeyear{Schmid:Springer2001,Hirsch:AP2013}|        &\citeyear{Schmid:Springer2001,Hirsch:AP2013} \\
        \verb|\citeyearpar{Schmid:Springer2001,Hirsch:AP2013}|     &\citeyearpar{Schmid:Springer2001,Hirsch:AP2013}
    \end{tabular}
\end{table}

\begin{table}[t]
    \centering
    \caption{Commands and output results for the multiple references from the same authors.}
    \label{tab:same_authors}
    \begin{tabular}{ll}
        コマンド &出力結果 \\
        \multicolumn{2}{c}{------英語文献と英語文献------} \\
        \verb|\cite{Matsukawa:ICFD2022,Matsukawa:PoF2022}|            &\cite{Matsukawa:ICFD2022,Matsukawa:PoF2022} \\
        \verb|\citet{Matsukawa:ICFD2022,Matsukawa:PoF2022}|           &\citet{Matsukawa:ICFD2022,Matsukawa:PoF2022} \\
        \verb|\citep{Matsukawa:ICFD2022,Matsukawa:PoF2022}|           &\citep{Matsukawa:ICFD2022,Matsukawa:PoF2022} \\
        \verb|\citep[see][]{Matsukawa:ICFD2022,Matsukawa:PoF2022}|           &\citep[see][]{Matsukawa:ICFD2022,Matsukawa:PoF2022} \\
        \verb|\citealt{Matsukawa:ICFD2022,Matsukawa:PoF2022}|         &\citealt{Matsukawa:ICFD2022,Matsukawa:PoF2022} \\
        \verb|\citealp{Matsukawa:ICFD2022,Matsukawa:PoF2022}|         &\citealp{Matsukawa:ICFD2022,Matsukawa:PoF2022} \\
        \verb|\citeauthor{Matsukawa:ICFD2022,Matsukawa:PoF2022}|      &\citeauthor{Matsukawa:ICFD2022,Matsukawa:PoF2022} \\
        \verb|\citeyear{Matsukawa:ICFD2022,Matsukawa:PoF2022}|        &\citeyear{Matsukawa:ICFD2022,Matsukawa:PoF2022} \\
        \verb|\citeyearpar{Matsukawa:ICFD2022,Matsukawa:PoF2022}|     &\citeyearpar{Matsukawa:ICFD2022,Matsukawa:PoF2022} \\
        \multicolumn{2}{c}{------英語文献と日本語文献------} \\
        \verb|\cite{Matsukawa:ICFD2022,松川:流力年会2022}|            &\cite{Matsukawa:ICFD2022,松川:流力年会2022} \\
        \verb|\citet{Matsukawa:ICFD2022,松川:流力年会2022}|           &\citet{Matsukawa:ICFD2022,松川:流力年会2022} \\
        \verb|\citep{Matsukawa:ICFD2022,松川:流力年会2022}|           &\citep{Matsukawa:ICFD2022,松川:流力年会2022} \\
        \verb|\citep[see][]{Matsukawa:ICFD2022,松川:流力年会2022}|           &\citep[see][]{Matsukawa:ICFD2022,松川:流力年会2022} \\
        \verb|\citealt{Matsukawa:ICFD2022,松川:流力年会2022}|         &\citealt{Matsukawa:ICFD2022,松川:流力年会2022} \\
        \verb|\citealp{Matsukawa:ICFD2022,松川:流力年会2022}|         &\citealp{Matsukawa:ICFD2022,松川:流力年会2022} \\
        \verb|\citeauthor{Matsukawa:ICFD2022,松川:流力年会2022}|      &\citeauthor{Matsukawa:ICFD2022,松川:流力年会2022} \\
        \verb|\citeyear{Matsukawa:ICFD2022,松川:流力年会2022}|        &\citeyear{Matsukawa:ICFD2022,松川:流力年会2022} \\
        \verb|\citeyearpar{Matsukawa:ICFD2022,松川:流力年会2022}|     &\citeyearpar{Matsukawa:ICFD2022,松川:流力年会2022}
    \end{tabular}
\end{table}


\clearpage
\section{書誌情報ファイル(\texttt{bib}ファイル)の作り方}
\label{sec:bib}
ここでは書誌情報ファイル(\texttt{bib}ファイル)の作り方,使い方を説明します.
この\verb|pdf|の末尾の参考文献リストは同じディレクトリにある\verb|mybib_en.bib|と\verb|mybib_jp.bib|から作成しているので参考にしてください.
\verb|bib|ファイルに入力する書誌情報は次のような構造になっています.
\begin{tcolorbox}[enhanced, title=\textgt{\texttt{bib}ファイル内の書誌情報の構造}, drop fuzzy shadow]
\begin{verbatim}
@エントリー名{参照キー,
    フィールド1 = {},
    フィールド2 = {},
    フィールド3 = {}
}
\end{verbatim}
\end{tcolorbox}
\noindent
だいたいの雑誌論文のwebサイトでは\BibTeX{}形式で書誌情報を出力できる機能があるのでそこから\verb|bib|ファイルをダウンロードします.
もちろん,ダウンロードした\verb|bib|ファイルを自分で書き換えることもできますし,自分で一から\verb|bib|ファイルを作成することも可能です.

第~\ref{sec:cite}節で述べたように,文献を本文中で引用する際は\verb|\citet{Matsukawa:PoF2022}|のように書きます.
このときの\verb|Matsukawa:PoF2022|が参照キーです.
参照キーの書き方に特に規則は無く,半角カンマ以外の半角記号も使用可能です.
ただ,自分の中でマイルールを設けておくと引用する際に楽です.
私の場合は原則として\texttt{\colorbox[gray]{0.8}{著者}:\colorbox[gray]{0.8}{誌名}\colorbox[gray]{0.8}{年}}としています.
また,エントリー名とフィールド名は大文字と小文字を区別しませんが,参照キーは区別するので気をつけてください.

エントリーは雑誌論文や学位論文といった,その文献の該当する種別を表します.
\jsmefile でサポートされているエントリーは\hyperref[ssec:article]{\ttarticle}, \hyperref[ssec:book]{\ttbook}, \hyperref[ssec:booklet]{\ttbooklet}, \hyperref[ssec:comment]{\ttcomment}, \hyperref[ssec:conference]{\ttconference}, \hyperref[ssec:inbook]{\ttinbook}, \hyperref[ssec:incollection]{\ttincollection}, \hyperref[ssec:inproceedings]{\ttinproceedings}, \hyperref[ssec:manual]{\ttmanual}, \hyperref[ssec:mastersthesis]{\ttmastersthesis}, \hyperref[ssec:misc]{\ttmisc}, \hyperref[ssec:online]{\ttonline}, \hyperref[ssec:phdthesis]{\ttphdthesis}, \hyperref[ssec:proceedings]{\ttproceedings}, \hyperref[ssec:techreport]{\tttechreport}, \hyperref[ssec:unpublished]{\ttunpublished}の16種類です.
それぞれのエントリーで必須となるフィールドが異なり,文献一覧への出力の方法も異なるので面倒くさがらずに分類しましょう.
ただ,全ての文献を正確に分類することは難しく,判断が人により異なることもあります.
\JSMErepos における分類の仕方も見る人によっては違和感を覚えるものがあるかもしれません.
それぞれのエントリーに関する詳細を\pageref{ssec:article}ページ以降で説明しているので参考にしてください.
ただし,ここで説明する内容は一般的な\BibTeX{}におけるエントリーの説明と若干異なる箇所があるのでご了承ください.

フィールドはその文献の著者情報や誌名情報を入力するデータ項目です.
\jsmefile では\ttaccess, \ttaddress, \ttarchivePrefix, \ttauthor, \ttbooktitle, \ttchapter, \ttdoi, \ttedition, \tteditor, \tteprint, \tthowpublished, \ttinstitution, \ttjournal, \ttkey, \ttlangid, \ttlanguage, \ttmonth, \ttnote, \ttnumber, \ttorganization, \ttpages, \ttpublisher, \ttschool, \ttseries, \tttitle, \tttype, \tturl, \ttvolume, \ttyear, \ttyomi がサポートされています.
フィールドの値は
\begin{verbatim}
    author = {Matsukawa, Yuki and Tsukahara, Takahiro}
    author = "Matsukawa, Yuki and Tsukahara, Takahiro"
\end{verbatim}
のように\verb|{ }|または\verb|" "|で囲います.
フィールドの詳細は以下の通りです(エントリーの分類同様,人によって意見が異なる場合があります).
また,不要なフィールドがあっても無視されるだけなので邪魔であれば消しても構いません.

\subsection{各フィールドの詳細}
\label{ssec:field}
\begin{itemize}
    \item \ttaccess \\
        Webページを閲覧した日付(参照日)を記入します.
        \ttonline でのみ有効です.
        書き方の詳細は第~\ref{ssec:online}節(\ttonline )を参照.
    \item \ttaddress \\
        出版社(\ttpublisher)の住所.
        日本機械学会の原稿テンプレートでは住所を書かないので必要ありません.
    \item \ttarchivePrefix \\
        arXiv上の文献を引用する際に自動で出力されます.
        \jsmefile では\ttarchivePrefix があるとarXivの文献として判断します.
    \item \ttauthor \\
        文献の著者情報を入力します.
        他の\BibTeX{}スタイルファイルと仕様が異なる場合があるので注意が必要です.
        日本語文献でも英語文献でも
        \begin{verbatim}
        author = {Family, Given and Family, Given and Family, Given}
        author = {Given Family and Given Family and Given Family}
        \end{verbatim}
        の形式で書いてください.
        例えば\citet{Tsukahara:TSFP2005}と\citet{堀本:可視化情報2020}の場合だと
        \begin{verbatim}
        author = {Tsukahara, Takahiro and Seki, Yohji and 
                    Kawamura, Hiroshi and Tochio, Daisuke}
        author = {堀本, 康文 and 川口, 靖夫 and 塚原, 隆裕}
        \end{verbatim}
        のようになります.
        \verb|Family, Given|で書く場合は日本語文献でも半角カンマと半角スペースで姓と名を区切ります.
        また,著者が複数いる場合は\verb| and |で著者を区切ります.
        日本語文献の場合は後述の\ttyomi フィールドで読み方を指定してください.
    \item \ttbooktitle \\
        書籍の名前ですが,\hyperref[ssec:conference]{\ttconference}, \hyperref[ssec:incollection]{\ttincollection}, \hyperref[ssec:inproceedings]{\ttinproceedings}で使われることからわかるように,引用する文献が書籍のうちの一部である場合の書籍そのものの題名を書きます.
        例えば,\citet{Lueptow:Springer2000}はそれ単独でStability and experimental velocity field in Taylor--Couette flow with axial and radial flowという題目(\tttitle)を持っていますが,これはPhysics of Rotating Fluidsという書籍(\ttbooktitle)の一部です.
    \item \ttchapter \\
        書籍の一部の章を引用するときに使用します.
        日本機械学会では章番号ではなくページ数で引用するので使う必要はありません.
    \item \ttdoi \\
        文献のデジタルオブジェクト識別子(Digital Object Identifier, DOI)を入力します.
        学術論文だけでなくSpringer等の書籍などにもDOIが割り当てられています.
        \jsmefile では\ttdoi フィールドの値を読み込んで誌名(に相当する箇所)にハイパーリンクを埋め込むようにしています.
    \item \ttedition \\
        (第3版などの)版を入力します.
        \citet{奥村:技評2020}のように\verb|title = {[改訂第8版]\LaTeXe{}美文書作成入門}|と\tttitle の中に書いてしまってもいいと思います.
    \item \tteditor \\
        編者名を入力します.
        書き方は著者名(\ttauthor)と同じですが,\jsmefile では\tteditor に値を入れても基本的には文献一覧には出力されません.
        ただし,著者(\ttauthor)無し文献の場合,\verb|natbib|による文章中での引用がおかしくなるのでその際は\ttauthor 代わりに\tteditor を使用します(詳細は\hyperref[ssec:proceedings]{\ttproceedings}を参照).
        また,日本機械学会原稿テンプレートでは『伝熱ハンドブック』の編者として日本機械学会を指定し,その後ろに「編」と入れるように書かれていますが,現状これを\jsmefile で再現する際は\tteditor を使用せずに\verb|author = {日本機械学会編}|とするのが無難です.
        近い将来修正して\tteditor で対応できるようにしたいです.
    \item \tteprint \\
        論文のeprintを入力します.
        arXivから\BibTeX{}形式で書誌情報を出力するとデフォルトで入ってくるフィールドです.
        \jsmefile では\ttarchivePrefix フィールドがあった場合に\tteprint の情報からarXivの該当論文へのハイパーリンクを生成します(詳細は第~\ref{ssec:misc}節\ttmisc を参照).
    \item \tthowpublished \\
        特殊な出版形態をとる場合の説明を入力します.
        また,第~\ref{ssec:misc}節では学部の卒業論文を\ttmisc に分類する際に\tthowpublished に「xx大学xx学部xx学科卒業論文」と書くことにしています.
    \item \ttinstitution \\
        技術報告書(\hyperref[ssec:techreport]{\tttechreport})でのみ使用されるフィールドです.
        報告書が出された機関名を入力します.
    \item \ttjournal \\
        学術雑誌論文が出された誌名を入力します.
        \hyperref[ssec:article]{\ttarticle}でのみ有効なフィールドです.
    \item \ttkey \\
        著者名に相当するものが無い場合,ソートに利用します.
        \jsmefile では動作が完璧ではないのであまり使わないでほしいです.
    \item \ttlangid, \ttlanguage \\
        その文献が書かれている言語を指定するフィールドです.
        \verb|japanese|と書かれていると日本語文献として処理されますが,指定されていなくても基本的には自動で日本語文献を判別してくれます\footnote{\ttauthor や\tttitle などのフィールドに日本語が含まれていると\texttt{is.kanji.str\$}という機能で日本語文献と判断します.}.
        \verb|biblatex|では\ttlangid フィールドが使用されるので残しておいてもいいかもしれません.
        \JSMErepos もそのうち\verb|biblatex|対応バージョンを作成したいと考えています.
        また,TeX Live 2020以前の\upBibTeX{}では\verb|is.kanji.str$|が適切に動作しないため,\ttlangid を使用してください\footnote{詳細は\href{https://github.com/ShiroTakeda/jecon-bst}{\texttt{jecon-bst (GitHub)}}の\texttt{jecon-example.pdf}を確認してください.}.
    \item \ttmonth \\
        出版された月を入力します.
        日本機械学会では年(\ttyear)だけ書けばいいので\ttmonth は不要.
    \item \ttnote \\
        注記.
        \jsmefile では講演番号や論文番号をここに書いてください.
    \item \ttnumber \\
        雑誌等の号数を入力します.
    \item \ttorganization \\
        会議の主催者・団体やマニュアルを出している機関を入力します.
    \item \ttpages \\
        ページ番号を入力します.
        例えば35ページのみ引用する場合は\verb|pages = {35}|とし,35ページから64ページまでを引用する場合は\verb|pages = {35--64}|とします.
        このとき,開始ページと終了ページを結ぶ横棒はハイフン(\verb|-|)ではなくenダッシュとします(\verb|--|).
        ただし,\jsmefile を使用すれば\ttpages 内でハイフンとしていても自動でenダッシュに変換してくれるので安心です.
    \item \ttpublisher \\
        出版社の名前を入力します.
    \item \ttschool \\
        学校名を入力します.
        \hyperref[ssec:mastersthesis]{\ttmastersthesis}と\hyperref[ssec:phdthesis]{\ttphdthesis}でのみ有効なフィールドです.
        学部の卒業論文の場合は\hyperref[ssec:misc]{\ttmisc}に分類し,\ttschool フィールドの代わりに\tthowpublished を使用してください.
    \item \ttseries \\
        書籍のシリーズを入力します.
        \hyperref[ssec:book]{\ttbook}と\hyperref[ssec:inbook]{\ttinbook}でのみ有効なフィールドです.
        日本機械学会では不要かと思います.
    \item \tttitle \\
        文献のタイトルを入力します.
        \jsmefile では標準設定として,英語文献の場合はタイトル冒頭以外のアルファベットを全て小文字に変換して出力します(日本機械学会の原稿テンプレート通りの出力).
        ただし,固有名詞や二次元を表す2Dなどのようにタイトルの途中で大文字を使用する場合は
        \begin{verbatim}
        title = {Subcritical transition of {Taylor--Couette--Poiseuille} flow 
                    at high radius ratio}
        title = {A mathematical consideration of vortex thinning in {2D} turbulence}
        \end{verbatim}
        のように\verb|{ }|で囲めば該当箇所はそのままの形で出力してくれます(\citealp{Matsukawa:PoF2022,Yoneda:arXiv2016}).
    \item \tttype \\
        \hyperref[ssec:techreport]{\tttechreport}でのみ有効なフィールドです.
        今後仕様を変えるかもしれませんが,\jsmefile では使わないでください.
    \item \tturl \\
        文献やwebページのURLを入力します.
        \ttdoi フィールドと\tturl フィールドの両方に値が入っていた場合は\ttdoi の内容を優先してハイパーリンクを埋め込みます.
    \item \ttvolume \\
        雑誌等の巻数(第3巻,Vol.~3など)を入力します.
    \item \ttyear \\
        発行年を入力します.
        学位論文の場合は修了「年」を記入します(第~\ref{ssec:mastersthesis}節\ttmastersthesis を参照).
    \item \ttyomi \\
        著者(\ttauthor)の読みを入力します.
        \ttyomi の内容から判断してアルファベット順に文献をソートしてくれます.
        \begin{verbatim}
        yomi = {Matsukawa, Yuki and Tsukahara, Takahiro}
        \end{verbatim}
        のようにローマ字で読みを書くと,英語文献と日本語文献を混ぜてアルファベット順でソートしてくれます.
        ひらがなで読みを書くと日本語文献と英語文献が分けられてしまうので日本機械学会の規定に合いません.
\end{itemize}


\subsection{\ttarticle}
\label{ssec:article}
\begin{tcolorbox}[enhanced, title=\ttarticle, drop fuzzy shadow]
    \begin{itemize}
        \item 必須項目 \\
        \ttauthor, \tttitle, \ttjournal, \ttyear
        \item オプション項目 \\
        \ttvolume, \ttnumber, \ttpages, \ttmonth, \ttnote, \ttkey, \ttdoi, \tturl
        \item 出力(英語文献) \\
            \colorbox[gray]{0.8}{\ttauthorf}, \colorbox[gray]{0.8}{\ttauthors} and \colorbox[gray]{0.8}{\ttauthort}, \colorbox[gray]{0.8}{\tttitle}, \colorbox[gray]{0.8}{\ttjournal} (\colorbox[gray]{0.8}{\ttyear}), Vol.~\colorbox[gray]{0.8}{\ttvolume}, No.~\colorbox[gray]{0.8}{\ttnumber}, pp.~\colorbox[gray]{0.8}{\ttpages}, \colorbox[gray]{0.8}{\ttnote}.
        \item 出力(日本語文献) \\
            \colorbox[gray]{0.8}{\ttauthorf}, \colorbox[gray]{0.8}{\ttauthors}, \colorbox[gray]{0.8}{\ttauthort}, \colorbox[gray]{0.8}{\tttitle}, \colorbox[gray]{0.8}{\ttjournal} (\colorbox[gray]{0.8}{\ttyear}), Vol.~\colorbox[gray]{0.8}{\ttvolume}, No.~\colorbox[gray]{0.8}{\ttnumber}, pp.~\colorbox[gray]{0.8}{\ttpages}, \colorbox[gray]{0.8}{\ttnote}.
        \item \verb|bib|ファイル作成例 \vspace{-3mm}
\begin{verbatim}
@article{Matsukawa:PoF2022,
    author  = {Matsukawa, Yuki and Tsukahara, Takahiro},
    title   = {Subcritical transition of {Taylor--Couette--Poiseuille} flow 
                at high radius ratio},
    journal = {Physics of Fluids},
    volume  = {34},
    number  = {7},
    year    = {2022},
    doi     = {10.1063/5.0096676},
    url     = {https://doi.org/10.1063/5.0096676},
    note    = {074109}
}
\end{verbatim}
    \end{itemize}

\end{tcolorbox}

\ttarticle は雑誌に掲載された論文です.
流体力学分野では英文雑誌だとJournal of Fluid Mechanics\footnote{Journal of Fluid Mechanics, \textless\url{https://www.cambridge.org/core/journals/journal-of-fluid-mechanics}\textgreater}やPhysics of Fluids\footnote{Physics of Fluids, \textless\url{https://pubs.aip.org/aip/pof}\textgreater}などが挙げられます.
国内雑誌だと日本機械学会誌\footnote{日本機械学会誌, \textless\url{https://www.jsme.or.jp/publication/kaisi/}\textgreater}や日本流体力学会誌『ながれ』\footnote{日本流体力学会誌『ながれ』, \textless\url{https://www.nagare.or.jp/publication/nagare.html}\textgreater}などが該当します.
\ttdoi と\tturl のフィールドは\jsmefile ではどちらも\ttjournal にハイパーリンクを埋め込む操作を行うのでどちらを使っても同様の結果を得られます.
\ttdoi も\tturl も無い場合は\ttjournal にハイパーリンクを埋め込みません.
また,ジャーナルによっては書誌情報を\BibTeX{}形式で出力した際に論文番号が\ttpages となっています.
上記のPhysics of Fluidsの場合(\citealp{Matsukawa:PoF2022})だとデフォルトで\verb|pages = {074109}|となっているため,このままだと
\begin{quote}
    Matsukawa, Y. and Tsukahara, T., Subcritical transition of Taylor--Couette--Poiseuille flow at high radius ratio, \href{https://doi.org/10.1063/5.0096676}{Physics of Fluids}, Vol.~34, No.~7 (2022b), \textcolor{red}{p.~074109}.
\end{quote}
と出力され,違和感があります.
そのため,論文番号や講演番号は\ttnote に入れるようにしましょう.


\subsection{\ttbook}
\label{ssec:book}
\begin{tcolorbox}[enhanced, title=\ttbook, drop fuzzy shadow]
    \begin{itemize}
        \item 必須項目 \\
        \ttauthor / \tteditor, \tttitle, \ttpublisher, \ttyear
        \item オプション項目 \\
        \ttvolume, \ttnumber, \ttseries, \ttaddress, \ttedition, \ttmonth, \ttnote, \ttkey, \ttdoi, \tturl
        \item 出力(英語文献) \\
            \colorbox[gray]{0.8}{\ttauthorf}, \colorbox[gray]{0.8}{\ttauthors} and \colorbox[gray]{0.8}{\ttauthort}, \colorbox[gray]{0.8}{\tttitle}, \colorbox[gray]{0.8}{\ttpublisher} (\colorbox[gray]{0.8}{\ttyear}), \colorbox[gray]{0.8}{\ttnote}.
        \item 出力(日本語文献) \\
            \colorbox[gray]{0.8}{\ttauthorf}, \colorbox[gray]{0.8}{\ttauthors}, \colorbox[gray]{0.8}{\ttauthort}, \colorbox[gray]{0.8}{\tttitle}, \colorbox[gray]{0.8}{\ttpublisher} (\colorbox[gray]{0.8}{\ttyear}), \colorbox[gray]{0.8}{\ttnote}.
        \item \verb|bib|ファイル作成例 \vspace{-3mm}
\begin{verbatim}
@book{Schmid:Springer2001,
    author      = {Peter J. Schmid and Dan S. Henningson},
    title       = {Stability and Transition in Shear Flows},
    publisher   = {Springer New York},
    year        = {2001},
    doi         = {10.1007/978-1-4613-0185-1}
}
\end{verbatim}
    \end{itemize}
\end{tcolorbox}

出版社が刊行した書籍を引用する際は\ttbook を使います.
似たエントリーとして\hyperref[ssec:inbook]{\ttinbook}がありますが,特定のページを参照したのではなく,書籍全体を参照した場合は\ttbook を使いましょう.
\JSMErepos ではSpringer\footnote{Springer, \textless\url{https://www.springer.com/jp}\textgreater}や朝倉書店\footnote{朝倉書店, \textless\url{https://www.asakura.co.jp/}\textgreater},丸善出版\footnote{丸善出版, \textless\url{https://www.maruzen-publishing.co.jp/}\textgreater}等の出版社から出た書籍を\ttbook に分類しています.
\ttpublisher フィールドにはこれらの出版社名を入れましょう.

\subsection{\ttbooklet}
\label{ssec:booklet}
\begin{tcolorbox}[enhanced, title=\ttbooklet, drop fuzzy shadow]
    \begin{itemize}
        \item 必須項目 \\
        \tttitle
        \item オプション項目 \\
        \ttauthor, \tthowpublished, \ttaddress, \ttmonth, \ttyear, \ttnote, \ttkey, \ttdoi, \tturl
        \item 出力(英語文献) \\
            \colorbox[gray]{0.8}{\ttauthorf}, \colorbox[gray]{0.8}{\ttauthors} and \colorbox[gray]{0.8}{\ttauthort}, \colorbox[gray]{0.8}{\tttitle}, \colorbox[gray]{0.8}{\tthowpublished} (\colorbox[gray]{0.8}{\ttyear}), \colorbox[gray]{0.8}{\ttnote}.
        \item 出力(日本語文献) \\
            \colorbox[gray]{0.8}{\ttauthorf}, \colorbox[gray]{0.8}{\ttauthors}, \colorbox[gray]{0.8}{\ttauthort}, \colorbox[gray]{0.8}{\tttitle}, \colorbox[gray]{0.8}{\tthowpublished} (\colorbox[gray]{0.8}{\ttyear}), \colorbox[gray]{0.8}{\ttnote}.
        \item \verb|bib|ファイル作成例 \vspace{-3mm}
\begin{verbatim}
@booklet{Wang:MEnews2014,
    author          = {Wang, Lin},
    title           = {Exchange student from 
                        {Northwestern Polytechnical University (China)}},
    howpublished    = {ME Newsletter, Department of Mechanical Engineering, 
                        Tokyo University of Science},
    year            = {2014},
    url             = {https://www.rs.tus.ac.jp/me/pdf/newsletter/ME_NL_No15.pdf}
}
\end{verbatim}
    \end{itemize}
\end{tcolorbox}

\ttbooklet は使う機会が少ないため分類が難しいエントリーですが,出版社が明記されていないような(薄い)冊子媒体が該当します.
\JSMErepos では例として東京理科大学理工学部機械工学科(現・創域理工学部機械航空宇宙工学科)が毎年出しているMEニュースレター\footnote{MEニュースレター, \textless\url{https://www.rs.tus.ac.jp/me/newsletter.html}\textgreater}という広報の冊子を引用しました.

\subsection{\ttcomment}
\label{ssec:comment}
\begin{tcolorbox}[enhanced, title=\ttcomment, drop fuzzy shadow]
    \begin{itemize}
        \item 必須項目 \\
        無し
        \item オプション項目 \\
        無し
        \item 出力 \\
        出力されない
        \item \verb|bib|ファイル作成例 \vspace{-3mm}
\begin{verbatim}
@comment{
%%%%%%%%%%%%%%%%%%%
%%%%% 英語文献 %%%%%
%%%%%%%%%%%%%%%%%%%
}
\end{verbatim}
    \end{itemize}
\end{tcolorbox}

\verb|bib|ファイル内でコメントを書く場合に用います.
通常の\verb|tex|ファイルや\verb|sty|ファイル内では\verb|%|を書くことでその行はコメントアウトされますが,\verb|bib|ファイル内では使えません.


\subsection{\ttconference}
\label{ssec:conference}
\hyperref[ssec:inproceedings]{\ttinproceedings}と同様なので省略.
Scribeというシステムとの互換性のために残されているらしい(\citealp{奥村:技評2020}).


\subsection{\ttinbook}
\label{ssec:inbook}
\begin{tcolorbox}[enhanced, title=\ttinbook, drop fuzzy shadow]
    \begin{itemize}
        \item 必須項目 \\
        \ttauthor / \tteditor, \tttitle, \ttchapter / \ttpages, \ttpublisher, \ttyear
        \item オプション項目 \\
        \ttvolume, \ttseries, \ttaddress, \ttedition, \ttmonth, \ttnote, \ttkey, \ttdoi, \tturl
        \item 出力(英語文献) \\
            \colorbox[gray]{0.8}{\ttauthorf}, \colorbox[gray]{0.8}{\ttauthors} and \colorbox[gray]{0.8}{\ttauthort}, \colorbox[gray]{0.8}{\tttitle}, \colorbox[gray]{0.8}{\ttpublisher} (\colorbox[gray]{0.8}{\ttyear}), \colorbox[gray]{0.8}{\ttpages}, \colorbox[gray]{0.8}{\ttnote}.
        \item 出力(日本語文献) \\
            \colorbox[gray]{0.8}{\ttauthorf}, \colorbox[gray]{0.8}{\ttauthors}, \colorbox[gray]{0.8}{\ttauthort}, \colorbox[gray]{0.8}{\tttitle}, \colorbox[gray]{0.8}{\ttpublisher} (\colorbox[gray]{0.8}{\ttyear}), \colorbox[gray]{0.8}{\ttpages}, \colorbox[gray]{0.8}{\ttnote}.
        \item \verb|bib|ファイル作成例 \vspace{-3mm}
\begin{verbatim}
@inbook{Davidson:Oxford2015,
    author      = {Peter A. Davidson},
    title       = {Turbulence: 
                    An Introduction for Scientists and Engineers, Second Edition},
    publisher   = {Oxford University Press},
    year        = {2015},
    pages       = {61--104}
}
\end{verbatim}
    \end{itemize}
\end{tcolorbox}

\hyperref[ssec:book]{\ttbook}に似ていますが,\ttbook が書籍丸々一冊なのに対して\ttinbook は書籍中の一部から引用する場合に使用します.
そのため,\ttbook と異なり\ttpages フィールドが使用可能です.


\subsection{\ttincollection}
\label{ssec:incollection}
\begin{tcolorbox}[enhanced, title=\ttincollection, drop fuzzy shadow]
    \begin{itemize}
        \item 必須項目 \\
        \ttauthor, \tttitle, \ttbooktitle, \ttyear
        \item オプション項目 \\
        \tteditor, \ttpages, \ttorganization, \ttpublisher, \ttaddress, \ttmonth, \ttnote, \ttkey, \ttdoi, \tturl
        \item 出力(英語文献) \\
            \colorbox[gray]{0.8}{\ttauthorf}, \colorbox[gray]{0.8}{\ttauthors} and \colorbox[gray]{0.8}{\ttauthort}, \colorbox[gray]{0.8}{\tttitle}, \colorbox[gray]{0.8}{\ttbooktitle}, \colorbox[gray]{0.8}{\ttpublisher} (\colorbox[gray]{0.8}{\ttyear}), \colorbox[gray]{0.8}{\ttpages}, \colorbox[gray]{0.8}{\ttnote}.
        \item 出力(日本語文献) \\
            \colorbox[gray]{0.8}{\ttauthorf}, \colorbox[gray]{0.8}{\ttauthors}, \colorbox[gray]{0.8}{\ttauthort}, \colorbox[gray]{0.8}{\tttitle}, \colorbox[gray]{0.8}{\ttbooktitle}, \colorbox[gray]{0.8}{\ttpublisher} (\colorbox[gray]{0.8}{\ttyear}), \colorbox[gray]{0.8}{\ttpages}, \colorbox[gray]{0.8}{\ttnote}.
        \item \verb|bib|ファイル作成例 \vspace{-3mm}
\begin{verbatim}
@incollection{Lueptow:Springer2000,
    author      = {Lueptow, Richard M.},
    title       = {Stability and experimental velocity field 
                    in {Taylor--Couette} flow with axial and radial flow},
    booktitle   = {Physics of Rotating Fluids},
    publisher   = {Springer-Verlag Berlin Heidelberg New York},
    pages       = {137--155},
    year        = {2000},
    doi         = {10.1007/3-540-45549-3}
}
\end{verbatim}
    \end{itemize}
\end{tcolorbox}

\ttincollection は分類が難しいエントリーの一つです.
これは書籍の一部からの引用ですが,\hyperref[ssec:inbook]{\ttinbook}と異なる点は,引用箇所が独立して表題を持っているようなものを指します.
学会等があるテーマについて組んだ特集といったイメージです.
上記の\citet{Lueptow:Springer2000}の例では,それ単独でStability and experimental velocity field in Taylor--Couette flow with axial and radial flowという題目(\tttitle)を持っていますが,これはPhysics of Rotating Fluidsという書籍(\ttbooktitle)の一部です.
日本語文献では京都大学数理解析研究所の講究録\footnote{京都大学数理解析研究所(RIMS)講究録, \textless\url{https://www.kurims.kyoto-u.ac.jp/ja/kokyuroku.html}\textgreater}や文部科学省科学研究補助金における特定の新学術領域研究の研究成果報告書等を\ttincollection に分類しています.
それは\hyperref[ssec:inproceedings]{\ttinproceedings}だろとか\hyperref[ssec:techreport]{\tttechreport}だろとか言われそうな気もします.

\subsection{\ttinproceedings}
\label{ssec:inproceedings}
\begin{tcolorbox}[enhanced, title=\ttinproceedings, drop fuzzy shadow]
    \begin{itemize}
        \item 必須項目 \\
        \ttauthor, \tttitle, \ttbooktitle, \ttyear
        \item オプション項目 \\
        \tteditor, \ttpages, \ttorganization, \ttpublisher, \ttaddress, \ttmonth, \ttnote, \ttkey, \ttdoi, \tturl
        \item 出力(英語文献) \\
            \colorbox[gray]{0.8}{\ttauthorf}, \colorbox[gray]{0.8}{\ttauthors} and \colorbox[gray]{0.8}{\ttauthort}, \colorbox[gray]{0.8}{\tttitle}, \colorbox[gray]{0.8}{\ttbooktitle} (\colorbox[gray]{0.8}{\ttyear}), \colorbox[gray]{0.8}{\ttnote}.
        \item 出力(日本語文献) \\
            \colorbox[gray]{0.8}{\ttauthorf}, \colorbox[gray]{0.8}{\ttauthors}, \colorbox[gray]{0.8}{\ttauthort}, \colorbox[gray]{0.8}{\tttitle}, \colorbox[gray]{0.8}{\ttbooktitle} (\colorbox[gray]{0.8}{\ttyear}), \colorbox[gray]{0.8}{\ttnote}.
        \item \verb|bib|ファイル作成例 \vspace{-3mm}
\begin{verbatim}
@inproceedings{Matsukawa:ICFD2022,
    author      = {Matsukawa, Yuki and Tsukahara, Takahiro},
    title       = {Laminarization in Subcritical {Taylor--Couette--Poiseuille} Flow 
                    with Increasing Pressure Gradient},
    booktitle   = {Proceedings of 19th International Conference on Flow Dynamics},
    year        = {2022},
    note        = {OS15-10}
}
\end{verbatim}
    \end{itemize}
\end{tcolorbox}

\hyperref[ssec:conference]{\ttconference}と同様です.
学会等の講演論文集の一部を引用するときに\ttinproceedings を使用します.
\hyperref[ssec:article]{\ttarticle}や\hyperref[ssec:inbook]{\ttinbook}と並んで使用頻度の高いエントリーだと思います.
\ttbooktitle フィールドには日本語なら\texttt{booktitle = \{\colorbox[gray]{0.8}{学会名}講演論文集\}},英語なら\texttt{booktitle = \{Proceedings of \colorbox[gray]{0.8}{学会名}\}}と入力しましょう.
\hyperref[ssec:proceedings]{\ttproceedings}は講演論文集全体を引用しているのに対して,\ttinproceedings は講演論文集の中の一講演を引用しています.
上の例では\ttnote に講演番号を入れています.


\subsection{\ttmanual}
\label{ssec:manual}
\begin{tcolorbox}[enhanced, title=\ttmanual, drop fuzzy shadow]
    \begin{itemize}
        \item 必須項目 \\
        \tttitle
        \item オプション項目 \\
        \ttauthor, \ttorganization, \ttaddress, \ttedition, \ttmonth, \ttyear, \ttnote, \ttkey, \ttdoi, \tturl
        \item 出力(英語文献) \\
            \colorbox[gray]{0.8}{\ttauthorf}, \colorbox[gray]{0.8}{\ttauthors} and \colorbox[gray]{0.8}{\ttauthort}, \colorbox[gray]{0.8}{\tttitle} (\colorbox[gray]{0.8}{\ttyear}), \colorbox[gray]{0.8}{\ttnote}.
        \item 出力(日本語文献) \\
            \colorbox[gray]{0.8}{\ttauthorf}, \colorbox[gray]{0.8}{\ttauthors}, \colorbox[gray]{0.8}{\ttauthort}, \colorbox[gray]{0.8}{\tttitle} (\colorbox[gray]{0.8}{\ttyear}), \colorbox[gray]{0.8}{\ttnote}.
        \item \verb|bib|ファイル作成例 \vspace{-3mm}
\begin{verbatim}
@manual{Tecplot2023,
    author  = "{Tecplot, Inc.}",
    title   = {Tecplot 360 Getting Started Manual},
    year    = {2023},
    url    
     = {https://tecplot.azureedge.net/products/360/current/360_getting_started.pdf}
}
\end{verbatim}
    \end{itemize}
\end{tcolorbox}

マニュアルや技術文書は\ttmanual に分類しましょう.
ただし,\ttauthor フィールドが必須項目ではないので,\jsmefile では企業名を\ttauthor に入れています.
これはソートの際に\ttauthor フィールドが空欄だと参考文献一覧の変な場所に配置されてしまうからです.
本来であれば\ttkey フィールドで解決したいところですが,\jsmefile では動作が不安定なので避けたいところです.
実は\verb|natbib.sty|において著者無し文献の本文中での引用は課題が残されているようです\footnote{詳細は\texttt{natbib.sty}の公式ドキュメント\textless\url{https://ctan.org/pkg/natbib}\textgreater をご覧ください.}.
\ttmanual での不具合回避方法としては一旦,マニュアルを発行している機関・企業を\ttauthor とすることで解決としますが,著者無し文献の不具合回避方法は第~\ref{ssec:proceedings}節\ttproceedings でも説明します.


\subsection{\ttmastersthesis}
\label{ssec:mastersthesis}
\begin{tcolorbox}[enhanced, title=\ttmastersthesis, drop fuzzy shadow]
    \begin{itemize}
        \item 必須項目 \\
        \ttauthor, \tttitle, \ttschool, \ttyear
        \item オプション項目 \\
        \ttaddress, \ttmonth, \ttnote, \ttkey, \ttdoi, \tturl
        \item 出力(英語文献) \\
            \colorbox[gray]{0.8}{\ttauthorf}, \colorbox[gray]{0.8}{\ttauthors} and \colorbox[gray]{0.8}{\ttauthort}, \colorbox[gray]{0.8}{\tttitle}, Master's thesis, \colorbox[gray]{0.8}{\ttschool} (\colorbox[gray]{0.8}{\ttyear}), \colorbox[gray]{0.8}{\ttnote}.
        \item 出力(日本語文献) \\
            \colorbox[gray]{0.8}{\ttauthorf}, \colorbox[gray]{0.8}{\ttauthors}, \colorbox[gray]{0.8}{\ttauthort}, \colorbox[gray]{0.8}{\tttitle}, \colorbox[gray]{0.8}{\ttschool}修士論文 (\colorbox[gray]{0.8}{\ttyear}), \colorbox[gray]{0.8}{\ttnote}.
        \item \verb|bib|ファイル作成例 \vspace{-3mm}
\begin{verbatim}
@mastersthesis{松川:修論2023,
    author  = {松川, 裕樹},
    yomi    = {Matsukawa, Yuki},
    title   = {直接数値解析を用いた高円筒比
                Taylor--Couette--Poiseuille流の流動状態遷移過程の分類},
    school  = {東京理科大学大学院理工学研究科機械工学専攻},
    year    = {2023}
}
\end{verbatim}
    \end{itemize}
\end{tcolorbox}

修士論文は\ttmastersthesis に分類します.
\verb|@masterthesis|ではなく\texttt{@master\textcolor{red}{s}thesis}です.
\verb|s|を忘れないでください.
また,\ttyear は修了「年度」ではなく修了「年」を西暦で書いてください.
例えば,日本の大学を2023年3月に修了した人は2022年度修了生ですが\verb|year = {2023}|です.

\subsection{\ttmisc}
\label{ssec:misc}
\begin{tcolorbox}[enhanced, title=\ttmisc, drop fuzzy shadow]
    \begin{itemize}
        \item 必須項目 \\
        無し
        \item オプション項目 \\
        \ttauthor, \tttitle, \tthowpublished, \ttarchivePrefix, \tteprint, \ttmonth, \ttyear, \ttnote, \ttkey, \ttdoi, \tturl
        \item 出力(通常,英語文献) \\
            \colorbox[gray]{0.8}{\ttauthorf}, \colorbox[gray]{0.8}{\ttauthors} and \colorbox[gray]{0.8}{\ttauthort}, \colorbox[gray]{0.8}{\tttitle}, \colorbox[gray]{0.8}{\tthowpublished} (\colorbox[gray]{0.8}{\ttyear}), \colorbox[gray]{0.8}{\ttnote}.
        \item 出力(通常,日本語文献) \\
            \colorbox[gray]{0.8}{\ttauthorf}, \colorbox[gray]{0.8}{\ttauthors}, \colorbox[gray]{0.8}{\ttauthort}, \colorbox[gray]{0.8}{\tttitle}, \colorbox[gray]{0.8}{\tthowpublished} (\colorbox[gray]{0.8}{\ttyear}), \colorbox[gray]{0.8}{\ttnote}.
        \item 出力(arXivの場合) \\
            \colorbox[gray]{0.8}{\ttauthorf}, \colorbox[gray]{0.8}{\ttauthors} and \colorbox[gray]{0.8}{\ttauthort}, \colorbox[gray]{0.8}{\tttitle}, arXiv: \colorbox[gray]{0.8}{\tteprint} (\colorbox[gray]{0.8}{\ttyear}), \colorbox[gray]{0.8}{\ttnote}.
        \item \verb|bib|ファイル作成例(通常) \vspace{-3mm}
\begin{verbatim}
@misc{湯村:卒論2006,
    author          = {湯村, 翼},
    yomi            = {Yumura, Tsubasa},
    title           = {レイリーテイラー不安定による赤道電離圏プラズマバブルの発生},
    howpublished    = {北海道大学理学部地球科学科卒業論文},
    year            = {2006},
    url             = {https://researchmap.jp/yumu/published_papers/1902404}
}
\end{verbatim}
        \item \verb|bib|ファイル作成例(arXivの場合) \vspace{-3mm}
\begin{verbatim}
@misc{Araki:arXiv2023,
    author          = {Araki, Ryo and Bos, Wouter J. T. and Goto, Susumu},
    title           = {Space-local {Navier--Stokes} turbulence}, 
    year            = {2023},
    eprint          = {2308.07255},
    archivePrefix   = {arXiv},
    primaryClass    = {physics.flu-dyn}
}
\end{verbatim}
    \end{itemize}
\end{tcolorbox}

その他該当種別が無いものは\ttmisc とします.
学部の卒業論文は\ttmisc でいいと思います.
ただし,\ttmastersthesis や\ttphdthesis と異なり,\ttschool のフィールドを使用できないので\tthowpublished で代用します.
したがって,\ttmastersthesis や\ttphdthesis では\ttschool に所属名だけ(例:\verb|school = {東京理科大学大学院理工学研究科機械工学専攻}|)書けばよかったものが\ttmisc で卒論を出力する際には\verb|howpublished = {北海道大学理学部地球科学科卒業論文}|のように「\verb|卒業論文|」の文字まで書く必要があります.
該当するエントリーがよくわからなかったらとりあえず\ttmisc に入れておくという人は多いと思います.
また,arXiv\footnote{arXiv(「アーカイブ」と読みます), \textless\url{https://arxiv.org/}\textgreater}と呼ばれるプレプリントサーバーから引用した文献は\ttmisc に分類します.
arXiv上のExport BibTeX Citationと書いてあるところから文献情報を見ると\ttmisc に分類されていることがわかると思います.
この文献情報では上記のように\verb|eprint = {2308.07255}|, \verb|archivePrefix = {arXiv}|などと書かれていることが多いです.
\jsmefile では\ttarchivePrefix フィールドがあるとこの文献がarXiv上の文献だと認識してくれて\tteprint の情報と合わせて
\begin{quote}
    Araki, R., Bos, W. J. T. and Goto, S., Space-local Navier--Stokes turbulence, \href{https://doi.org/10.48550/arXiv.2308.07255}{arXiv: 2308.07255} (2023).
\end{quote}
のように自動で書いてくれます.
\tteprint の情報からURLを自動生成するので,\href{https://doi.org/10.48550/arXiv.2308.07255}{arXiv: 2308.07255}と書かれている(青字の)箇所をクリックしたらarXivの該当ページにジャンプできます.

また,\ttmisc では\ttauthor フィールドが任意項目となっています.
第~\ref{ssec:manual}節\ttmanual でも説明したように,著者無し文献は不具合の元なので著者に相当しそうなものを\ttauthor に入れておいてください.

\subsection{\ttonline}
\label{ssec:online}
\begin{tcolorbox}[enhanced, title=\ttonline, drop fuzzy shadow]
    \begin{itemize}
        \item 使用可能項目 \\
        \ttauthor, \tttitle, \tthowpublished, \ttmonth, \ttyear, \tturl, \ttdoi, \ttaccess, \ttnote
        \item 出力(英語文献) \\
            \colorbox[gray]{0.8}{\ttauthorf}, \colorbox[gray]{0.8}{\ttauthors} and \colorbox[gray]{0.8}{\ttauthort}, \colorbox[gray]{0.8}{\tttitle}, available from \textless\colorbox[gray]{0.8}{\tturl}\textgreater, (accessed on \colorbox[gray]{0.8}{access}).
        \item 出力(日本語文献) \\
            \colorbox[gray]{0.8}{\ttauthorf}, \colorbox[gray]{0.8}{\ttauthors}, \colorbox[gray]{0.8}{\ttauthort}, \colorbox[gray]{0.8}{\tttitle}, \textless\colorbox[gray]{0.8}{\tturl}\textgreater, (参照日 \colorbox[gray]{0.8}{access}).
        \item \verb|bib|ファイル作成例 \vspace{-3mm}
\begin{verbatim}
@online{Kawamura_Ret64,
    author  = "{Kawamura Laboratory}",
    title   = {{DNS} Database of Wall Turbulence and Heat Transfer: 
                Text database of {Poiseuille} flow for $\mathit{Re}_\tau = 64$},
    url     = {https://www.rs.tus.ac.jp/~t2lab/db/index.html},
    access  = {10 October, 2023}
}
\end{verbatim}
    \end{itemize}
\end{tcolorbox}

\ttonline は\jsmefile 独自のエントリーなので他の\BibTeX{}スタイルファイルを用いるときには注意してください(\jsmefile の元となった\verb|jecon.bst|では使用できます).
本来,webページ等の引用はあまり推奨されるものではありませんが,データベースを研究室のwebページ等で公開していることがある\footnote{乱流の分野におけるデータベースとしては,東京理科大学河村研究室(現在は塚原研究室が管理) \textless\url{https://www.rs.tus.ac.jp/~t2lab/db/index.html}\textgreater や東京大学笠木研究室(現在は複数の大学によって管理) \textless\url{https://thtlab.jp}\textgreater などが挙げられます.}ので使う機会がゼロとは言えないでしょう.
英語のwebページの場合は
\renewcommand\UrlFont{\rmfamily}
\begin{quote}
    Kawamura Laboratory, DNS database of wall turbulence and heat transfer: Text database of Poiseuille flow for $\mathit{Re}_\tau = 64$, available from \textless\url{https://www.rs.tus.ac.jp/~t2lab/db/index.html}\textgreater, (accessed on 10 October, 2023).
\end{quote}
のように,末尾にwebページのURLと参照日を入れることが日本機械学会の原稿テンプレートに記載されています.
英語での参照日の書き方は\verb|access  = {10 October, 2023}|のように日,月(アルファベット),年の順です.
日本語のwebページの場合は
\begin{quote}
    立川裕二, 博士論文執筆の際にお願いしたいこと, \textless\url{https://member.ipmu.jp/yuji.tachikawa/misc/dron.html}\textgreater, (参照日 2023年10月10日).
\end{quote}
\renewcommand\UrlFont{\ttfamily}
のように出力します.

\subsection{\ttphdthesis}
\label{ssec:phdthesis}
\begin{tcolorbox}[enhanced, title=\ttphdthesis, drop fuzzy shadow]
    \begin{itemize}
        \item 必須項目 \\
        \ttauthor, \tttitle, \ttschool, \ttyear
        \item オプション項目 \\
        \ttaddress, \ttmonth, \ttnote, \ttkey, \ttdoi, \tturl
        \item 出力(英語文献) \\
            \colorbox[gray]{0.8}{\ttauthorf}, \colorbox[gray]{0.8}{\ttauthors} and \colorbox[gray]{0.8}{\ttauthort}, \colorbox[gray]{0.8}{\tttitle}, Ph.D. dissertation, \colorbox[gray]{0.8}{\ttschool} (\colorbox[gray]{0.8}{\ttyear}), \colorbox[gray]{0.8}{\ttnote}.
        \item 出力(日本語文献) \\
            \colorbox[gray]{0.8}{\ttauthorf}, \colorbox[gray]{0.8}{\ttauthors}, \colorbox[gray]{0.8}{\ttauthort}, \colorbox[gray]{0.8}{\tttitle}, \colorbox[gray]{0.8}{\ttschool}博士論文 (\colorbox[gray]{0.8}{\ttyear}), \colorbox[gray]{0.8}{\ttnote}.
        \item \verb|bib|ファイル作成例 \vspace{-3mm}
\begin{verbatim}
@phdthesis{塚原:博論2007,
    author  = {塚原, 隆裕},
    yomi    = {Tsukahara, Takahiro},
    title   = {大規模直接数値シミュレーションによる低レイノルズ数平行平板間乱流の研究},
    school  = {東京理科大学大学院理工学研究科機械工学専攻},
    year    = {2007},
    url     = {https://iss.ndl.go.jp/books/R100000002-I000009177724-00}
}
\end{verbatim}
    \end{itemize}
\end{tcolorbox}

\ttphdthesis は博士論文が該当します.
エントリー名に\verb|thesis|と入っていますが,英語文献の場合はdissertationと出力されます.
基本的な使い方は\hyperref[ssec:mastersthesis]{\ttmastersthesis}と同じです.

\subsection{\ttproceedings}
\label{ssec:proceedings}
\begin{tcolorbox}[enhanced, title=\ttproceedings, drop fuzzy shadow]
    \begin{itemize}
        \item 必須項目 \\
        \tttitle, \ttyear
        \item オプション項目 \\
        \tteditor, \ttorganization, \ttpublisher, \ttaddress, \ttmonth, \ttnote, \ttkey, \ttdoi, \tturl
        \item 出力 \\
            \colorbox[gray]{0.8}{\tttitle} (\colorbox[gray]{0.8}{\ttyear}), \colorbox[gray]{0.8}{\ttnote}.
        \item \verb|bib|ファイル作成例 \vspace{-3mm}
\begin{verbatim}
@proceedings{THMT2023,
    editor  = {THMT},
    title   = "{Proceedings of 10th International Symposium 
                on Turbulence, Heat and Mass Transfer}",
    yomi    = {Proceedings},
    year    = {2023}
}
\end{verbatim}
    \end{itemize}
\end{tcolorbox}

学会の講演論文集全体を引用する際には\ttproceedings を使用します.
\hyperref[ssec:conference]{\ttconference}や\hyperref[ssec:inproceedings]{\ttinproceedings}は講演論文集の中の一講演を引用しているのに対して,\ttproceedings は講演論文集全体を引用しています.
\hyperref[ssec:manual]{\ttmanual}や\hyperref[ssec:misc]{\ttmisc}と同様,\ttauthor フィールドが必須項目ではありません.
したがって,文献一覧のソートの際は\tttitle の読みを和文・英文問わず\ttyomi フィールドで指定してあげないと変な場所に飛ばされてしまいます.
また,もし
\begin{quote}
\begin{verbatim}
@proceedings{THMT2023,
    title   = "{Proceedings of 10th International Symposium 
                on Turbulence, Heat and Mass Transfer}",
    yomi    = {Proceedings},
    year    = {2023}
}
@proceedings{流力年会2023,
    title   = {日本流体力学会年会2023講演論文集},
    yomi    = {Nihonryuutairikigakkainenkai2023},
    year    = {2023}
}
\end{verbatim}
\end{quote}
のように\verb|bib|ファイルに書いていた場合,文中で引用した際に以下のようにおかしな表示となってしまいます.
\begin{table}[h]
    \centering
    \begin{tabular}{ll}
        コマンド    &出力結果 \\
        \verb|\citet{THMT2023}| &THM (2023) \\
        \verb|\citet{流力年会2023}| &流 (2023) \\
    \end{tabular}
\end{table}
そのため,\tteditor フィールドに「引用時に著者代わりに出力したい文字列」を書いておきましょう.
文献一覧には\tteditor は出力されませんが,本文中での引用では\citet{THMT2023}や\citet{流力年会2023}のように\verb|editor (year)|で出力してくれます.


\subsection{\tttechreport}
\label{ssec:techreport}
\begin{tcolorbox}[enhanced, title=\tttechreport, drop fuzzy shadow]
    \begin{itemize}
        \item 必須項目 \\
        \ttauthor, \tttitle, \ttinstitution, \ttyear
        \item オプション項目 \\
        \tttype, \ttnumber, \ttaddress, \ttmonth, \ttnote, \ttkey, \ttdoi, \tturl
        \item 出力(英語文献) \\
            \colorbox[gray]{0.8}{\ttauthorf}, \colorbox[gray]{0.8}{\ttauthors} and \colorbox[gray]{0.8}{\ttauthort}, \colorbox[gray]{0.8}{\tttitle}, \colorbox[gray]{0.8}{\ttinstitution} (\colorbox[gray]{0.8}{\ttyear}), \colorbox[gray]{0.8}{\ttnote}.
        \item 出力(日本語文献) \\
            \colorbox[gray]{0.8}{\ttauthorf}, \colorbox[gray]{0.8}{\ttauthors}, \colorbox[gray]{0.8}{\ttauthort}, \colorbox[gray]{0.8}{\tttitle}, \colorbox[gray]{0.8}{\ttinstitution} (\colorbox[gray]{0.8}{\ttyear}), \colorbox[gray]{0.8}{\ttnote}.
        \item \verb|bib|ファイル作成例 \vspace{-3mm}
\begin{verbatim}
@techreport{Neuhart:NASAreport2004,
    author      = {Neuhart, Dan H. and McGinley, Catherine B.},
    title       = {Free-Stream Turbulence Intensity in the {Langley} 
                    14- by 22-Foot Subsonic Tunnel},
    institution = {NASA Technical Publication},
    year        = {2004},
    url         = {https://ntrs.nasa.gov/citations/20040120956},
    note        = {TP-2004-213247}
}
\end{verbatim}
    \end{itemize}
\end{tcolorbox}

研究機関等から発行された技術報告書は\tttechreport に分類します.
技術報告書を発行している研究機関はさまざまありますが,例えばNASA\footnote{アメリカ航空宇宙局(NASA) Technical Reports Server, \textless\url{https://ntrs.nasa.gov/}\textgreater}や国立天文台\footnote{大学共同利用機関法人 自然科学研究機構 国立天文台,国立天文台欧文報告, \textless\url{https://www.nao.ac.jp/about-naoj/reports/publications-naoj.html}\textgreater},鉄道総研\footnote{公益財団法人 鉄道総合技術研究所(鉄道総研),鉄道総研報告, \textless\url{https://www.rtri.or.jp/publish/rtrirep/}\textgreater}などが挙げられます.
また,企業によっては技術報告書を公開しているところもあります.
\tttechreport には\tttype という任意フィールドがありますが,諸事情でここに何か書いても何も表示しないようにしているので使わないでください.
本当は\verb|type = {Technical Report}|などと書くとそのように表示されます.

\subsection{\ttunpublished}
\label{ssec:unpublished}
\begin{tcolorbox}[enhanced, title=\ttunpublished, drop fuzzy shadow]
    \begin{itemize}
        \item 必須項目 \\
        \ttauthor, \tttitle, \ttnote
        \item オプション項目 \\
        \ttmonth, \ttyear, \ttkey, \ttdoi, \tturl
        \item 出力(英語文献) \\
            \colorbox[gray]{0.8}{\ttauthorf}, \colorbox[gray]{0.8}{\ttauthors} and \colorbox[gray]{0.8}{\ttauthort}, \colorbox[gray]{0.8}{\tttitle}, (\colorbox[gray]{0.8}{\ttyear}), \colorbox[gray]{0.8}{\ttnote}.
        \item 出力(日本語文献) \\
            \colorbox[gray]{0.8}{\ttauthorf}, \colorbox[gray]{0.8}{\ttauthors}, \colorbox[gray]{0.8}{\ttauthort}, \colorbox[gray]{0.8}{\tttitle}, (\colorbox[gray]{0.8}{\ttyear}), \colorbox[gray]{0.8}{\ttnote}.
        \item \verb|bib|ファイル作成例 \vspace{-3mm}
\begin{verbatim}
@unpublished{Dunkel:MITOCW2015,
    author  = {Dunkel, J\:{o}rn},
    title   = {Nonlinear Dynamics {II}: Continuum Systems, 
        Linear Stability Analysis and Pattern Formation, {MIT Open Course Ware}},
    year    = {2015},
    url     = {https://ocw.mit.edu/courses/
        18-354j-nonlinear-dynamics-ii-continuum-systems-spring-2015/
        resources/mit18_354js15_ch7/}
}
\end{verbatim}
    \end{itemize}
\end{tcolorbox}

\ttunpublished は正式には出版されていない本などが該当します.
ただ,\ttunpublished を使う機会はほぼ無いと思われますし,そもそも具体的に何が該当するのかもパッとイメージできるものではありません.
また,大抵の人は分類がよくわからなかったら\hyperref[ssec:misc]{\ttmisc}に入れてしまう気がします.
しかし,\jsmefile を作成するうえでパターンを網羅しておきたかったので(無理矢理)分類したものとしては大学のオープンコースウェア(OCW)の講義資料等です.
講義資料であれば正式に出版された書籍ではありませんがweb上にあるのをよく見かけると思います.
上記の例はMITのOCW\footnote{MIT Open Course Ware, \textless\url{https://ocw.mit.edu/}\textgreater}における講義資料です.


\clearpage
\section{\jsmefile の使い方}
それでは実際に\jsmefile を使ってみましょう.
必要なファイルは以下の通りです.
括弧の中は\JSMErepos の場合の該当ファイルを指します.
\begin{itemize}
    \item 本文の\verb|tex|ファイル(\verb|JSME-template1.tex|)
    \item \BibTeX{}の\verb|bst|スタイルファイル(\jsmefile)
    \item 読み込む文献リストの\verb|bib|ファイル(\verb|mybib_en.bib|, \verb|mybib_jp.bib|)
\end{itemize}
次に本文の\verb|tex|ファイルの設定ですが,おおまかに下記のようになります.
\begin{tcolorbox}[enhanced, title=\textgt{本文の\texttt{tex}ファイルに必要な設定}, drop fuzzy shadow]
\begin{verbatim}
\documentclass[a4paper,fleqn,uplatex,dvipdfmx]{jsarticle}
%%% hyperref
%%% ハイパーリンクの色を全て黒にしたいときは allcolors=blue の箇所を hidelinks にする
\usepackage[bookmarks=true,setpagesize=false,colorlinks,allcolors=blue]{hyperref}
\usepackage{pxjahyper}
%%% natbib
\usepackage{natbib}
%%% 参考文献リストの見出しを「参考文献」から「文献」に変更.
\renewcommand{\refname}{文献}
\begin{document}

本文

%%% ハイパーリンクのズレを調整
\phantomsection
%%% \nocite{*}が有効のとき,引用していない文献も含めて全て表示
\nocite{*}
%%% 目次に「文献」を追加
\addcontentsline{toc}{section}{\refname}
%%% 使用する bst ファイル
\bibliographystyle{jsme}
%%% 読み込む bib ファイル
\bibliography{mybib_en.bib,mybib_jp.bib}
\end{document}
\end{verbatim}
\end{tcolorbox}
まず\verb|\documentclass|と\verb|\begin{document}|の間のプリアンブルで必要なパッケージを入れます.
ここでは\verb|hyperref|を入れることでハイパーリンクを有効にし,\verb|pxjahyper|を入れることでしおりの日本語表示を有効にしています.
ここで使用するオプション等はユーザーに任せます.
そして,本文中で\verb|\citet|や\verb|\citep|等のコマンドを有効化するために\verb|natbib|を入れます.
次に,\verb|\begin{document}|以降の本文の末尾に\verb|\bibliographystyle{jsme}|と入れることで\jsmefile がここで使用する\BibTeX{}スタイルファイルとして読み込まれます.
他の\BibTeX{}スタイルファイルを使用する際はここを変更してください.
次に,読み込む文献リストの\verb|bib|ファイルを\verb|\bibliography{}|で指定してください.
ここでは\verb|mybib_en.bib|と\verb|mybib_jp.bib|が読み込まれています.
実際に皆さんの環境で使用する際には本文の\verb|tex|ファイルと同じディレクトリに\verb|bib|ファイルを入れるのが普通ですが,小分けにした\verb|bib|ファイルが多くあるのであれば,\texttt{\textcolor{red}{bibliography/}}\verb|mybib_en.bib|みたいに階層を変えてもいいです.
ただし,その場合には\verb|\bibliography{|\texttt{\textcolor{red}{bibliography/}}\verb|mybib_en.bib,|\texttt{\textcolor{red}{bibliography/}}\verb|mybib_jp.bib}|のように\verb|\bibliography|のパスも変更してください.
\verb|\addcontentsline{toc}{section}{\refname}|は目次に「文献」を追加したい場合にはこのように入れます.
ここでは「文献」に相当するものが節なので\verb|section|と書いていますが,章に相当する場合は\verb|chapter|に変えてあげてください.
これは使用するドキュメントクラス\footnote{\texttt{tex}ファイル冒頭の\texttt{\textbackslash documentclass}の記述を確認.}が\verb|book|または\verb|report|の名前を含む場合が該当します.
同時に,これらのドキュメントクラスでは参考文献の節の名前が\verb|\refname|ではなく\verb|\bibname|で定義されるので
\begin{tcolorbox}[enhanced, drop fuzzy shadow]
\begin{verbatim}
%%% 参考文献リストの見出しを「参考文献」から「文献」に変更.
\renewcommand{\bibname}{文献}
%%% 目次に「文献」を追加
\addcontentsline{toc}{chapter}{\bibname}
\end{verbatim}
\end{tcolorbox}
\noindent
と変更する必要があります.
また,目次で生成される「文献」へのハイパーリンクにズレが発生することがあるので,その際は\verb|\phantomsection|で空の節(幻の節)を入れてあげることで調整します.
\verb|\nocite{*}|は本文中で引用していない文献も含め,読み込んだ\verb|bib|ファイル内の全ての文献をリストアップします.
本文中で引用したもののみ一覧に載せたい場合はこの行をコメントアウトしてください.
その他細かいことは\verb|JSME-template1.tex|の中身も読んでみてください.

次に,コンパイルについてです.
\verb|JSME-template1.tex|から\verb|JSME-template1.pdf|の生成は\upLaTeX{}$+$\upBibTeX{}を使用しているので,ユーザーの使用環境に応じて変更してください.
ここでは\pLaTeX{}$+$\pBibTeX{}の場合と\upLaTeX{}$+$\upBibTeX{}の場合の説明をします.
\pLaTeX{}$+$\pBibTeX{}を使用する場合は以下のようなコマンドになります.
\begin{tcolorbox}[enhanced, title=\pLaTeX{}$+$\pBibTeX{}, drop fuzzy shadow]
\begin{verbatim}
$ platex JSME-template1.tex
$ pbibtex JSME-template1
$ platex JSME-template1.tex (複数回)
$ dvipdfmx JSME-template1
\end{verbatim}
\end{tcolorbox}
\noindent
\upLaTeX{}$+$\upBibTeX{}を使用する場合は以下のようなコマンドになります.
\begin{tcolorbox}[enhanced, title=\upLaTeX{}$+$\upBibTeX{}, drop fuzzy shadow]
\begin{verbatim}
$ uplatex JSME-template1.tex
$ upbibtex JSME-template1
$ uplatex JSME-template1.tex (複数回)
$ dvipdfmx JSME-template1
\end{verbatim}
\end{tcolorbox}
\noindent
これでうまくいけば\verb|pdf|が生成されているはずです.
\verb|latexmk|を設定していればコマンド一つで簡単に処理できます.

\clearpage
\section*{謝辞}
\addcontentsline{toc}{section}{謝辞}
\jsmefile は武田史郎氏作の経済学用\BibTeX{}スタイルファイル\verb|jecon.bst|\footnote{\texttt{jecon-bst}, \textless\url{https://github.com/ShiroTakeda/jecon-bst}\textgreater}を改変して作成したものです.
\verb|jecon.bst|内に残されていた武田氏の懇切丁寧なコメントおよび説明用の\verb|jecon-example.pdf|は\jsmefile を作成するにあたり大変参考になり,これらが無ければ実現しませんでした.
また,\citet{奥村:技評2020}と\citet{吉永:翔泳社2018}は\JSMErepos を作成するにあたり参考にした,この文書における「本当の」参考文献です.
ここに,深く感謝の意を表します.

\clearpage
%%% ハイパーリンクのズレを調整
\phantomsection
%%% 参考文献内の URL 表示をタイプライター調にしない
\renewcommand\UrlFont{\rmfamily}
%%% \nocite{*}が有効のとき,引用していない文献も含めて全て表示
\nocite{*}
%%% 目次に「文献」を追加
\addcontentsline{toc}{section}{\refname}
%%% 使用する bst ファイル
\bibliographystyle{jsme}
%%% 読み込む bib ファイル
\bibliography{
mybib_en.bib,
mybib_jp.bib
}


\end{document}