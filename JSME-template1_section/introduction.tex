\section{はじめに}
\label{sec:introduction}
\jsmefile は日本機械学会\footnote{一般社団法人 日本機械学会(The Japan Society of Mechanical Engineers, JSME),\textless\url{https://www.jsme.or.jp/}\textgreater}の原稿テンプレート\footnote{日本機械学会 原稿テンプレート,\textless\url{https://www.jsme.or.jp/publish/transact/for-authors.html}\textgreater}に基づいた参考文献の出力を実現するために作成した,非公式 \BibTeX スタイルファイルテンプレートです.
必要なファイル一式はGitHubの\JSMErepos\footnote{\JSMErepos, \textless\url{https://github.com/Yuki-MATSUKAWA/JSME-bst}\textgreater}から入手可能なので,用途に応じて自由に改変してください.
今読んでいるこのファイル(\LaTeX ソース:\verb|JSME-template1.tex|,出力結果:\verb|JSME-template1.pdf|)では \BibTeX で用意されている全てのエントリーの出力結果を表示しています.
参考文献のリストはこの\verb|pdf|の末尾で,出力している文献の\verb|bib|ファイルは英語文献\verb|mybib_en.bib|と日本語文献\verb|mybib_jp.bib|の二つです.
\JSMErepos の作成者である松川が流体力学,特に乱流遷移の研究をしているため,引用している文献は乱流遷移の周辺のものが多くなっています(全てではありません).
ただ,材料力学など他分野の方でも基本的な使い方は同じです.
また,第~\ref{sec:cite}節でも述べますが,著者数が1名,2名,3名以上のそれぞれで引用時の出力結果が異なります.
そのため\verb|mybib_en.bib|,\verb|mybib_jp.bib|では可能な限り著者数が1名,2名,3名以上の計3パターンを用意しています.
ただ,基本的に実在の文献を集めて載せているので学位論文(\ttphdthesis, \ttmastersthesis)のように原則著者が一人のものや \ttmanual,\ttunpublished など一部のエントリーでは全てのパターンを網羅できていない場合もあるのでご了承ください(ただ,参考にするうえで困ることのないくらいにはパターンを網羅しているつもりです).

この文書では \BibTeX 初心者でも使いやすいよう,\BibTeX そのものの使い方や\verb|bib|ファイルの作成方法といった内容も可能な限り説明します.
それでもわからなければさまざまな書籍やwebサイトがあるので参考にしてみてください.
また,その都度説明しますが,日本機械学会の原稿テンプレートに沿ったリストを作成するために\verb|bib|ファイルの作成方法が一部通常の \BibTeX と異なる場合があるのでご了承ください.

\JSMErepos 作成にあたり可能な限り日本機械学会の原稿テンプレートを忠実に再現できるよう努力していますが,まだ十分に再現できていない箇所もあります.
もし何か問題や不明点があればGitHubにコメントしていただくかメールをいただけると幸いです.
可能な限り対応しますが,\JSMErepos を使用したことにより発生した問題に対しては一切の責任を持ちませんのでご了承ください.
また,\JSMErepos リポジトリ内のファイルは全て日本機械学会の書式の実現のために作成した非公式ファイルなので使用方法等を日本機械学会に問い合わせることもご遠慮ください.
