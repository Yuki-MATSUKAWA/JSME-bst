\clearpage
\section{\jsmefile の使い方}
\label{sec:howtouse}
それでは実際に \jsmefile を使ってみましょう.
必要なファイルは以下の通りです.
括弧の中は \JSMErepos の場合の該当ファイルを指します.
\begin{itemize}
    \item 本文の\verb|tex|ファイル(\verb|JSME-template2.tex|)
    \item \BibTeX の\verb|bst|スタイルファイル(\jsmefile)
    \item 読み込む文献リストの\verb|bib|ファイル(\verb|mybib_en.bib|, \verb|mybib_jp.bib|)
\end{itemize}
次に本文の\verb|tex|ファイルの設定ですが,おおまかに下記のようになります.
\begin{tcolorbox}[enhanced, title=\textgt{本文の\texttt{tex}ファイルに必要な設定}, drop fuzzy shadow]
\begin{verbatim}
\documentclass[a4paper,fleqn,uplatex,dvipdfmx]{jsarticle}
%%% URL
\usepackage{url}
%%% natbib
\usepackage{natbib}
%%% 参考文献リストの見出しを「参考文献」から「文献」に変更.
\renewcommand{\refname}{文献}
\begin{document}

本文

%%% 参考文献内の URL 表示をタイプライター調にしない
\renewcommand\UrlFont{\rmfamily}
%%% \nocite{*}が有効のとき,引用していない文献も含めて全て表示
\nocite{*}
%%% 目次に「文献」を追加
\addcontentsline{toc}{section}{\refname}
%%% 使用する bst ファイル
\bibliographystyle{jsme}
%%% 読み込む bib ファイル
\bibliography{mybib_en.bib,mybib_jp.bib}
\end{document}
\end{verbatim}
\end{tcolorbox}

まず\verb|\documentclass|と\verb|\begin{document}|の間のプリアンブルで必要なパッケージを入れます.
ここでは\verb|url|パッケージを入れていますが,これは \ttonline を使用する際にURLの表示がおかしくならないようにするためです.
\jsmefile ではハイパーリンクを使用しないため\verb|hyperref|を入れていませんが,\verb|url|は必要になります.
そして,本文中で\verb|\citet|や\verb|\citep|等のコマンドを有効化するために\verb|natbib|を入れます.
次に,\verb|\begin{document}|以降の本文の末尾に\verb|\bibliographystyle{jsme}|と入れることで \jsmefile がここで使用する \BibTeX スタイルファイルとして読み込まれます.
他の \BibTeX スタイルファイルを使用する際はここを変更してください.
次に,読み込む文献リストの\verb|bib|ファイルを\verb|\bibliography{}|で指定してください.
ここでは\verb|mybib_en.bib|と\verb|mybib_jp.bib|が読み込まれています.
実際に皆さんの環境で使用する際には本文の\verb|tex|ファイルと同じディレクトリに\verb|bib|ファイルを入れるのが普通ですが,小分けにした\verb|bib|ファイルが多くあるのであれば,\texttt{\textcolor{red}{bibliography/}}\verb|mybib_en.bib|みたいに階層を変えてもいいです.
ただし,その場合には\verb|\bibliography{|\texttt{\textcolor{red}{bibliography/}}\verb|mybib_en.bib,|\texttt{\textcolor{red}{bibliography/}}\verb|mybib_jp.bib}|のように\verb|\bibliography|のパスも変更してください.
\verb|\addcontentsline{toc}{section}{\refname}|は目次に「文献」を追加したい場合にはこのように入れます.
ここでは「文献」に相当するものが節なので\verb|section|と書いていますが,章に相当する場合は\verb|chapter|に変えてあげてください.
これは使用するドキュメントクラス\footnote{\texttt{tex}ファイル冒頭の\texttt{\textbackslash documentclass}の記述を確認.}が\verb|book|または\verb|report|の名前を含む場合が該当します.
同時に,これらのドキュメントクラスでは参考文献の節の名前が\verb|\refname|ではなく\verb|\bibname|で定義されるので
\begin{tcolorbox}[enhanced, drop fuzzy shadow]
\begin{verbatim}
%%% 参考文献リストの見出しを「参考文献」から「文献」に変更.
\renewcommand{\bibname}{文献}
%%% 目次に「文献」を追加
\addcontentsline{toc}{chapter}{\bibname}
\end{verbatim}
\end{tcolorbox}
\noindent
と変更する必要があります.
\verb|\nocite{*}|は本文中で引用していない文献も含め,読み込んだ\verb|bib|ファイル内の全ての文献をリストアップします.
本文中で引用したもののみ一覧に載せたい場合はこの行をコメントアウトしてください.
その他細かいことは\verb|JSME-template2.tex|の中身も読んでみてください.

次に,コンパイルについてです.
\verb|JSME-template2.tex|から\verb|JSME-template2.pdf|の生成は \upLaTeX$+$\upBibTeX を使用しているので,ユーザーの使用環境に応じて変更してください.
ここでは \pLaTeX$+$\pBibTeX の場合と \upLaTeX$+$\upBibTeX の場合の説明をします.
\pLaTeX$+$\pBibTeX を使用する場合は以下のようなコマンドになります.
\begin{tcolorbox}[enhanced, title=\pLaTeX$+$\pBibTeX, drop fuzzy shadow]
\begin{verbatim}
$ platex JSME-template2.tex
$ pbibtex JSME-template2
$ platex JSME-template2.tex (複数回)
$ dvipdfmx JSME-template2
\end{verbatim}
\end{tcolorbox}
\noindent
\upLaTeX$+$\upBibTeX を使用する場合は以下のようなコマンドになります.
\begin{tcolorbox}[enhanced, title=\upLaTeX$+$\upBibTeX, drop fuzzy shadow]
\begin{verbatim}
$ uplatex JSME-template2.tex
$ upbibtex JSME-template2
$ uplatex JSME-template2.tex (複数回)
$ dvipdfmx JSME-template2
\end{verbatim}
\end{tcolorbox}
\noindent
これでうまくいけば\verb|pdf|が生成されているはずです.
\verb|latexmk|を設定していればコマンド一つで簡単に処理できます.