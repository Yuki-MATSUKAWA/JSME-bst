\section{日本機械学会の原稿執筆要領における注意点}
\label{sec:caution}
この節では日本機械学会の原稿テンプレートに記載されている執筆要領の中でも,特に文献の記載に関する注意点をまとめておきます.
日本機械学会の規定に合わせて論文執筆する際は是非参考にしてください\footnote{\JSMErepos 自体はもともと,日本機械学会の規定に沿って卒業論文執筆を行うことが決められている東京理科大学創域理工学部機械航空宇宙工学科の卒研生向けに作成したものです.もちろんGitHub上で公開しているので世界中の誰が使っても構いません.}.

\subsection{使用する書体・フォントについて}
日本機械学会では本文に使用する書体を和文は明朝体,欧文はSerif体,サイズは$10\,\mathrm{pt}$と規定されています.
これに倣い,参考文献を記載する際も和文を明朝体,欧文をSerif体とします.
Serif体のフォントとしては{\usefont{T1}{ptm}{m}{n} Times}や{\usefont{T1}{pplx}{m}{n} Palatino},{\usefont{OT1}{cmr}{m}{n} Computer Modern}などが挙げられます.
\LaTeX{}標準のフォントは{\usefont{OT1}{cmr}{m}{n} Computer Modern}であって,{\usefont{T1}{ptm}{m}{n} Times}ではないので注意.
フォントを変える際はプリアンブル(\verb|\begin{document}|以前に書く設定)で指定します.
この文書の場合はプリアンブルの\verb|\usepackage{newtxtext,newtxmath}|でフォントを{\usefont{T1}{ptm}{m}{n} Times}に変更しています.
もし文書全体を{\usefont{OT1}{cmr}{m}{n} Computer Modern}に変更したいときは\verb|\usepackage{newtxtext,newtxmath}|をコメントアウトし,{\usefont{T1}{pplx}{m}{n} Palatino}に変更したいときは\verb|\usepackage{mathpazo}|のコメントアウトを外します.
ここでは参考文献の書き方に焦点を当てて説明するのでその他の箇所の書式は日本機械学会の原稿テンプレートをよく読んで作成してください.

\subsection{文献の並べ方・引用の仕方}
この節では文献リストのフィールド(\BibTeX における \ttauthor や \tttitle)の並べ方について説明します.
\jsmefile のユーザーは気にせず使用することができますが,改めて整理しておくので \jsmefile を使わないときにも役に立つと思います.
原則,日本機械学会の規定に合わせていますが,原稿テンプレートからは判断が難しい事項は一部独自の解釈を加えているものがあります.

文献を並べる順序の規則は以下の通りです.
\begin{tcolorbox}[enhanced, title=\textgt{文献ソート規則}, drop fuzzy shadow]
    \begin{enumerate}
        \item (Family, Givenの順で並べた際の)筆頭著者の氏名のアルファベット順.並べる際,日本人の氏名は漢字と仮名を用いた日本語表記で構わないが順序はアルファベット順とする.
        \item 筆頭著者が同一人物の場合,第二著者以降のアルファベット順で並べる.著者数が異なる場合は著者数が少ない方が先.これを最後の著者まで繰り返す.
        \item 著者が全員一致する文献があった場合は発行が早い順で並べる.
        \item 確認できる範囲で発行年月日が同じだった場合,タイトルのアルファベット順で並べ,西暦の後に小文字でアルファベットを順に振る(現在,\jsmefile の設定を調整中).
        \begin{quote}
            Matsukawa, Y. and Tsukahara, T., Laminarization in subcritical Taylor--Couette--Poiseuille flow with increasing pressure gradient, Proceedings of Nineteenth International Conference on Flow Dynamics (2022a). \\
            Matsukawa, Y. and Tsukahara, T., Subcritical transition of Taylor--Couette--Poiseuille flow at high radius ratio, Physics of Fluids, Vol.~34, No.~7 (2022b), 074109.
        \end{quote}
    \end{enumerate}
\end{tcolorbox}
\noindent
文献を並べる際の表記に関する注意事項は以下の通りです.
\renewcommand\UrlFont{\rmfamily}
\begin{tcolorbox}[enhanced, title=\textgt{文献リスト作成の注意事項(全体)}, drop fuzzy shadow]
    \begin{itemize}
        \item 並び順は原則として \\
            著者(\ttauthor)$\to$タイトル(\tttitle)$\to$誌名(\ttjournal)・書名(\ttbooktitle)$\to$出版社(\ttpublisher)・大学名(\ttschool)・機関名(\ttinstitution)$\to$巻(\ttvolume)$\to$号(\ttnumber)$\to$発行年(\ttyear)$\to$ページ(\ttpages)$\to$論文番号・講演番号(\ttnote) \\
            とする.
            ただし,存在しないフィールドがある場合は抜かす.
            また,webページ等の引用は極力避けるべきだが,引用する場合には必ず末尾にURLと参照日を明記する(第~\ref{ssec:online}節 \ttonline を参照).
            \begin{quote}
                Kawamura Laboratory, DNS database of wall turbulence and heat transfer: Text database of Poiseuille flow for $\mathit{Re}_\tau = 64$, available from \textless\url{https://www.rs.tus.ac.jp/~t2lab/db/index.html}\textgreater, (accessed on 10 October, 2023).
            \end{quote}
        \item 文献自体が日本語で書かれている場合は外国人が書いていても日本語文献とする.
        \item 文献自体が英語で書かれている場合は日本人が書いていても英語文献とする.
        \item フィールドとフィールドの間は日本語文献・英語文献問わず半角カンマと半角スペース(, )で繋ぐ.全角カンマ(,)ではないので注意.
        \item 他学会の文献テンプレートでは誌名を\textit{Italic}体にしたり巻数を\textbf{Boldface}体にしたりすることがあるが,日本機械学会では全て\textrm{Roman}体で統一する(WebページのURLを \LaTeX で出力するときに\texttt{Typewriter}体が用いられることが多いが,これも\textrm{Roman}体にする.).
        \item 以上の内容が満たされていて,文献にアクセスするうえで十分な情報が書かれていれば月(\ttmonth)や章(\ttchapter)等は不要.
    \end{itemize}
\end{tcolorbox}
\renewcommand\UrlFont{\ttfamily}
\noindent
\ttauthor や \tttitle などのフィールドについての詳細は第~\ref{ssec:field}節を,各エントリー(\ttarticle や \ttbook など)ごとの詳細は第~\ref{ssec:article}節以降を参照してください.
次に,文献リストを作成する際の著者名表記に関する注意事項です.

\begin{tcolorbox}[enhanced, title=\textgt{文献リスト作成の注意事項(著者名)}, drop fuzzy shadow]
    \begin{itemize}
        \item 著者の氏名は「姓」「名」の順で書き,間にスペース等は入れない(例:松川裕樹).
        \item 英語文献の著者の氏名はFamily nameのみ略記せず,Middle nameやGiven nameはイニシャルで記載する(例:Matsukawa Yuki $\to$ Matsukawa, Y.).
        \item 著者は全員記載する.
        \item 日本語文献の著者数が二人以上の場合は
            \begin{quote}
                松川裕樹, 塚原隆裕        
            \end{quote}
            のように半角カンマと半角スペースを間に入れて繋ぐ.
        \item 英語文献の著者数が二人の場合は
            \begin{quote}
                Matsukawa, Y. and Tsukahara, T.        
            \end{quote}
            のようにandで繋ぎ,著者数が三人以上の場合は
            \begin{quote}
                Araki, R., Bos, W. J. T. and Goto, S.            
            \end{quote}
            のように最後だけandで繋ぐ.最後のandの前にカンマは入れない.
    \end{itemize}
\end{tcolorbox}
\noindent
最後に,文献リスト作成時のその他の注意事項を説明します.
\begin{tcolorbox}[enhanced, title=\textgt{\large 文献リスト作成の注意事項(その他)}, drop fuzzy shadow]
    \begin{itemize}
        \item 英語文献のタイトルは最初の単語の頭文字のみ大文字(固有名詞等は除く)\footnote{文頭の単語の頭文字のみ大文字で表記する方法を sentence case と言う.}.
        \item 誌名・書名は省略せずに記載する.その際,アルファベットの大文字・小文字は誌名通りに表記する.冠詞や前置詞以外の各単語の頭文字を大文字にすることが多い\footnote{冠詞や前置詞以外の各単語の頭文字を大文字表記する方法を title case と言う.}.
        \item 巻・号は日本語・英語文献問わずVol.~xx,No.~xxとする.
        \item 発行年は西暦で表記し,括弧で括る.ただし,発行年の前のみ半角カンマは不要.
        \item ページ数は単ページの場合はp.~xxとし,複数ページに亘る場合はpp.~xx--yyとする.
        \item 「2014年以降発行の日本機械学会論文集は通しページを廃止したためDOIを記載するように」との注釈が日本機械学会の原稿テンプレートに書かれているが,\jsmefile では \ttdoi フィールドに値を入れていた場合,末尾にDOIを記載するように設定している.
    \end{itemize}
\end{tcolorbox}
