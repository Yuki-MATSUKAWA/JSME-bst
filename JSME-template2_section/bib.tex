\clearpage
\section{書誌情報ファイル(\texttt{bib}ファイル)の作り方}
\label{sec:bib}
ここでは書誌情報ファイル(\texttt{bib}ファイル)の作り方,使い方を説明します.
この\verb|pdf|の末尾の参考文献リストは同じディレクトリにある\verb|mybib_en.bib|と\verb|mybib_jp.bib|から作成しているので参考にしてください.
\verb|bib|ファイルに入力する書誌情報は次のような構造になっています.
\begin{tcolorbox}[enhanced, title=\textgt{\texttt{bib}ファイル内の書誌情報の構造}, drop fuzzy shadow]
\begin{verbatim}
@エントリー名{参照キー,
    フィールド1 = {},
    フィールド2 = {},
    フィールド3 = {}
}
\end{verbatim}
\end{tcolorbox}
\noindent
だいたいの雑誌論文のwebサイトでは \BibTeX 形式で書誌情報を出力できる機能があるのでそこから\verb|bib|ファイルをダウンロードします.
もちろん,ダウンロードした\verb|bib|ファイルを自分で書き換えることもできますし,自分で一から\verb|bib|ファイルを作成することも可能です.

第~\ref{sec:cite}節で述べたように,文献を本文中で引用する際は\verb|\citet{Matsukawa:PoF2022}|のように書きます.
このときの\verb|Matsukawa:PoF2022|が参照キーです.
参照キーの書き方に特に規則は無く,半角カンマ以外の半角記号も使用可能です.
ただ,自分の中でマイルールを設けておくと引用する際に楽です.
私の場合は原則として\texttt{\colorbox[gray]{0.8}{著者}:\colorbox[gray]{0.8}{誌名}\colorbox[gray]{0.8}{年}}としています.
また,エントリー名とフィールド名は大文字と小文字を区別しませんが,参照キーは区別するので気をつけてください.

エントリーは雑誌論文や学位論文といった,その文献の該当する種別を表します.
\jsmefile でサポートされているエントリーは \ttarticle, \ttbook, \ttbooklet, \ttcomment, \ttconference, \ttinbook, \ttincollection, \ttinproceedings, \ttmanual, \ttmastersthesis, \ttmisc, \ttonline, \ttphdthesis, \ttproceedings, \tttechreport, \ttunpublished の16種類です.
それぞれのエントリーで必須となるフィールドが異なり,文献一覧への出力の方法も異なるので面倒くさがらずに分類しましょう.
ただ,全ての文献を正確に分類することは難しく,判断が人により異なることもあります.
\JSMErepos における分類の仕方も見る人によっては違和感を覚えるものがあるかもしれません.
それぞれのエントリーに関する詳細を\pageref{ssec:article}ページ以降で説明しているので参考にしてください.
ただし,ここで説明する内容は一般的な \BibTeX におけるエントリーの説明と若干異なる箇所があるのでご了承ください.

フィールドはその文献の著者情報や誌名情報を入力するデータ項目です.
\jsmefile では \ttaccess, \ttaddress, \ttarchivePrefix, \ttauthor, \ttbooktitle, \ttchapter, \ttdoi, \ttedition, \tteditor, \tteprint, \tthowpublished, \ttinstitution, \ttjournal, \ttkey, \ttlangid, \ttlanguage, \ttmonth, \ttnote, \ttnumber, \ttorganization, \ttpages, \ttpublisher, \ttschool, \ttseries, \tttitle, \tttype, \tturl, \ttvolume, \ttyear, \ttyomi がサポートされています.
フィールドの値は
\begin{verbatim}
    author = {Matsukawa, Yuki and Tsukahara, Takahiro}
    author = "Matsukawa, Yuki and Tsukahara, Takahiro"
\end{verbatim}
のように\verb|{ }|または\verb|" "|で囲います.
フィールドの詳細は以下の通りです(エントリーの分類同様,人によって意見が異なる場合があります).
また,不要なフィールドがあっても無視されるだけなので邪魔であれば消しても構いません.

\subsection{各フィールドの詳細}
\label{ssec:field}
\begin{itemize}
    \item \ttaccess \\
        Webページを閲覧した日付(参照日)を記入します.
        \ttonline でのみ有効です.
        書き方の詳細は第~\ref{ssec:online}節(\ttonline)を参照.
    \item \ttaddress \\
        出版社(\ttpublisher)の住所.
        日本機械学会の原稿テンプレートでは住所を書かないので必要ありません.
    \item \ttarchivePrefix \\
        arXiv上の文献を引用する際に自動で出力されます.
        \jsmefile では\ttarchivePrefix があるとarXivの文献として判断します.
    \item \ttauthor \\
        文献の著者情報を入力します.
        他の \BibTeX スタイルファイルと仕様が異なる場合があるので注意が必要です.
        日本語文献でも英語文献でも
        \begin{verbatim}
        author = {Family, Given and Family, Given and Family, Given}
        author = {Given Family and Given Family and Given Family}
        \end{verbatim}
        の形式で書いてください.
        例えば\citet{Tsukahara:TSFP2005}と\citet{堀本:可視化情報2020}の場合だと
        \begin{verbatim}
        author = {Tsukahara, Takahiro and Seki, Yohji and 
                    Kawamura, Hiroshi and Tochio, Daisuke}
        author = {堀本, 康文 and 川口, 靖夫 and 塚原, 隆裕}
        \end{verbatim}
        のようになります.
        \verb|Family, Given|で書く場合は日本語文献でも半角カンマと半角スペースで姓と名を区切ります.
        また,著者が複数いる場合は\verb| and |で著者を区切ります.
        日本語文献の場合は後述の\ttyomi フィールドで読み方を指定してください.
    \item \ttbooktitle \\
        書籍の名前ですが,\ttconference, \ttincollection, \ttinproceedings で使われることからわかるように,引用する文献が書籍のうちの一部である場合の書籍そのものの題名を書きます.
        例えば,\citet{Lueptow:Springer2000}はそれ単独でStability and experimental velocity field in Taylor--Couette flow with axial and radial flowという題目(\tttitle)を持っていますが,これはPhysics of Rotating Fluidsという書籍(\ttbooktitle)の一部です.
    \item \ttchapter \\
        書籍の一部の章を引用するときに使用します.
        日本機械学会では章番号ではなくページ数で引用するので使う必要はありません.
    \item \ttdoi \\
        文献のデジタルオブジェクト識別子(Digital Object Identifier, DOI)を入力します.
        学術論文だけでなくSpringer等の書籍などにもDOIが割り当てられています.
        \jsmefile では \ttdoi フィールドの値を読み込んで末尾に記載するようにしています.
    \item \ttedition
        (第3版などの)版を入力します.
        \citet{奥村:技評2020}のように\verb|title = {[改訂第8版]\LaTeXe{}美文書作成入門}|と\tttitle の中に書いてしまってもいいと思います.
    \item \tteditor \\
        編者名を入力します.
        書き方は著者名(\ttauthor)と同じですが,\jsmefile では \ttauthor を使用している場合,\tteditor に値を入れても基本的には文献一覧には出力されません.
        ただし,著者(\ttauthor)無し文献の場合,\verb|natbib|による文章中での引用がおかしくなるのでその際は \ttauthor 代わりに \tteditor を使用します(詳細は \ttproceedings を参照).
        例えば,日本機械学会原稿テンプレートでは『伝熱ハンドブック』の編者として日本機械学会を指定し,その後ろに「編」と入れるように書かれています.
        このような場合に \tteditor フィールドを使用してください.
    \item \tteprint \\
        論文のeprintを入力します.
        arXivから \BibTeX 形式で書誌情報を出力するとデフォルトで入ってくるフィールドです.
        \jsmefile では \ttarchivePrefix フィールドがあった場合に \tteprint の情報を記載します(詳細は第~\ref{ssec:misc}節\ttmisc を参照).
    \item \tthowpublished \\
        特殊な出版形態をとる場合の説明を入力します.
        また,第~\ref{ssec:misc}節では学部の卒業論文を \ttmisc に分類する際に \tthowpublished に「xx大学xx学部xx学科卒業論文」と書くことにしています.
    \item \ttinstitution \\
        技術報告書(\tttechreport)でのみ使用されるフィールドです.
        報告書が出された機関名を入力します.
    \item \ttjournal \\
        学術雑誌論文が出された誌名を入力します.
        \ttarticle でのみ有効なフィールドです.
    \item \ttkey \\
        著者名に相当するものが無い場合,ソートに利用します.
        \jsmefile では動作が完璧ではないのであまり使わないでほしいです.
    \item \ttlangid, \ttlanguage \\
        その文献が書かれている言語を指定するフィールドです.
        \verb|japanese|と書かれていると日本語文献として処理されますが,指定されていなくても基本的には自動で日本語文献を判別してくれます\footnote{\ttauthor や \tttitle などのフィールドに日本語が含まれていると\texttt{is.kanji.str\$}という機能で日本語文献と判断します.}.
        \verb|biblatex|では \ttlangid フィールドが使用されるので残しておいてもいいかもしれません.
        \JSMErepos もそのうち\verb|biblatex|対応バージョンを作成したいと考えています.
        また,TeX Live 2020以前の \upBibTeX では\verb|is.kanji.str$|が適切に動作しないため,\ttlangid を使用してください\footnote{詳細は\verb|jecon-bst (GitHub)|\textless \url{https://github.com/ShiroTakeda/jecon-bst}\textgreater の\verb|jecon-example.pdf|を確認してください.}.
    \item \ttmonth \\
        出版された月を入力します.
        日本機械学会では年(\ttyear)だけ書けばいいので \ttmonth は不要.
    \item \ttnote \\
        注記.
        \jsmefile では講演番号や論文番号をここに書いてください.
    \item \ttnumber \\
        雑誌等の号数を入力します.
    \item \ttorganization \\
        会議の主催者・団体やマニュアルを出している機関を入力します.
    \item \ttpages \\
        ページ番号を入力します.
        例えば35ページのみ引用する場合は\verb|pages = {35}|とし,35ページから64ページまでを引用する場合は\verb|pages = {35--64}|とします.
        このとき,開始ページと終了ページを結ぶ横棒はハイフン(\verb|-|)ではなくenダッシュとします(\verb|--|).
        ただし,\jsmefile を使用すれば \ttpages 内でハイフンとしていても自動でenダッシュに変換してくれるので安心です.
    \item \ttpublisher \\
        出版社の名前を入力します.
    \item \ttschool \\
        学校名を入力します.
        \ttmastersthesis と \ttphdthesis でのみ有効なフィールドです.
        学部の卒業論文の場合は \ttmisc に分類し,\ttschool フィールドの代わりに \tthowpublished を使用してください.
    \item \ttseries \\
        書籍のシリーズを入力します.
        \ttbook と \ttinbook でのみ有効なフィールドです.
        日本機械学会では不要かと思います.
    \item \tttitle \\
        文献のタイトルを入力します.
        \jsmefile では標準設定として,英語文献の場合はタイトル冒頭以外のアルファベットを全て小文字に変換して出力します(日本機械学会の原稿テンプレート通りの出力).
        ただし,固有名詞や二次元を表す2Dなどのようにタイトルの途中で大文字を使用する場合は
        \begin{verbatim}
        title = {Subcritical transition of {Taylor--Couette--Poiseuille} flow 
                    at high radius ratio}
        title = {A mathematical consideration of vortex thinning in {2D} turbulence}
        \end{verbatim}
        のように\verb|{ }|で囲めば該当箇所はそのままの形で出力してくれます(\citealp{Matsukawa:PoF2022,Yoneda:arXiv2016}).
    \item \tttype \\
        \tttechreport でのみ有効なフィールドです.
        今後仕様を変えるかもしれませんが,\jsmefile では使わないでください.
    \item \tturl \\
        文献やwebページのURLを入力します.
        \jsmefile では \ttonline でのみ有効なフィールドです.
    \item \ttvolume \\
        雑誌等の巻数(第3巻,Vol.~3など)を入力します.
    \item \ttyear \\
        発行年を入力します.
        学位論文の場合は修了「年」を記入します(第~\ref{ssec:mastersthesis}節 \ttmastersthesis を参照).
    \item \ttyomi \\
        著者(\ttauthor)の読みを入力します.
        \ttyomi の内容から判断してアルファベット順に文献をソートしてくれます.
        \begin{verbatim}
        yomi = {Matsukawa, Yuki and Tsukahara, Takahiro}
        \end{verbatim}
        のようにローマ字で読みを書くと,英語文献と日本語文献を混ぜてアルファベット順でソートしてくれます.
        ひらがなで読みを書くと日本語文献と英語文献が分けられてしまうので日本機械学会の規定に合いません.
\end{itemize}


\subsection{\ttarticle}
\label{ssec:article}
\begin{tcolorbox}[enhanced, title=\ttarticle, drop fuzzy shadow]
    \begin{itemize}
        \item 必須項目 \\
        \ttauthor, \tttitle, \ttjournal, \ttyear
        \item オプション項目 \\
        \ttvolume, \ttnumber, \ttpages, \ttmonth, \ttnote, \ttkey, \ttdoi
        \item 出力(英語文献) \\
            \colorbox[gray]{0.8}{\ttauthorf}, \colorbox[gray]{0.8}{\ttauthors} and \colorbox[gray]{0.8}{\ttauthort}, \colorbox[gray]{0.8}{\tttitle}, \colorbox[gray]{0.8}{\ttjournal} (\colorbox[gray]{0.8}{\ttyear}), Vol.~\colorbox[gray]{0.8}{\ttvolume}, No.~\colorbox[gray]{0.8}{\ttnumber}, pp.~\colorbox[gray]{0.8}{\ttpages}, \colorbox[gray]{0.8}{\ttnote}, DOI: \colorbox[gray]{0.8}{\ttdoi}.
        \item 出力(日本語文献) \\
            \colorbox[gray]{0.8}{\ttauthorf}, \colorbox[gray]{0.8}{\ttauthors}, \colorbox[gray]{0.8}{\ttauthort}, \colorbox[gray]{0.8}{\tttitle}, \colorbox[gray]{0.8}{\ttjournal} (\colorbox[gray]{0.8}{\ttyear}), Vol.~\colorbox[gray]{0.8}{\ttvolume}, No.~\colorbox[gray]{0.8}{\ttnumber}, pp.~\colorbox[gray]{0.8}{\ttpages}, \colorbox[gray]{0.8}{\ttnote}, DOI: \colorbox[gray]{0.8}{\ttdoi}.
        \item \verb|bib|ファイル作成例 \vspace{-3mm}
\begin{verbatim}
@article{Matsukawa:PoF2022,
    author  = {Matsukawa, Yuki and Tsukahara, Takahiro},
    title   = {Subcritical transition of {Taylor--Couette--Poiseuille} flow 
                at high radius ratio},
    journal = {Physics of Fluids},
    volume  = {34},
    number  = {7},
    year    = {2022},
    doi     = {10.1063/5.0096676},
    url     = {https://doi.org/10.1063/5.0096676},
    note    = {074109}
}
\end{verbatim}
    \end{itemize}
\end{tcolorbox}

\ttarticle は雑誌に掲載された論文です.
流体力学分野では英文雑誌だとJournal of Fluid Mechanics\footnote{Journal of Fluid Mechanics, \textless\url{https://www.cambridge.org/core/journals/journal-of-fluid-mechanics}\textgreater}やPhysics of Fluids\footnote{Physics of Fluids, \textless\url{https://pubs.aip.org/aip/pof}\textgreater}などが挙げられます.
国内雑誌だと日本機械学会誌\footnote{日本機械学会誌, \textless\url{https://www.jsme.or.jp/publication/kaisi/}\textgreater}や日本流体力学会誌『ながれ』\footnote{日本流体力学会誌『ながれ』, \textless\url{https://www.nagare.or.jp/publication/nagare.html}\textgreater}などが該当します.
\jsmefile では \ttdoi フィールドに値が入っている場合,文献末尾にDOIを記載するように設定しています.
\ttdoi が無い場合は記載しません.
また,ジャーナルによっては書誌情報を \BibTeX 形式で出力した際に論文番号が \ttpages となっています.
上記のPhysics of Fluidsの場合(\citealp{Matsukawa:PoF2022})だとデフォルトで\verb|pages = {074109}|となっているため,このままだと
\begin{quote}
    Matsukawa, Y. and Tsukahara, T., Subcritical transition of Taylor--Couette--Poiseuille flow at high radius ratio, Physics of Fluids, Vol.~34, No.~7 (2022b), \textcolor{red}{p.~074109}.
\end{quote}
と出力され,違和感があります.
そのため,論文番号や講演番号は \ttnote に入れるようにしましょう.


\subsection{\ttbook}
\label{ssec:book}
\begin{tcolorbox}[enhanced, title=\ttbook, drop fuzzy shadow]
    \begin{itemize}
        \item 必須項目 \\
        \ttauthor / \tteditor, \tttitle, \ttpublisher, \ttyear
        \item オプション項目 \\
        \ttvolume, \ttnumber, \ttseries, \ttaddress, \ttedition, \ttmonth, \ttnote, \ttkey, \ttdoi
        \item 出力(英語文献) \\
            \colorbox[gray]{0.8}{\ttauthorf}, \colorbox[gray]{0.8}{\ttauthors} and \colorbox[gray]{0.8}{\ttauthort}, \colorbox[gray]{0.8}{\tttitle}, \colorbox[gray]{0.8}{\ttpublisher} (\colorbox[gray]{0.8}{\ttyear}), \colorbox[gray]{0.8}{\ttnote}, DOI: \colorbox[gray]{0.8}{\ttdoi}.
        \item 出力(日本語文献) \\
            \colorbox[gray]{0.8}{\ttauthorf}, \colorbox[gray]{0.8}{\ttauthors}, \colorbox[gray]{0.8}{\ttauthort}, \colorbox[gray]{0.8}{\tttitle}, \colorbox[gray]{0.8}{\ttpublisher} (\colorbox[gray]{0.8}{\ttyear}), \colorbox[gray]{0.8}{\ttnote}, DOI: \colorbox[gray]{0.8}{\ttdoi}.
        \item \verb|bib|ファイル作成例 \vspace{-3mm}
\begin{verbatim}
@book{Schmid:Springer2001,
    author      = {Peter J. Schmid and Dan S. Henningson},
    title       = {Stability and Transition in Shear Flows},
    publisher   = {Springer New York},
    year        = {2001},
    doi         = {10.1007/978-1-4613-0185-1}
}
\end{verbatim}
    \end{itemize}
\end{tcolorbox}

出版社が刊行した書籍を引用する際は \ttbook を使います.
似たエントリーとして \ttinbook がありますが,特定のページを参照したのではなく,書籍全体を参照した場合は \ttbook を使いましょう.
\JSMErepos ではSpringer\footnote{Springer, \textless\url{https://www.springer.com/jp}\textgreater}や朝倉書店\footnote{朝倉書店, \textless\url{https://www.asakura.co.jp/}\textgreater},丸善出版\footnote{丸善出版, \textless\url{https://www.maruzen-publishing.co.jp/}\textgreater}等の出版社から出た書籍を \ttbook に分類しています.
\ttpublisher フィールドにはこれらの出版社名を入れましょう.
また,\ttbook では \ttauthor か \tteditor か選べるようになっています.
特定の著者ではなく学会等の編者を記載する場合には \tteditor を使用してください.
\ttauthor と \tteditor の両方を使用した場合は \ttauthor のみ文献一覧に表示されます.
日本機械学会が編集している『JSMEテキストシリーズ』や『伝熱工学資料』などを引用する際は \tteditor を使いましょう.

\subsection{\ttbooklet}
\label{ssec:booklet}
\begin{tcolorbox}[enhanced, title=\ttbooklet, drop fuzzy shadow]
    \begin{itemize}
        \item 必須項目 \\
        \tttitle
        \item オプション項目 \\
        \ttauthor, \tthowpublished, \ttaddress, \ttmonth, \ttyear, \ttnote, \ttkey, \ttdoi
        \item 出力(英語文献) \\
            \colorbox[gray]{0.8}{\ttauthorf}, \colorbox[gray]{0.8}{\ttauthors} and \colorbox[gray]{0.8}{\ttauthort}, \colorbox[gray]{0.8}{\tttitle}, \colorbox[gray]{0.8}{\tthowpublished} (\colorbox[gray]{0.8}{\ttyear}), \colorbox[gray]{0.8}{\ttnote}, DOI: \colorbox[gray]{0.8}{\ttdoi}.
        \item 出力(日本語文献) \\
            \colorbox[gray]{0.8}{\ttauthorf}, \colorbox[gray]{0.8}{\ttauthors}, \colorbox[gray]{0.8}{\ttauthort}, \colorbox[gray]{0.8}{\tttitle}, \colorbox[gray]{0.8}{\tthowpublished} (\colorbox[gray]{0.8}{\ttyear}), \colorbox[gray]{0.8}{\ttnote}, DOI: \colorbox[gray]{0.8}{\ttdoi}.
        \item \verb|bib|ファイル作成例 \vspace{-3mm}
\begin{verbatim}
@booklet{Wang:MEnews2014,
    author          = {Wang, Lin},
    title           = {Exchange student from 
                        {Northwestern Polytechnical University (China)}},
    howpublished    = {ME Newsletter, Department of Mechanical Engineering, 
                        Tokyo University of Science},
    year            = {2014},
    url             = {https://www.rs.tus.ac.jp/me/pdf/newsletter/ME_NL_No15.pdf}
}
\end{verbatim}
    \end{itemize}
\end{tcolorbox}

\ttbooklet は使う機会が少ないため分類が難しいエントリーですが,出版社が明記されていないような(薄い)冊子媒体が該当します.
\JSMErepos では例として東京理科大学理工学部機械工学科(現・創域理工学部機械航空宇宙工学科)が毎年出しているMEニュースレター\footnote{MEニュースレター, \textless\url{https://www.rs.tus.ac.jp/me/newsletter.html}\textgreater}という広報の冊子を引用しました.

\subsection{\ttcomment}
\label{ssec:comment}
\begin{tcolorbox}[enhanced, title=\ttcomment, drop fuzzy shadow]
    \begin{itemize}
        \item 必須項目 \\
        無し
        \item オプション項目 \\
        無し
        \item 出力 \\
        出力されない
        \item \verb|bib|ファイル作成例 \vspace{-3mm}
\begin{verbatim}
@comment{
%%%%%%%%%%%%%%%%%%%
%%%%% 英語文献 %%%%%
%%%%%%%%%%%%%%%%%%%
}
\end{verbatim}
    \end{itemize}
\end{tcolorbox}

\verb|bib|ファイル内でコメントを書く場合に用います.
通常の\verb|tex|ファイルや\verb|sty|ファイル内では\verb|%|を書くことでその行はコメントアウトされますが,\verb|bib|ファイル内では使えません.


\subsection{\ttconference}
\label{ssec:conference}
\ttinproceedings と同様なので省略.
Scribeというシステムとの互換性のために残されているらしい(\citealp{奥村:技評2020}).


\subsection{\ttinbook}
\label{ssec:inbook}
\begin{tcolorbox}[enhanced, title=\ttinbook, drop fuzzy shadow]
    \begin{itemize}
        \item 必須項目 \\
        \ttauthor / \tteditor, \tttitle, \ttchapter / \ttpages, \ttpublisher, \ttyear
        \item オプション項目 \\
        \ttvolume, \ttseries, \ttaddress, \ttedition, \ttmonth, \ttnote, \ttkey, \ttdoi
        \item 出力(英語文献) \\
            \colorbox[gray]{0.8}{\ttauthorf}, \colorbox[gray]{0.8}{\ttauthors} and \colorbox[gray]{0.8}{\ttauthort}, \colorbox[gray]{0.8}{\tttitle}, \colorbox[gray]{0.8}{\ttpublisher} (\colorbox[gray]{0.8}{\ttyear}), \colorbox[gray]{0.8}{\ttpages}, \colorbox[gray]{0.8}{\ttnote}, DOI: \colorbox[gray]{0.8}{\ttdoi}.
        \item 出力(日本語文献) \\
            \colorbox[gray]{0.8}{\ttauthorf}, \colorbox[gray]{0.8}{\ttauthors}, \colorbox[gray]{0.8}{\ttauthort}, \colorbox[gray]{0.8}{\tttitle}, \colorbox[gray]{0.8}{\ttpublisher} (\colorbox[gray]{0.8}{\ttyear}), \colorbox[gray]{0.8}{\ttpages}, \colorbox[gray]{0.8}{\ttnote}, DOI: \colorbox[gray]{0.8}{\ttdoi}.
        \item \verb|bib|ファイル作成例 \vspace{-3mm}
\begin{verbatim}
@inbook{Davidson:Oxford2015,
    author      = {Peter A. Davidson},
    title       = {Turbulence: 
                    An Introduction for Scientists and Engineers, Second Edition},
    publisher   = {Oxford University Press},
    year        = {2015},
    pages       = {61--104}
}
\end{verbatim}
    \end{itemize}
\end{tcolorbox}

\ttbook に似ていますが,\ttbook が書籍丸々一冊なのに対して \ttinbook は書籍中の一部から引用する場合に使用します.
そのため,\ttbook と異なり \ttpages フィールドが使用可能です.
また \ttbook と同様,\ttauthor と \tteditor を選択することができます.
\JSMErepos では例として日本機械学会の『伝熱工学資料』と朝倉書店の『乱流工学ハンドブック』を \tteditor で引用しています.
\tteditor を使用した際の文章中での引用は\citet{JSME:丸善2013}や\citet{笠木:朝倉2009}のようになります.


\subsection{\ttincollection}
\label{ssec:incollection}
\begin{tcolorbox}[enhanced, title=\ttincollection, drop fuzzy shadow]
    \begin{itemize}
        \item 必須項目 \\
        \ttauthor, \tttitle, \ttbooktitle, \ttyear
        \item オプション項目 \\
        \tteditor, \ttpages, \ttorganization, \ttpublisher, \ttaddress, \ttmonth, \ttnote, \ttkey, \ttdoi
        \item 出力(英語文献) \\
            \colorbox[gray]{0.8}{\ttauthorf}, \colorbox[gray]{0.8}{\ttauthors} and \colorbox[gray]{0.8}{\ttauthort}, \colorbox[gray]{0.8}{\tttitle}, \colorbox[gray]{0.8}{\ttbooktitle}, \colorbox[gray]{0.8}{\ttpublisher} (\colorbox[gray]{0.8}{\ttyear}), \colorbox[gray]{0.8}{\ttpages}, \colorbox[gray]{0.8}{\ttnote}, DOI: \colorbox[gray]{0.8}{\ttdoi}.
        \item 出力(日本語文献) \\
            \colorbox[gray]{0.8}{\ttauthorf}, \colorbox[gray]{0.8}{\ttauthors}, \colorbox[gray]{0.8}{\ttauthort}, \colorbox[gray]{0.8}{\tttitle}, \colorbox[gray]{0.8}{\ttbooktitle}, \colorbox[gray]{0.8}{\ttpublisher} (\colorbox[gray]{0.8}{\ttyear}), \colorbox[gray]{0.8}{\ttpages}, \colorbox[gray]{0.8}{\ttnote}, DOI: \colorbox[gray]{0.8}{\ttdoi}.
        \item \verb|bib|ファイル作成例 \vspace{-3mm}
\begin{verbatim}
@incollection{Lueptow:Springer2000,
    author      = {Lueptow, Richard M.},
    title       = {Stability and experimental velocity field 
                    in {Taylor--Couette} flow with axial and radial flow},
    booktitle   = {Physics of Rotating Fluids},
    publisher   = {Springer-Verlag Berlin Heidelberg New York},
    pages       = {137--155},
    year        = {2000},
    doi         = {10.1007/3-540-45549-3}
}
\end{verbatim}
    \end{itemize}
\end{tcolorbox}

\ttincollection は分類が難しいエントリーの一つです.
これは書籍の一部からの引用ですが,\ttinbook と異なる点は,引用箇所が独立して表題を持っているようなものを指します.
学会等があるテーマについて組んだ特集といったイメージです.
上記の\citet{Lueptow:Springer2000}の例では,それ単独でStability and experimental velocity field in Taylor--Couette flow with axial and radial flowという題目(\tttitle)を持っていますが,これはPhysics of Rotating Fluidsという書籍(\ttbooktitle)の一部です.
日本語文献では京都大学数理解析研究所の講究録\footnote{京都大学数理解析研究所(RIMS)講究録, \textless\url{https://www.kurims.kyoto-u.ac.jp/ja/kokyuroku.html}\textgreater}や文部科学省科学研究補助金における特定の新学術領域研究の研究成果報告書等を \ttincollection に分類しています.
それは \ttinproceedings だろとか \tttechreport だろとか言われそうな気もします.

\subsection{\ttinproceedings}
\label{ssec:inproceedings}
\begin{tcolorbox}[enhanced, title=\ttinproceedings, drop fuzzy shadow]
    \begin{itemize}
        \item 必須項目 \\
        \ttauthor, \tttitle, \ttbooktitle, \ttyear
        \item オプション項目 \\
        \tteditor, \ttpages, \ttorganization, \ttpublisher, \ttaddress, \ttmonth, \ttnote, \ttkey, \ttdoi
        \item 出力(英語文献) \\
            \colorbox[gray]{0.8}{\ttauthorf}, \colorbox[gray]{0.8}{\ttauthors} and \colorbox[gray]{0.8}{\ttauthort}, \colorbox[gray]{0.8}{\tttitle}, \colorbox[gray]{0.8}{\ttbooktitle} (\colorbox[gray]{0.8}{\ttyear}), \colorbox[gray]{0.8}{\ttnote}, DOI: \colorbox[gray]{0.8}{\ttdoi}.
        \item 出力(日本語文献) \\
            \colorbox[gray]{0.8}{\ttauthorf}, \colorbox[gray]{0.8}{\ttauthors}, \colorbox[gray]{0.8}{\ttauthort}, \colorbox[gray]{0.8}{\tttitle}, \colorbox[gray]{0.8}{\ttbooktitle} (\colorbox[gray]{0.8}{\ttyear}), \colorbox[gray]{0.8}{\ttnote}, DOI: \colorbox[gray]{0.8}{\ttdoi}.
        \item \verb|bib|ファイル作成例 \vspace{-3mm}
\begin{verbatim}
@inproceedings{Matsukawa:ICFD2022,
    author      = {Matsukawa, Yuki and Tsukahara, Takahiro},
    title       = {Laminarization in Subcritical {Taylor--Couette--Poiseuille} Flow 
                    with Increasing Pressure Gradient},
    booktitle   = {Proceedings of 19th International Conference on Flow Dynamics},
    year        = {2022},
    note        = {OS15-10}
}
\end{verbatim}
    \end{itemize}
\end{tcolorbox}

\ttconference と同様です.
学会等の講演論文集の一部を引用するときに \ttinproceedings を使用します.
\ttarticle や \ttinbook と並んで使用頻度の高いエントリーだと思います.
\ttbooktitle フィールドには日本語なら\texttt{booktitle = \{\colorbox[gray]{0.8}{学会名}講演論文集\}},英語なら\texttt{booktitle = \{Proceedings of \colorbox[gray]{0.8}{学会名}\}}と入力しましょう.
\ttproceedings は講演論文集全体を引用しているのに対して,\ttinproceedings は講演論文集の中の一講演を引用しています.
上の例では \ttnote に講演番号を入れています.


\subsection{\ttmanual}
\label{ssec:manual}
\begin{tcolorbox}[enhanced, title=\ttmanual, drop fuzzy shadow]
    \begin{itemize}
        \item 必須項目 \\
        \tttitle
        \item オプション項目 \\
        \ttauthor, \ttorganization, \ttaddress, \ttedition, \ttmonth, \ttyear, \ttnote, \ttkey, \ttdoi
        \item 出力(英語文献) \\
            \colorbox[gray]{0.8}{\ttauthorf}, \colorbox[gray]{0.8}{\ttauthors} and \colorbox[gray]{0.8}{\ttauthort}, \colorbox[gray]{0.8}{\tttitle} (\colorbox[gray]{0.8}{\ttyear}), \colorbox[gray]{0.8}{\ttnote}, DOI: \colorbox[gray]{0.8}{\ttdoi}.
        \item 出力(日本語文献) \\
            \colorbox[gray]{0.8}{\ttauthorf}, \colorbox[gray]{0.8}{\ttauthors}, \colorbox[gray]{0.8}{\ttauthort}, \colorbox[gray]{0.8}{\tttitle} (\colorbox[gray]{0.8}{\ttyear}), \colorbox[gray]{0.8}{\ttnote}, DOI: \colorbox[gray]{0.8}{\ttdoi}.
        \item \verb|bib|ファイル作成例 \vspace{-3mm}
\begin{verbatim}
@manual{Tecplot2023,
    author  = "{Tecplot, Inc.}",
    title   = {Tecplot 360 Getting Started Manual},
    year    = {2023},
    url    
     = {https://tecplot.azureedge.net/products/360/current/360_getting_started.pdf}
}
\end{verbatim}
    \end{itemize}
\end{tcolorbox}

マニュアルや技術文書は \ttmanual に分類しましょう.
ただし,\ttauthor フィールドが必須項目ではないので,\jsmefile では企業名を \ttauthor に入れています.
これはソートの際に \ttauthor フィールドが空欄だと参考文献一覧の変な場所に配置されてしまうからです.
本来であれば \ttkey フィールドで解決したいところですが,\jsmefile では動作が不安定なので避けたいところです.
実は\verb|natbib.sty|において著者無し文献の本文中での引用は課題が残されているようです\footnote{詳細は\verb|natbib.sty|の公式ドキュメント\textless\url{https://ctan.org/pkg/natbib}\textgreater をご覧ください.}.
\ttmanual での不具合回避方法としては一旦,マニュアルを発行している機関・企業を \ttauthor とすることで解決とします.
著者無し文献の不具合回避方法は第~\ref{ssec:proceedings}節 \ttproceedings でも説明しますが,第~\ref{sec:cite}節 \verb|\citetalias|,\verb|\citepalias| で紹介した方法を使えばユーザーの好きな形で出力することができます.


\subsection{\ttmastersthesis}
\label{ssec:mastersthesis}
\begin{tcolorbox}[enhanced, title=\ttmastersthesis, drop fuzzy shadow]
    \begin{itemize}
        \item 必須項目 \\
        \ttauthor, \tttitle, \ttschool, \ttyear
        \item オプション項目 \\
        \ttaddress, \ttmonth, \ttnote, \ttkey, \ttdoi
        \item 出力(英語文献) \\
            \colorbox[gray]{0.8}{\ttauthorf}, \colorbox[gray]{0.8}{\ttauthors} and \colorbox[gray]{0.8}{\ttauthort}, \colorbox[gray]{0.8}{\tttitle}, Master's thesis, \colorbox[gray]{0.8}{\ttschool} (\colorbox[gray]{0.8}{\ttyear}), \colorbox[gray]{0.8}{\ttnote}, DOI: \colorbox[gray]{0.8}{\ttdoi}.
        \item 出力(日本語文献) \\
            \colorbox[gray]{0.8}{\ttauthorf}, \colorbox[gray]{0.8}{\ttauthors}, \colorbox[gray]{0.8}{\ttauthort}, \colorbox[gray]{0.8}{\tttitle}, \colorbox[gray]{0.8}{\ttschool}修士論文 (\colorbox[gray]{0.8}{\ttyear}), \colorbox[gray]{0.8}{\ttnote}, DOI: \colorbox[gray]{0.8}{\ttdoi}.
        \item \verb|bib|ファイル作成例 \vspace{-3mm}
\begin{verbatim}
@mastersthesis{松川:修論2023,
    author  = {松川, 裕樹},
    yomi    = {Matsukawa, Yuki},
    title   = {直接数値解析を用いた高円筒比
                Taylor--Couette--Poiseuille流の流動状態遷移過程の分類},
    school  = {東京理科大学大学院理工学研究科機械工学専攻},
    year    = {2023}
}
\end{verbatim}
    \end{itemize}
\end{tcolorbox}

修士論文は \ttmastersthesis に分類します.
\verb|@masterthesis|ではなく\texttt{@master\textcolor{red}{s}thesis}です.
\verb|s|を忘れないでください.
また,\ttyear は修了「年度」ではなく修了「年」を西暦で書いてください.
例えば,日本の大学を2023年3月に修了した人は2022年度修了生ですが\verb|year = {2023}|です.

\subsection{\ttmisc}
\label{ssec:misc}
\begin{tcolorbox}[enhanced, title=\ttmisc, drop fuzzy shadow]
    \begin{itemize}
        \item 必須項目 \\
        無し
        \item オプション項目 \\
        \ttauthor, \tttitle, \tthowpublished, \ttarchivePrefix, \tteprint, \ttmonth, \ttyear, \ttnote, \ttkey, \ttdoi
        \item 出力(通常,英語文献) \\
            \colorbox[gray]{0.8}{\ttauthorf}, \colorbox[gray]{0.8}{\ttauthors} and \colorbox[gray]{0.8}{\ttauthort}, \colorbox[gray]{0.8}{\tttitle}, \colorbox[gray]{0.8}{\tthowpublished} (\colorbox[gray]{0.8}{\ttyear}), \colorbox[gray]{0.8}{\ttnote}, DOI: \colorbox[gray]{0.8}{\ttdoi}.
        \item 出力(通常,日本語文献) \\
            \colorbox[gray]{0.8}{\ttauthorf}, \colorbox[gray]{0.8}{\ttauthors}, \colorbox[gray]{0.8}{\ttauthort}, \colorbox[gray]{0.8}{\tttitle}, \colorbox[gray]{0.8}{\tthowpublished} (\colorbox[gray]{0.8}{\ttyear}), \colorbox[gray]{0.8}{\ttnote}, DOI: \colorbox[gray]{0.8}{\ttdoi}.
        \item 出力(arXivの場合) \\
            \colorbox[gray]{0.8}{\ttauthorf}, \colorbox[gray]{0.8}{\ttauthors} and \colorbox[gray]{0.8}{\ttauthort}, \colorbox[gray]{0.8}{\tttitle}, arXiv: \colorbox[gray]{0.8}{\tteprint} (\colorbox[gray]{0.8}{\ttyear}), \colorbox[gray]{0.8}{\ttnote}.
        \item \verb|bib|ファイル作成例(通常) \vspace{-3mm}
\begin{verbatim}
@misc{湯村:卒論2006,
    author          = {湯村, 翼},
    yomi            = {Yumura, Tsubasa},
    title           = {レイリーテイラー不安定による赤道電離圏プラズマバブルの発生},
    howpublished    = {北海道大学理学部地球科学科卒業論文},
    year            = {2006},
    url             = {https://researchmap.jp/yumu/published_papers/1902404}
}
\end{verbatim}
        \item \verb|bib|ファイル作成例(arXivの場合) \vspace{-3mm}
\begin{verbatim}
@misc{Araki:arXiv2023,
    author          = {Araki, Ryo and Bos, Wouter J. T. and Goto, Susumu},
    title           = {Space-local {Navier--Stokes} turbulence}, 
    year            = {2023},
    eprint          = {2308.07255},
    archivePrefix   = {arXiv},
    primaryClass    = {physics.flu-dyn}
}
\end{verbatim}
    \end{itemize}
\end{tcolorbox}

その他該当種別が無いものは \ttmisc とします.
学部の卒業論文は\verb|misc|でいいと思います.
ただし,\ttmastersthesis や \ttphdthesis と異なり,\ttschool のフィールドを使用できないので \tthowpublished で代用します.
したがって,\ttmastersthesis や \ttphdthesis では \ttschool に所属名だけ(例:\verb|school = {東京理科大学大学院理工学研究科機械工学専攻}|)書けばよかったものが \ttmisc で卒論を出力する際には\verb|howpublished = {北海道大学理学部地球科学科卒業論文}|のように「\verb|卒業論文|」の文字まで書く必要があります.
該当するエントリーがよくわからなかったらとりあえず \ttmisc に入れておくという人は多いと思います.
また,arXiv\footnote{arXiv(「アーカイブ」と読みます), \textless\url{https://arxiv.org/}\textgreater}と呼ばれるプレプリントサーバーから引用した文献は \ttmisc に分類します.
arXiv上のExport BibTeX Citationと書いてあるところから文献情報を見ると \ttmisc に分類されていることがわかると思います.
この文献情報では上記のように\verb|eprint = {2308.07255}|, \verb|archivePrefix = {arXiv}|などと書かれていることが多いです.
\jsmefile では \ttarchivePrefix フィールドがあるとこの文献がarXiv上の文献だと認識してくれて \tteprint の情報と合わせて
\begin{quote}
    Araki, R., Bos, W. J. T. and Goto, S., Space-local Navier--Stokes turbulence, arXiv: 2308.07255 (2023).
\end{quote}
のように自動で書いてくれます.

また,\ttmisc では \ttauthor フィールドが任意項目となっています.
第~\ref{ssec:manual}節 \ttmanual でも説明したように,著者無し文献は不具合の元なので著者に相当しそうなものを \ttauthor に入れておいてください.

\subsection{\ttonline}
\label{ssec:online}
\begin{tcolorbox}[enhanced, title=\ttonline, drop fuzzy shadow]
    \begin{itemize}
        \item 使用可能項目 \\
        \ttauthor, \tttitle, \tthowpublished, \ttmonth, \ttyear, \tturl, \ttdoi, \ttaccess, \ttnote
        \item 出力(英語文献) \\
            \colorbox[gray]{0.8}{\ttauthorf}, \colorbox[gray]{0.8}{\ttauthors} and \colorbox[gray]{0.8}{\ttauthort}, \colorbox[gray]{0.8}{\tttitle}, available from \textless\colorbox[gray]{0.8}{\tturl}\textgreater, (accessed on \colorbox[gray]{0.8}{access}).
        \item 出力(日本語文献) \\
            \colorbox[gray]{0.8}{\ttauthorf}, \colorbox[gray]{0.8}{\ttauthors}, \colorbox[gray]{0.8}{\ttauthort}, \colorbox[gray]{0.8}{\tttitle}, \textless\colorbox[gray]{0.8}{\tturl}\textgreater, (参照日 \colorbox[gray]{0.8}{access}).
        \item \verb|bib|ファイル作成例 \vspace{-3mm}
\begin{verbatim}
@online{Kawamura_Ret64,
    author  = "{Kawamura Laboratory}",
    title   = {{DNS} Database of Wall Turbulence and Heat Transfer: 
                Text database of {Poiseuille} flow for $\mathit{Re}_\tau = 64$},
    year    = {},
    url     = {https://www.rs.tus.ac.jp/~t2lab/db/index.html},
    access  = {10 October, 2023}
}
\end{verbatim}
    \end{itemize}
\end{tcolorbox}

\ttonline は \jsmefile 独自のエントリーなので他の \BibTeX スタイルファイルを用いるときには注意してください(\jsmefile の元となった\verb|jecon.bst|では使用できます).
本来,webページ等の引用はあまり推奨されるものではありませんが,データベースを研究室のwebページ等で公開していることがある\footnote{乱流の分野におけるデータベースとしては,東京理科大学河村研究室(現在は塚原研究室が管理) \textless\url{https://www.rs.tus.ac.jp/~t2lab/db/index.html}\textgreater や東京大学笠木研究室(現在は複数の大学によって管理) \textless\url{https://thtlab.jp}\textgreater などが挙げられます.}ので使う機会がゼロとは言えないでしょう.
英語のwebページの場合は
\renewcommand\UrlFont{\rmfamily}
\begin{quote}
    Kawamura Laboratory, DNS database of wall turbulence and heat transfer: Text database of Poiseuille flow for $\mathit{Re}_\tau = 64$, available from \textless\url{https://www.rs.tus.ac.jp/~t2lab/db/index.html}\textgreater, (accessed on 10 October, 2023).
\end{quote}
\renewcommand\UrlFont{\ttfamily}
のように,末尾にwebページのURLと参照日を入れることが日本機械学会の原稿テンプレートに記載されています.
英語での参照日の書き方は\verb|access  = {10 October, 2023}|のように日,月(アルファベット),年の順です.
日本語のwebページの場合は
\renewcommand\UrlFont{\rmfamily}
\begin{quote}
    立川裕二, 博士論文執筆の際にお願いしたいこと, \textless\url{https://member.ipmu.jp/yuji.tachikawa/misc/dron.html}\textgreater, (参照日 2023年10月10日).
\end{quote}
\renewcommand\UrlFont{\ttfamily}
のように出力します.
また,他の文献と異なり発行年の \ttyear を出力する必要はありませんが,\jsmefile の仕様上,\ttyear フィールドを使わないとエラーが発生してしまいます.
そのため,上記サンプルコードのように \verb|year = {}| と値を空にすることで対処してください.

\subsection{\ttphdthesis}
\label{ssec:phdthesis}
\begin{tcolorbox}[enhanced, title=\ttphdthesis, drop fuzzy shadow]
    \begin{itemize}
        \item 必須項目 \\
        \ttauthor, \tttitle, \ttschool, \ttyear
        \item オプション項目 \\
        \ttaddress, \ttmonth, \ttnote, \ttkey, \ttdoi
        \item 出力(英語文献) \\
            \colorbox[gray]{0.8}{\ttauthorf}, \colorbox[gray]{0.8}{\ttauthors} and \colorbox[gray]{0.8}{\ttauthort}, \colorbox[gray]{0.8}{\tttitle}, Ph.D. dissertation, \colorbox[gray]{0.8}{\ttschool} (\colorbox[gray]{0.8}{\ttyear}), \colorbox[gray]{0.8}{\ttnote}, DOI: \colorbox[gray]{0.8}{\ttdoi}.
        \item 出力(日本語文献) \\
            \colorbox[gray]{0.8}{\ttauthorf}, \colorbox[gray]{0.8}{\ttauthors}, \colorbox[gray]{0.8}{\ttauthort}, \colorbox[gray]{0.8}{\tttitle}, \colorbox[gray]{0.8}{\ttschool}博士論文 (\colorbox[gray]{0.8}{\ttyear}), \colorbox[gray]{0.8}{\ttnote}, DOI: \colorbox[gray]{0.8}{\ttdoi}.
        \item \verb|bib|ファイル作成例 \vspace{-3mm}
\begin{verbatim}
@phdthesis{塚原:博論2007,
    author  = {塚原, 隆裕},
    yomi    = {Tsukahara, Takahiro},
    title   = {大規模直接数値シミュレーションによる低レイノルズ数平行平板間乱流の研究},
    school  = {東京理科大学大学院理工学研究科機械工学専攻},
    year    = {2007},
    url     = {https://iss.ndl.go.jp/books/R100000002-I000009177724-00}
}
\end{verbatim}
    \end{itemize}
\end{tcolorbox}

\ttphdthesis は博士論文が該当します.
エントリー名に\verb|thesis|と入っていますが,英語文献の場合はdissertationと出力されます.
基本的な使い方は \ttmastersthesis と同じです.

\subsection{\ttproceedings}
\label{ssec:proceedings}
\begin{tcolorbox}[enhanced, title=\ttproceedings, drop fuzzy shadow]
    \begin{itemize}
        \item 必須項目 \\
        \tttitle, \ttyear
        \item オプション項目 \\
        \tteditor, \ttorganization, \ttpublisher, \ttaddress, \ttmonth, \ttnote, \ttkey, \ttdoi
        \item 出力 \\
            \colorbox[gray]{0.8}{\tttitle} (\colorbox[gray]{0.8}{\ttyear}), \colorbox[gray]{0.8}{\ttnote}, DOI: \colorbox[gray]{0.8}{\ttdoi}.
        \item \verb|bib|ファイル作成例 \vspace{-3mm}
\begin{verbatim}
@proceedings{THMT2023,
    editor  = {THMT},
    title   = "{Proceedings of 10th International Symposium 
                on Turbulence, Heat and Mass Transfer}",
    yomi    = {Proceedings},
    year    = {2023}
}
\end{verbatim}
    \end{itemize}
\end{tcolorbox}

学会の講演論文集全体を引用する際には \ttproceedings を使用します.
\ttconference や \ttinproceedings は講演論文集の中の一講演を引用しているのに対して,\ttproceedings は講演論文集全体を引用しています.
\ttmanual や \ttmisc と同様,\ttauthor フィールドが必須項目ではありません.
したがって,文献一覧のソートの際は \tttitle の読みを和文・英文問わず \ttyomi フィールドで指定してあげないと変な場所に飛ばされてしまいます.
また,もし
\begin{quote}
\begin{verbatim}
@proceedings{THMT2023,
    title   = "{Proceedings of 10th International Symposium 
                on Turbulence, Heat and Mass Transfer}",
    yomi    = {Proceedings},
    year    = {2023}
}
@proceedings{流力年会2023,
    title   = {日本流体力学会年会2023講演論文集},
    yomi    = {Nihonryuutairikigakkainenkai2023},
    year    = {2023}
}
\end{verbatim}
\end{quote}
のように\verb|bib|ファイルに書いていた場合,文中で引用した際に以下のようにおかしな表示となってしまいます.
\begin{table}[h]
    \centering
    \begin{tabular}{ll}
        コマンド    &出力結果 \\
        \verb|\citet{THMT2023}| &THM (2023) \\
        \verb|\citet{流力年会2023}| &流 (2023) \\
    \end{tabular}
\end{table}
そのため,\tteditor フィールドに「引用時に著者代わりに出力したい文字列」を書いておきましょう.
文献一覧には \tteditor は出力されませんが,本文中での引用では\citet{THMT2023}や\citet{流力年会2023}のように\verb|editor (year)|で出力してくれます.
これらの出力形式が気に入らない場合は,第~\ref{sec:cite}節で紹介した \verb|\citetalias| や \verb|\citepalias| を使って引用することで任意の出力形式を得られます.


\subsection{\tttechreport}
\label{ssec:techreport}
\begin{tcolorbox}[enhanced, title=\tttechreport, drop fuzzy shadow]
    \begin{itemize}
        \item 必須項目 \\
        \ttauthor, \tttitle, \ttinstitution, \ttyear
        \item オプション項目 \\
        \tttype, \ttnumber, \ttaddress, \ttmonth, \ttnote, \ttkey, \ttdoi
        \item 出力(英語文献) \\
            \colorbox[gray]{0.8}{\ttauthorf}, \colorbox[gray]{0.8}{\ttauthors} and \colorbox[gray]{0.8}{\ttauthort}, \colorbox[gray]{0.8}{\tttitle}, \colorbox[gray]{0.8}{\ttinstitution} (\colorbox[gray]{0.8}{\ttyear}), \colorbox[gray]{0.8}{\ttnote}, DOI: \colorbox[gray]{0.8}{\ttdoi}.
        \item 出力(日本語文献) \\
            \colorbox[gray]{0.8}{\ttauthorf}, \colorbox[gray]{0.8}{\ttauthors}, \colorbox[gray]{0.8}{\ttauthort}, \colorbox[gray]{0.8}{\tttitle}, \colorbox[gray]{0.8}{\ttinstitution} (\colorbox[gray]{0.8}{\ttyear}), \colorbox[gray]{0.8}{\ttnote}, DOI: \colorbox[gray]{0.8}{\ttdoi}.
        \item \verb|bib|ファイル作成例 \vspace{-3mm}
\begin{verbatim}
@techreport{Neuhart:NASAreport2004,
    author      = {Neuhart, Dan H. and McGinley, Catherine B.},
    title       = {Free-Stream Turbulence Intensity in the {Langley} 
                    14- by 22-Foot Subsonic Tunnel},
    institution = {NASA Technical Publication},
    year        = {2004},
    url         = {https://ntrs.nasa.gov/citations/20040120956},
    note        = {TP-2004-213247}
}
\end{verbatim}
    \end{itemize}
\end{tcolorbox}

研究機関等から発行された技術報告書は \tttechreport に分類します.
技術報告書を発行している研究機関はさまざまありますが,例えばNASA\footnote{アメリカ航空宇宙局(NASA) Technical Reports Server, \textless\url{https://ntrs.nasa.gov/}\textgreater}や国立天文台\footnote{大学共同利用機関法人 自然科学研究機構 国立天文台,国立天文台欧文報告, \textless\url{https://www.nao.ac.jp/about-naoj/reports/publications-naoj.html}\textgreater},鉄道総研\footnote{公益財団法人 鉄道総合技術研究所(鉄道総研),鉄道総研報告, \textless\url{https://www.rtri.or.jp/publish/rtrirep/}\textgreater}などが挙げられます.
また,企業によっては技術報告書を公開しているところもあります.
\tttechreport には \tttype という任意フィールドがありますが,諸事情でここに何か書いても何も表示しないようにしているので使わないでください.
本当は\verb|type = {Technical Report}|などと書くとそのように表示されます.

\subsection{\ttunpublished}
\label{ssec:unpublished}
\begin{tcolorbox}[enhanced, title=\ttunpublished, drop fuzzy shadow]
    \begin{itemize}
        \item 必須項目 \\
        \ttauthor, \tttitle, \ttnote
        \item オプション項目 \\
        \ttmonth, \ttyear, \ttkey, \ttdoi
        \item 出力(英語文献) \\
            \colorbox[gray]{0.8}{\ttauthorf}, \colorbox[gray]{0.8}{\ttauthors} and \colorbox[gray]{0.8}{\ttauthort}, \colorbox[gray]{0.8}{\tttitle}, (\colorbox[gray]{0.8}{\ttyear}), \colorbox[gray]{0.8}{\ttnote}, DOI: \colorbox[gray]{0.8}{\ttdoi}.
        \item 出力(日本語文献) \\
            \colorbox[gray]{0.8}{\ttauthorf}, \colorbox[gray]{0.8}{\ttauthors}, \colorbox[gray]{0.8}{\ttauthort}, \colorbox[gray]{0.8}{\tttitle}, (\colorbox[gray]{0.8}{\ttyear}), \colorbox[gray]{0.8}{\ttnote}, DOI: \colorbox[gray]{0.8}{\ttdoi}.
        \item \verb|bib|ファイル作成例 \vspace{-3mm}
\begin{verbatim}
@unpublished{Dunkel:MITOCW2015,
    author  = {Dunkel, J\:{o}rn},
    title   = {Nonlinear Dynamics {II}: Continuum Systems, 
        Linear Stability Analysis and Pattern Formation, {MIT Open Course Ware}},
    year    = {2015},
    url     = {https://ocw.mit.edu/courses/
        18-354j-nonlinear-dynamics-ii-continuum-systems-spring-2015/
        resources/mit18_354js15_ch7/}
}
\end{verbatim}
    \end{itemize}
\end{tcolorbox}

\ttunpublished は正式には出版されていない本などが該当します.
ただ,\ttunpublished を使う機会はほぼ無いと思われますし,そもそも具体的に何が該当するのかもパッとイメージできるものではありません.
また,大抵の人は分類がよくわからなかったら \ttmisc に入れてしまう気がします.
しかし,\jsmefile を作成するうえでパターンを網羅しておきたかったので(無理矢理)分類したものとしては大学のオープンコースウェア(OCW)の講義資料等です.
講義資料であれば正式に出版された書籍ではありませんがweb上にあるのをよく見かけると思います.
上記の例はMITのOCW\footnote{MIT Open Course Ware, \textless\url{https://ocw.mit.edu/}\textgreater}における講義資料です.
